Алгоритм AdaBoost (сокр. от adaptive boosting)~\cite{ada_boost} является мета-алгоритмом, в процессе обучения 
строит композицию из базовых алгоритмов обучения для улучшения их эффективности.

Достоинства:
\begin{itemize}
 \item хорошая обобщающая способность --- в реальных задачах (не всегда, но часто) удаётся строить композиции, 
превосходящие по качеству базовые алгоритмы, при этом обобщающая способность может улучшаться (в некоторых задачах) 
по мере увеличения числа базовых алгоритмов;
 \item простота реализации;
 \item время построения композиции практически полностью определяется временем обучения базовых алгоритмов.
\end{itemize}


Недостатки алгоритма классификаций AdaBoost:
\begin{itemize}
 \item склонен к переобучению при наличии значительного уровня шума в данных;
 \item требует достаточно длинных обучающих выборок;
 \item бустинг может приводить к построению громоздких композиций, состоящих из сотен алгоритмов, 
 такие композиции исключают возможность содержательной интерпретации, требуют больших объёмов памяти 
 для хранения базовых алгоритмов и существенных затрат времени на вычисление классификаций.
\end{itemize}