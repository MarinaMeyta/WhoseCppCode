Алгоритм AdaBoost (сокр. от adaptive boosting) является мета-алгоритмом, в процессе обучения 
строит композицию из базовых алгоритмов обучения для улучшения их эффективности.

Достоинства:
\begin{itemize}
 \item Хорошая обобщающая способность. В реальных задачах (не всегда, но часто) удаётся строить композиции, 
превосходящие по качеству базовые алгоритмы. Обобщающая способность может улучшаться (в некоторых задачах) 
по мере увеличения числа базовых алгоритмов.
 \item Простота реализации.
 \item Время построения композиции практически полностью определяется временем обучения базовых алгоритмов.
\end{itemize}


Недостатки алгоритма классификаций AdaBoost:
\begin{itemize}
 \item Cклонен к переобучению при наличии значительного уровня шума в данных.
 \item Требует достаточно длинных обучающих выборок.
 \item Бустинг может приводить к построению громоздких композиций, состоящих из сотен алгоритмов. 
 Такие композиции исключают возможность содержательной интерпретации, требуют больших объёмов памяти 
 для хранения базовых алгоритмов и существенных затрат времени на вычисление классификаций.~\cite{ada_boost}
\end{itemize}