\newpage
\ESKDthisStyle{empty}
\paragraph{\hfill ABSTRACT \hfill}
Explanatory note contains \ESKDtotal{page} pages, \ESKDtotal{figure} pictures, \ESKDtotal{table} tables,
\ESKDtotal{bibitem} sources, \ESKDtotal{appendix} appendicies.

STYLOMETRY, SOURCE CODE, AUTHORSHIP ATTRIBUTION, C/C++, CLASSIFICATION, PYTHON, SKLEARN,
JUPYTER NOTEBOOK, DECISION TREES, RANDOM FOREST CLASSIFIER, CROSS-VALIDATION, 
ADA BOOST, EXTREMLY RANDOMIZED TREES, GITHUB, LATEX.

The aim of this work is a software development of the tool for deanonimization of programmers, 
based on stylometry analysis of source code written in C/C++ programming language for future
usage in information security field: copyright/copyleft research, cybercrime investigation and patent licensing.

This specialist work solves the following tasks:
\begin{itemize}
  \item review of existing research, development, methods of stylometric analysis of source code;
  \item building a model for the process of determining the authorship of source code;
  \item software development of the system for source code analysis based on stylometry;
  \item development of software interface;
  \item preparation and processing of a test data set;
  \item study of the effectiveness of the developed program based on the analysis model of source codes;
  \item analysis of results.
\end{itemize}

Research object: deanonimization of the software developers.

Research subject: source code authorship attribution based on stylometry methods.

Achieved results: the main result of this work is developed software for C/C++ source code 
authorship attribution called <<WhoseCppCode>>, which is described in this paper.

The report is made in accordance with the OS TUSUR 01-2013~\cite{ostusur} using a word processor \LaTeX.
