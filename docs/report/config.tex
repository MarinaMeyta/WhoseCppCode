\documentclass[russian,12pt,floatsection,nocolumnsxix]{eskdtext}


% закомментировать для рамок 
\usepackage[numbertop, numbercenter]{eskdplain}

\usepackage[utf8x]{inputenc}

\usepackage{setspace}
\onehalfspacing % полуторный интервал для всего текста

% - Подключаем шрифты из пакета scalable-cyrfonts-tex
\usepackage{cyrtimes}

% - Отступ красной строки
\setlength{\parindent}{1.25cm}

% - Убирает точку в списке литературы
\makeatletter
\def\@biblabel#1{#1 }

% ограничение для оглавления
%\usepackage{tocvsec2}
\setcounter{tocdepth}{2}

% - Точки для всех пунктов в оглавлении
\renewcommand*{\l@section}{\@dottedtocline{1}{1.5em}{2.3em}}
\renewcommand*{\l@subsection}{\@dottedtocline{1}{1.5em}{2.3em}}
\renewcommand*{\l@subsubsection}{\@dottedtocline{1}{1.5em}{2.3em}}

% - Для переопределения списков
\renewcommand{\theenumi}{\arabic{enumi}}
\renewcommand{\labelenumi}{\theenumi)}
\makeatother

\usepackage{enumitem}
\setlist{nolistsep, itemsep=0.3cm,parsep=0pt}

% - ГОСТ списка литературы
\bibliographystyle{utf8gost705u}


% - Верикальные отступы заголовков 
\ESKDsectSkip{section}{1em}{1em}
\ESKDsectSkip{subsection}{1em}{1em}
\ESKDsectSkip{subsubsection}{1em}{1em}

% - Изменение заголовков
\usepackage{titlesec}
\titleformat{\section}{\normalfont\normalsize\centering}{\thesection}{1.0em}{}
\titleformat{\subsection}{\normalfont\normalsize\centering}{\thesubsection}{1.0em}{}
\titleformat{\subsubsection}{\normalfont\normalsize\centering}{\thesubsubsection}{1.0em}{}
\titleformat{\paragraph}{\normalfont\normalsize\centering}{\theparagraph}{1.0em}{}

% - Оставим место под ТЗ 
\setcounter{page}{1}

% - Для больших таблиц
\usepackage{longtable}
\usepackage{tabularx}
\renewcommand{\thetable}{\thesection.\arabic{table}}

% - Используем графику в документе
\usepackage{graphicx}
\graphicspath{{images/}}
\renewcommand{\thefigure}{\thesection.\arabic{figure}}

% - Счётчики
\usepackage{eskdtotal}

% - Выравнивание по ширине
\sloppy

% - Разрешить перенос двух последних букв слова
\righthyphenmin=2

% - Оформление списков
\RequirePackage{enumitem}
\renewcommand{\alph}[1]{\asbuk{#1}}
\setlist{nolistsep}
\setitemize[1]{label=--, fullwidth, itemindent=\parindent, 
  listparindent=\parindent}% для дефисного списка
\setitemize[2]{label=--, fullwidth, itemindent=\parindent, 
  listparindent=\parindent, leftmargin=\parindent}
\setenumerate[1]{label=\arabic*), fullwidth, itemindent=\parindent, 
  listparindent=\parindent}% для нумерованного списка
\setenumerate[2]{label=\alph*), fullwidth, itemindent=\parindent, 
  listparindent=\parindent, leftmargin=\parindent}% для списка 2-ой ступени, который будет нумероваться а), б) и т.д.
  
% - Оформляем листинг кода (не использовать комментарии на русском!)
\usepackage{listings}  
\lstset{basicstyle=\ttfamily\scriptsize}
\lstset{extendedchars=\true}

% - выводим текст как есть с размером шрифта scriptsize
\makeatletter
\def\verbatim{\scriptsize\@verbatim \frenchspacing\@vobeyspaces \@xverbatim}
\makeatother

% - Вставка pdf
\usepackage[enable-survey]{pdfpages}

%межстрочный интервал
\usepackage{setspace}
\linespread{1.5}

%фамилии для рамок
\author{\ESKDfontII Мейта М.В.}
\ESKDchecker{\ESKDfontII Романов А.С.}
\ESKDnormContr{\ESKDfontII Якимук А.Ю.}
\ESKDapprovedBy{\ESKDfontII Шелупанов А.А.}
\ESKDcolumnI{\ESKDfontIII Определение авторства исходного кода}
\ESKDcolumnIX{\ESKDfontIII ТУСУР, ФБ, каф.~КИБЭВС, гр.~722}
\ESKDsignature{КИБЭВС.501410.001 ПЗ}