Согласно статье~\cite{caliskan}, обучение классификатора необходимо производить на программных файлах, решающих схожие задачи, для повышения точности классификации. Данное условие необходимо для того, чтобы раделять файлы исходного кода по стилю программирования, а не по функциональному назначению.

В качестве тестовых данных для обучения и классификации использовались лабораторные работы трех студентов первого курса обучения по предмету <<Основы программирования>>. При этом каждым студентом было выполнено 7 лабораторных работ, в каждой из которых по 2 задачи на программирование на языке С/С++. Таким образом, выборка состояла из 14 примеров (файлов исходного кода) на каждого из трех объектов (авторов), что в общей сложности составило 42 файла формата *.cpp. 

Тестовая и обучающая выборки генерировались случайным образом без повторений из начальной выборки, описанной выше. При этом четверть всех примеров использовалась для тестирования (классификации). 
