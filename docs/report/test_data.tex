Burrows в работе~\cite{burrows_big} выделяет следующие ключевые параметры тестовых данных,
которые могут влиять на точность классификации:

\begin{itemize}
 \item Число авторов --- с увеличением числа авторов сложность классификации увеличивается, 
 точность --- снижается.
 \item Число экземпляров выборки для каждого автора --- желательно соблюдать одинаковым для всех авторов во 
 избежание отклонения в сторону наиболее точно описанных авторов, а также иметь больше экземпляров 
 для увеличения размера тестовой выборки. 
 \item Средняя длина образца кода (количество непустых строк кода) --- чем длиннее, тем выше точность
 классификации. Изменение длины экземпляров выборки может влиять на отклонение в сторону наиболее 
 точно описанных авторов, однако не представляется возможным соблюдать длину экземпляра выборки постоянной.
 \item <<Стилистическая зрелость>> (stylistic maturity) авторов --- уровень квалификации, 
 личные и профессиональные предпочтения в стиле написания программ.
 \item Временные метки образцов кода (подразумевается, что со временем программы устаревают, технологии
 и методы программирования меняются и, как следствие, изменяется стиль программирования).
 \item Репрезентативность выборки --- демографические, социальные и другие факторы.
 \item Типы авторов --- студент, фрилансер, профессиональный разработчик. 
 В идеале система должна включать в себя разные типы.
 \item Языки программирования --- если тестировать несколько языков одновременно, 
 результат будет зависеть от характерных признаков языка.
 \item Авторство в одном лице --- большинство проектов выполняются в 
 сотрудничестве с другими разработчиками.
 \item Корректное авторство --- без плагиата, копирования и т.п.
\end{itemize}

Burrows упоминает также от том, что характерный стиль программирования нестабилен в начале карьеры программиста,
что может существенно отличать начинающего специалиста и профессионала разработки.

Программное обеспечение <<WhoseCppCode>> тестировалось на трех наборах данных:

\begin{enumerate}
 \item <<Students>> --- выборка представляет собой работы студентов первого курса обучения по 
 предмету <<Основы программирования>>. Все программы реализуют решения однотипных задач в рамках учебной 
 дисциплины, что исключет их разделение при классификации по функциональному назначению вместо 
 стилистических особенностей и снижение точности классификации. 
 \item <<Google Code Jam>> --- общедоступные данные ежегодной международной олимпиады по программированию Google Code Jam 
 2016.~\cite{GoogleCodeJam} Так же, как и в первой выборке, авторы решали схожие задачи, используя различные
 подходы и алгоритмы. 
\item <<GitHub>> --- данные, собранные с сайта GitHub~\cite{GitHub} --- крупнейшего~\cite{GH_domain} 
веб-сервиса для хостинга IT-проектов и их совместной разработки. 
\end{enumerate}


Сбор данных с веб-хостинга GitHub производился по следующему принципу: 
\begin{enumerate}
 \item Выбирались крупные open-source репозитории (удаленные хранилища программного кода и данных),  
 посвященные разработке проектов на C/C++.
 \item Просматривался список контрибьюторов (пользователей, вносивших изменения в проект).
 \item В качестве авторов выбирались те контрибьюторы, у которых имеются личные проекты, написанные 
 на C/C++.
 \item На основе списка пользователей автоматически, средствами программы <<WhoseCppCode>>, производился
 сбор и сохрание файлов исходного кода для каждого автора. 
\end{enumerate}

В таблице~\ref{tab:data} приводится описание некоторых характеристик каждого набора данных.
В данном случае под смешанным типом авторов подразумевается, что разработчики могли быть совершенно
разного уровня квалификации и рода деятельности (студенты, фрилансеры, начинающие и 
профессиональные разработчики, программисты-любители и т.д.).

\begin{table}[h!]
\caption{ Тестовые данные }
\label{tab:data}
\begin{center}
\begin{tabularx}{\linewidth}{|X|X|X|X|X|X|}
\hline
Набор данных & <<Students>> & <<Google Code Jam>> & <<GitHub>> \\
\hline
Число авторов & 3 & 30 & 30 \\
\hline
исло файлов исходного кода на одного автора & 14 & 9 & 78 \\
\hline
Всего файлов исходного кода & 42 & 278 & 2334 \\
\hline
Минимальное число строк кода & 33 & 36 & 26 \\
\hline
Максимальное число строк кода & 160 & 461 & 16348 \\
\hline
Среднее число строк кода на один файл исходного кода & 45 & 87 & 234 \\
\hline
Тип авторов & Студенты & Смешанный & Смешанный \\
\hline
\end{tabularx}
\end{center}
\end{table}


Каждая выборка представляет собой совокупность файлов исходного кода программ на языке C/C++ с расширениями
*.cpp, *.c, *.h, *.hpp, *.cxx, *.cc, *.ii, *.ixx, *.ipp, *.inl, *.txx, *.tpp, *.tpl.

Тестовая и обучающая выборки генерировались случайным образом без повторений из начальной выборки, 
описанной выше. При этом четверть всех примеров использовалась для тестирования (классификации), а также
использовалась процедура скользящего контроля --- 10-фолдовая кросс-валидация. Более подробное 
описание процедуры рзбиения данных и тестирования обучаемой модели содержится в разделе~\label{classifiers}.



