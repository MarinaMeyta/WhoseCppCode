\subsection{Анализ опасных и вредных производственных факторов на рабочем месте}





\subsubsection{Перечень продукции и контролируемые гигиенические параметры вредных и опасных факторов}


СанПиН 2.2.2/2.4.1340-03

Перечень продукции и контролируемых гигиенических параметров приведен в таблице~\ref{tab:life_1}.

\begin{table}[h!]
\caption{ Перечень продукции, контролируемых гигиенических параметров }
\label{tab:life_1}
\begin{center}
\begin{tabularx}{\linewidth}{|X|X|}
\hline
Вид продукции & Контролируемые гигиенические параметры\\
\hline
\item Машина вычислительная электронная цифровая персональная (ПЭВМ) & 
\begin{itemize}
 \item уровни электромагнитных полей (ЭМП);
 \item уровни акустического шума;
 \item концентрация вредных веществ в воздухе;
 \item визуальные показатели видеодисплейного терминала
\end{itemize}									     
\\
\hline
\item Устройства периферийные: модем, клавиатура, принтер, устройства хранения информации & 
\begin{itemize}
 \item уровни ЭМП;
 \item уровни акустического шума;
 \item концентрация вредных веществ в воздухе
\end{itemize}
 \\
\hline
\end{tabularx}
\end{center}
\end{table}




\subsubsection{Требования к уровням шума на рабочих местах, оборудованных ПЭВМ}


СанПиН 2.2.2/2.4.1340-03

\begin{table}[h!]
\caption{ Допустимые значения уровней звукового давления }
\label{tab:life_1}
\begin{center}
\begin{tabularx}{\linewidth}{|X|X|X|X|X|X|X|X|X|X|}
\hline
\multicolumn{9}{|c|}{Уровни звукового давления в октавных полосах со среднегеометрическими частотами} & \multirow{2}{\hsize}{Уровни звука в дБА}\\
\cline{1-9}
31,5 Гц & 63 Гц & 125 Гц & 250 Гц & 500 Гц & 1000 Гц & 2000 Гц & 4000 Гц & 8000 Гц & \\
\hline
31,5 Гц & 63 Гц & 125 Гц & 250 Гц & 500 Гц & 1000 Гц & 2000 Гц & 4000 Гц & 8000 Гц & 50\\
\hline
\end{tabularx}
\end{center}
\end{table}


% \multirow{2}{\hsize}{Уровни звука в дБА}


\subsubsection{Требования к уровням электромагнитных полей на рабочих местах, оборудованных ПЭВМ}

\subsubsection{Требования к визуальным параметрам устройств отображения информации}

\subsubsection{Требования к микроклимату. Концентрации вредных веществ, выделяемых ПЭВМ в воздух помещения}

\subsubsection{Требования к освещенности}

\subsubsection{Требования к организации рабочих мест пользователей ПЭВМ}


\subsection{Инструкция по работе на персональном компьютере}

\subsubsection{Противопожарная безопасность}

\subsubsection{Электрическая безопасность}

\subsubsection{Требования охраны труда во время работы. Оказание первой медицинской помощи}