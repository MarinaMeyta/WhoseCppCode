\subsection{Анализ опасных и вредных производственных факторов на рабочем месте}

В ходе трудового процесса организм человека может подвергаться различным воздействиям, 
оказывающим влияние на его здоровье и работоспособность. Подобное воздействие может приводить к 
различным результирующим последствиям, которые зависят от характера воздействия (прямого или
опосредованного), наличия тех или иных факторов производственной среды, а также условий
их проявления. 

Принято разграничивать производственные факторы на две основные группы --- опасные производственные факторы
(ОПФ) и вредные производственные факторы (ВПФ). При этом однозначно отнести тот или иной
фактор к подмножеству опасных или вредных не всегда представляется возможным, поскольку
даже нейтральные производственные факторы при наличии определенных условий и обстоятельств
могут становиться вредными или опасными для человека, приводить к травмам и заболеваниям, 
связанным с трудовой деятельностью.

Согласно~\cite{gost_12.0.003-2015} опасные и вредные производственные факторы производственной
среды по природе их воздействия на организм работающего человека подразделяются на:
\begin{itemize}
 \item факторы, воздействие которых носит физическую природу;
 \item факторы, воздействие которых носит химическую природу;
 \item факторы, воздействие которых носит биологическую природу.
\end{itemize}

При работе за ПЭВМ воздействие на организм человека носит физическую природу. Кроме того
работники подвергаются нервно-психическим перегрузкам.

Основываясь на классификации вредных и опасных факторов производства из ГОСТ 12.0.003--2015~\cite{gost_12.0.003-2015} 
можно выделить следующие физические факторы, связанные с работой за ПЭВМ:
\begin{itemize}
 \item повышенный уровень и другие неблагоприятные факторы шума;
 \item повышенная или пониженная температура воздуха рабочей зоны;
 \item повышенная или пониженная влажность воздуха;
 \item повышенная или пониженная подвижность воздуха;
 \item повышенное значение напряжения в электрической цепи;
 \item повышенный уровень статического электричества;
 \item повышенный уровень электромагнитных излучений;
 \item отсутствие или недостаток естественного освещения;
 \item отсутствие или недостаток искусственного освещения;
 \item повышенная яркость, пульсация света. 
\end{itemize}

К нервно-психическим перегрузкам относят~\cite{gost_12.0.003-2015}:
\begin{itemize}
 \item умственное перенапряжение, в том числе вызванное информационной перегрузкой;
 \item перенапряжение анализаторов, в том числе вызванное информационной перегрузкой;
 \item монотонность труда, вызывающая монотонию;
 \item эмоциональные перегрузки.
\end{itemize}

Под информационной перегрузкой понимается воспринимаемая сенсорными системами организма
человека интенсивность поступления информации, воздействующая на центральную нервную систему человека
и способная приводить к различным неблагоприятным последствиям для здоровья.

\subsubsection{Перечень продукции и контролируемые гигиенические параметры вредных и опасных факторов}

Перечень продукции и контролируемых гигиенических параметров огласно~\cite{sanpin_2.4.1340-03} 
приведен в таблице~\ref{tab:life_1}.

\begin{table}[h!]
\caption{ Перечень продукции, контролируемых гигиенических параметров }
\label{tab:life_1}
\begin{center}
\begin{tabularx}{\linewidth}{|X|X|}
\hline
Вид продукции & Контролируемые гигиенические параметры\\
\hline
\item Машина вычислительная электронная цифровая персональная (ПЭВМ) & 
\begin{itemize}
 \item уровни электромагнитных полей (ЭМП);
 \item уровни акустического шума;
 \item концентрация вредных веществ в воздухе;
 \item визуальные показатели видеодисплейного терминала
\end{itemize}									     
\\
\hline
\item Устройства периферийные: модем, клавиатура, устройства хранения информации & 
\begin{itemize}
 \item уровни ЭМП;
 \item уровни акустического шума;
 \item концентрация вредных веществ в воздухе
\end{itemize}
 \\
\hline
\end{tabularx}
\end{center}
\end{table}




\subsubsection{Требования к уровням шума на рабочих местах, оборудованных ПЭВМ}

Допустимые уровни звукового давления и уровней звука, создаваемых ПЭВМ, не должны превышать значений, 
представленных в таблице~\ref{tab:life_2}. Измерение уровня звука и уровней звукового давления 
проводится на расстоянии 50 см от поверхности оборудования и на высоте расположения 
источника(ков) звука. Шумящее оборудование (печатающие устройства, серверы и т.п.), уровни шума которого превышают нормативные, должно 
размещаться вне помещений с ПЭВМ.~\cite{sanpin_2.4.1340-03}

\begin{table}[h!]
\caption{ Допустимые значения уровней звукового давления }
\label{tab:life_2}
\begin{center}
\begin{tabularx}{\linewidth}{|X|X|X|X|X|X|X|X|X|X|}
\hline
\multicolumn{9}{|c|}{Уровни звукового давления в октавных полосах со среднегеометрическими частотами} & \multirow{2}{\hsize}{Уровни звука в дБА}\\
\cline{1-9}
31,5 Гц & 63 Гц & 125 Гц & 250 Гц & 500 Гц & 1000 Гц & 2000 Гц & 4000 Гц & 8000 Гц & \\
\hline
31,5 Гц & 63 Гц & 125 Гц & 250 Гц & 500 Гц & 1000 Гц & 2000 Гц & 4000 Гц & 8000 Гц & 50\\
\hline
\end{tabularx}
\end{center}
\end{table}




\subsubsection{Требования к уровням электромагнитных полей на рабочих местах, оборудованных ПЭВМ}

Временные допустимые уровни ЭМП, создаваемых ПЭВМ на рабочих местах пользователей
представлены в таблице~\ref{tab:life_3}.~\cite{sanpin_2.4.1340-03}

\begin{table}[h!]
\caption{ Временные допустимые уровни ЭМП, создаваемых ПЭВМ на рабочих местах }
\label{tab:life_3}
\begin{center}
\begin{tabularx}{\linewidth}{|>{\hsize=0.45\hsize}X|>{\hsize=0.45\hsize}X|>{\hsize=0.1\hsize}X|}
\hline
\multicolumn{2}{|c|}{Наименование параметров} & ВДУ\\
\hline
\multirow{2}{\hsize}{Напряженность электрического поля} & в диапазоне частот 5 Гц -- 2 кГц & 25 В/м\\
\cline{2-3}
 & в диапазоне частот 2 кГц -- 400 кГц & 2,5 В/м \\
\hline
\multirow{2}{\hsize}{Плотность магнитного потока} & в диапазоне частот 5 Гц - 2 кГц & 250 нТл \\
\cline{2-3}
 & в диапазоне частот 2 кГц -- 400 кГц & 25 нТл \\
\hline
\multicolumn{2}{|c|}{Напряженность электростатического поля} & 15 кВ/м \\
\hline
\end{tabularx}
\end{center}
\end{table}


\subsubsection{Требования к визуальным параметрам устройств отображения информации}

Предельно допустимые значения визуальных параметров визуального дисплейного терминала, 
контролируемые на рабочих местах, представлены в таблице~\ref{tab:life_4}.~\cite{sanpin_2.4.1340-03}


\begin{table}[h!]
\caption{ Визуальные параметры устройств отображения информации }
\label{tab:life_4}
\begin{center}
\begin{tabularx}{\linewidth}{|>{\hsize=0.7\hsize}X|>{\hsize=0.3\hsize}X|}
\hline
Параметры & Допустимые значения\\
\hline
Яркость белого поля & Не менее 35 кд/кв.м\\
\hline
Неравномерность яркости рабочего поля & Не более ± 20\%\\
\hline
Контрасность (для монохромного режима) & Не менее 3:1\\
\hline
Временная нестабильность изображений (непреднамеренное изменение во времени яркости 
изображения на экране дисплея) & Не должна фиксироваться\\
\hline
Пространственная нестабильность изображения (непреднамеренные изменения 
положения фрагментов изображения на экране)& Не более $2\times1E(-4L)$, где $L$~--- проектное расстояние наблюдения, мм\\
\hline
\end{tabularx}
\end{center}
\end{table}

Для дисплеев на плоских дискретных экранах (жидкокристаллических, плазменных и т.п.) и не менее 60 Гц 
для дисплеев частота обновления изображения должна быть не менее 75 Гц при всех режимах 
разрешения экрана, гарантируемых нормативной документацией на конкретный тип дисплея .~\cite{sanpin_2.4.1340-03}


\subsubsection{Требования к микроклимату. Концентрации вредных веществ, выделяемых ПЭВМ в воздух помещения}


 Показатели микроклимата должны обеспечивать сохранение теплового баланса человека с окружающей средой и 
 поддержание оптимального или допустимого теплового состояния организма.
 Несоблюдение оптимальных микроклиматических условий может привести к ухудшению состояния здоровья,
 снижению работоспособности, ощущению дискомфорта, напряжению механизмов терморегуляции.
 
 Работа за ПЭВМ относится к категории Iа: с интенсивностью энерготрат до 120 ккал/ч (до 139 Вт), 
 производимая сидя и сопровождающаяся незначительным физическим напряжением.~\cite{sanpin_mikroclimate} 
 
 Согласно~\cite{sanpin_mikroclimate} допустимые величины показателей микроклимата на рабочих местах производственных помещений категории Iа 
 должны соответствовать значениям, приведенным в таблице~\ref{tab:climate_1}. 
 
 Амплитуда колебания температуры воздуха в течение смены не должна превышать~4\textdegree{} C.
 
\begin{table}[h!]
\caption{ Оптимальные величины показателей микроклимата на рабочих местах производственных помещений категории Iа }
\label{tab:climate_1}
\begin{center}
\begin{tabularx}{\linewidth}{|X|X|X|X|X|}
\hline
Период года & Температура воздуха, \textdegree{}C & Температура поверхностей, \textdegree{}C & Относительная влажность воздуха & Скорость движения воздуха, м/с\\
\hline
Холодный & 22-24 & 21-25 & 60-40 & 0,1\\
\hline
Теплый & 23-25 & 22-26 & 60-40 & 0,1\\
\hline
\end{tabularx}
\end{center}
\end{table}


\subsubsection{Требования к освещению на рабочих местах, оборудованных ПЭВМ}

Согласно~\cite{sanpin_2.4.1340-03}, к освещению рабочих мест, оборудованных ПЭВМ, представляются
следующие требования:
\begin{enumerate}
 \item Рабочие столы следует размещать таким образом, чтобы видеодисплейные 
 терминалы были ориентированы боковой стороной к световым проемам, чтобы естественный свет 
 падал преимущественно слева.
 \item Искусственное освещение в помещениях для эксплуатации ПЭВМ должно осуществляться 
 системой общего равномерного освещения. В производственных и административно-общественных помещениях, 
 в случаях преимущественной работы с документами, следует применять системы комбинированного освещения 
 (к общему освещению дополнительно устанавливаются светильники местного освещения, предназначенные для 
 освещения зоны расположения документов).
 \item Освещенность на поверхности стола в зоне размещения рабочего документа должна быть 300-500 лк. 
 Освещение не должно создавать бликов на поверхности экрана. Освещенность поверхности экрана не должна 
 быть более 300 лк.
 \item Следует ограничивать прямую блесткость от источников освещения, при этом яркость светящихся 
 поверхностей (окна, светильники и др.), находящихся в поле зрения, должна быть не более 200 кд/м$^{2}$.
 \item Следует ограничивать отраженную блесткость на рабочих поверхностях (экран, стол, клавиатура и др.) 
 за счет правильного выбора типов светильников и расположения рабочих мест по отношению к источникам 
 естественного и искусственного освещения, при этом яркость бликов на экране ПЭВМ не должна превышать
 40 кд/м$^{2}$ и яркость потолка не должна превышать 200 кд/м$^{2}$.
 \item Показатель ослепленности для источников общего искусственного освещения в 
 производственных помещениях должен быть не более 20. Показатель дискомфорта в административно-общественных 
 помещениях не более 40, в дошкольных и учебных помещениях не более 15.
 \item Яркость светильников общего освещения в зоне углов излучения от 50 до 90° с вертикалью в 
 продольной и поперечной плоскостях должна составлять не более 200 кд/м$^{2}$, 
 защитный угол светильников должен быть не менее 40\textdegree{}.
 \item Светильники местного освещения должны иметь непросвечивающий отражатель с защитным углом не менее 40\textdegree{}.
 \item Следует ограничивать неравномерность распределения яркости в поле зрения пользователя ПЭВМ, 
 при этом соотношение яркости между рабочими поверхностями не должно превышать 3:1-5:1, 
 а между рабочими поверхностями и поверхностями стен и оборудования 10:1.
 \item Общее освещение при использовании люминесцентных светильников следует 
 выполнять в виде сплошных или прерывистых линий светильников, расположенных сбоку от рабочих мест, 
 параллельно линии зрения пользователя при рядном расположении видеодисплейных терминалов. 
 При периметральном расположении компьютеров линии светильников должны располагаться локализованно 
 над рабочим столом ближе к его переднему краю, обращенному к оператору.
 \item Коэффициент запаса (Кз) для осветительных установок общего освещения должен приниматься равным 1,4.
 \item Коэффициент пульсации не должен превышать 5\%.
 \item Для обеспечения нормируемых значений освещенности в помещениях для использования 
 ПЭВМ следует проводить чистку стекол оконных рам и светильников не реже двух раз в год и 
 проводить своевременную замену перегоревших ламп.
\end{enumerate}


Рассчитаем показатели освещенности E на рабочем месте с помощью формул из стандарта~\cite{gost_8995}:
\begin{center}
$E = \dfrac{I}{r^{2}}cos(i),$
\end{center}
где~~~~~\ $\textit{I}$ --- сила света, кд;

$\textit{i}$ -- угол падения лучей света относительно нормали к поверхности;

$\textit{r}$ -- расстояние до источника света, м.

Формула для вычисления силы света:
\begin{center}
$I = \dfrac{F}{4\pi},$
\end{center}
где~~~~~\ $\textit{F}$ --- номинальный световой поток, лм.

Помещение оборудовано 1-ой люменисцентной лампой и 3-мя лампами накаливания.
Световой поток люменисцентной лампы мощностью 10 Вт составляет около 400 Лм, одной лампы
накаливания мощностью 60 Вт --- 710 Лм. 

Сила света люменисцентной лампы:
\begin{center}
$I_{1} = \dfrac{400}{4\pi}= 31,83$ кд.
\end{center}

Сила света ламп накаливания:
\begin{center}
$I_{2} = \dfrac{3\times710}{4\pi}= 169,5$ кд.
\end{center}

Показатель освещенности на поверхности стола составляет:
\begin{center}
$E_{стола} = \dfrac{31,83}{(0,3)^{2}}cos(30\textdegree{}) + \dfrac{169,5}{(1,5)^{2}}cos(45\textdegree{}) = 359,56$ лк.
\end{center}

Показатель освещенности на поверхности экрана составляет:
\begin{center}
$E_{экрана} = \dfrac{31,83}{(0,3)^{2}}cos(60\textdegree{}) + \dfrac{169,5}{(1,6)^{2}}cos(60\textdegree{}) = 209,94$ лк.
\end{center}

По результатам расчетов можно сделать вывод, что показатели освещенности соответствуют нормам, принятым 
в~\cite{sanpin_2.4.1340-03}. 


\subsubsection{Требования к организации рабочих мест пользователей ПЭВМ}

Организация и оборудование рабочих мест с ПЭВМ для взрослых пользователей согласно~\cite{sanpin_2.4.1340-03}
включает в себя следующие требования:
\begin{enumerate}
 \item Высота рабочей поверхности стола для взрослых пользователей должна регулироваться в пределах 
 680-800 мм; при отсутствии такой возможности высота рабочей поверхности стола должна составлять 725 мм.
 \item Модульными размерами рабочей поверхности стола для ПЭВМ, на основании которых должны 
 рассчитываться конструктивные размеры, следует считать: ширину 800, 1000, 1200 и 1400 мм, глубину 800 
 и 1000 мм при нерегулируемой его высоте, равной 725 мм.
 \item Рабочий стол должен иметь пространство для ног высотой не менее 600 мм, шириной --- не менее 500 мм, 
 глубиной на уровне колен --- не менее 450 мм и на уровне вытянутых ног - не менее 650 мм.
 \item Конструкция рабочего стула должна обеспечивать:
 \begin{itemize}
  \item ширину и глубину поверхности сиденья не менее 400 мм;
  \item поверхность сиденья с закругленным передним краем;
  \item регулировку высоты поверхности сиденья в пределах 400-550 мм и углам наклона вперед до 15\textdegree{} и назад до 5\textdegree{};
  \item высоту опорной поверхности спинки 300±20 мм, ширину --- не менее 380 мм и радиус кривизны горизонтальной плоскости --- 400 мм;
  \item угол наклона спинки в вертикальной плоскости в пределах ±30\textdegree{};
  \item регулировку расстояния спинки от переднего края сиденья в пределах 260-400 мм;
  \item стационарные или съемные подлокотники длиной не менее 250 мм и шириной --- 50-70 мм;
  \item регулировку подлокотников по высоте над сиденьем в пределах 230±30 мм и внутреннего расстояния между 
  подлокотниками в пределах 350-500 мм.
 \end{itemize}
 \item Рабочее место пользователя ПЭВМ следует оборудовать подставкой для ног, имеющей ширину не менее 300 мм, глубину не менее 400 мм, 
 регулировку по высоте в пределах до 150 мм и по углу наклона опорной поверхности подставки 
 до 20\textdegree. Поверхность подставки должна быть рифленой и иметь по переднему краю бортик высотой 10 мм.
 \item Клавиатуру следует располагать на поверхности стола на расстоянии 100-300 мм от края, обращенного к пользователю, или на специальной, 
 регулируемой по высоте рабочей поверхности, отделенной от основной столешницы.
\end{enumerate}

Результаты сравнения показали, что значения параметров рабочего места 
соответствуют значениям рекомендуемых параметров.

\subsection{Инструкция по работе на персональном компьютере}

Инструкция по охране труда для пользователей ПЭВМ утверждена РД 153-34.0-03.2.98-2001~\cite{rd_computer}.

\subsubsection{Противопожарная безопасность}

Для соблюдения противопожарной безопасности запрещается:
\begin{itemize}
 \item хранить и применять горючие жидкости, взрывчатые вещества, баллоны с газами и др.;
 \item использовать электронагревательные приборы;
 \item эксплуатировать провода электроприборов с поврежденной изоляцией;
 \item применять открытый огонь;
 \item курить в помещении;
 \item оставлять без наблюдения включенную в сеть ПЭВМ, оргтехнику, бытовую технику.
\end{itemize}

По окончании работы необходимо осмотреть помещения на наличие признаков возгорания. 
При обнаружении возгорания работник обязан:
\begin{itemize}
 \item немедленно сообщить об этом по телефону «01» в пожарную охрану 
 (при этом необходимо назвать адрес, место возникновения пожара, а также сообщить свою фамилию и должность);
 \item сообщить руководителю или его заместителю о пожаре;
 \item принять меры по организации эвакуации людей (эвакуацию начинать из помещения, 
 где возник пожар, а также из помещений, которым угрожает опасность распространения огня и дыма);
 \item одновременно с эвакуацией людей, приступить к тушению пожара своими силами и 
 имеющимися средствами пожаротушения.
\end{itemize}

\subsubsection{Электрическая безопасность}

Для соблюдения электрической безопасности запрещается: 
\begin{itemize}
 \item прикасаться к проводам и розеткам, открывать электрощитки;
 \item разбирать и проводить самостоятельно ремонт оборудования, розеток и т.д.;
 \item пользоваться поврежденными розетками, рубильниками, вилками и прочим электрооборудованием;
 \item пользоваться неисправной или незаземленной аппаратурой;
 \item нарушать правила эксплуатации ПЭВМ и оргтехники, а так же инструкции по работе на ПЭВМ и 
средствах оргтехники;
 \item включать в сетевые фильтры, блоки бесперебойного питания и 
специализированные розетки, расположенные в коробах бытовую технику и другое, 
не относящееся к ПЭВМ оборудование.
\end{itemize}

По окончании работы необходимо обесточить все электроприборы. При наличии в 
помещении выделенной сети электропитания для ПЭВМ, необходимо выключить автомат питания 
в распределительном щите.

\subsubsection{Требования охраны труда во время работы. Оказание первой медицинской помощи}

При работе с персональным компьютером необходимо:
\begin{itemize}
 \item соблюдать оптимальное расстояние от экрана видеомонитора до глаз, поддерживать 
рациональную рабочую позу и оптимальное размещение на рабочей поверхности используемого 
оборудования с учетом его количества и конструктивных особенностей, характера выполняемой работы;
 \item осуществлять систематическое проветривание помещения после каждого часа работы с ПЭВМ;
 \item работу за экраном видеомонитора следует периодически прерывать на регламентированные 
перерывы, которые устанавливаются для обеспечения работоспособности и сохранения здоровья, 
или заменять другой работой с целью сокращения рабочей нагрузки у экрана;
 \item продолжительность непрерывной работы с ПЭВМ без регламентированного перерыва не должна превышать двух часов.
\end{itemize}

К непосредственной работе на ПЭВМ допускаются лица, не имеющие медицин­ских противопоказаний. 
Женщины со времени установления беременности и в период кормления ре­бенка грудью к выполнению 
всех видов работ, связанных с использованием ПЭВМ, не допускаются.

Последовательность действий при оказании первой помощи пострадавшему:
\begin{itemize}
 \item устранение воздействия на организм пострадавшего опасных и вредных факторов 
(освобождение его от действия электрического тока, гашение горящей одежды и т.п.);
 \item оценка состояния пострадавшего;
 \item определение характера травмы, создающей наибольшую угрозу для жизни пострадавшего, 
и последовательности действий по его спасению;
 \item выполнение необходимых мероприятий по спасению пострадавшего в порядке срочности 
(восстановление проходимости дыхательных путей, проведение искусственного дыхания, 
наружного массажа сердца, остановка кровотечения, иммобилизация места перелома, наложение повязки и т.д.);
 \item поддержание основных жизненных функций пострадавшего до прибытия медицинского персонала;
 \item вызов скорой медицинской помощи.
\end{itemize}

При поражении электрическим током необходимо как можно быстрее освободить пострадавшего от 
действия тока, так как от продолжительности его действия на организм зависит тяжесть электротравмы. 
Отключить электроустановку можно с помощью выключателя, рубильника или другого отключающего аппарата.
Если отсутствует возможность быстрого отключения электроустановки, то необходимо принять меры к 
отделению пострадавшего от токоведущих частей, к которым он прикасается. 
При этом во всех случаях оказывающий помощь не должен прикасаться к пострадавшему без применения 
надлежащих мер предосторожности, так как это опасно для жизни. Он должен также следить за тем, 
чтобы самому не оказаться в контакте с токоведущей частью или под напряжением шага, 
находясь в зоне растекания тока замыкания на землю.

При оказании помощи пострадавшему при ожогах во избежание заражения нельзя касаться руками 
обожженных участков кожи или смазывать их мазями, жирами, маслами и т. п.
Нельзя вскрывать пузыри, так как, удаляя их, можно легко содрать обожженную кожу и тем самым 
создать благоприятные условия для заражения раны.
При небольших ожогах степени нужно наложить на обожженный участок кожи стерильную повязку.

Одежду и обувь с обожженного места нельзя срывать, а следует разрезать ножницами и осторожно снять. 
Если обгоревшие куски одежды прилипли к обожженному участку кожи, то поверх них необходимо наложить 
стерильную повязку и направить пострадавшего в лечебное учреждение.

При тяжелых и обширных ожогах необходимо пострадавшего завернуть в чистую простыню или ткань, 
не раздевая его, укрыть, напоить теплым чаем и создать покой до прибытия врача.