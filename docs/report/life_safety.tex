\subsection{Анализ опасных и вредных производственных факторов на рабочем месте}





\subsubsection{Перечень продукции и контролируемые гигиенические параметры вредных и опасных факторов}


~\cite{sanpin_2.4.1340-03}

Перечень продукции и контролируемых гигиенических параметров приведен в таблице~\ref{tab:life_1}.

\begin{table}[h!]
\caption{ Перечень продукции, контролируемых гигиенических параметров }
\label{tab:life_1}
\begin{center}
\begin{tabularx}{\linewidth}{|X|X|}
\hline
Вид продукции & Контролируемые гигиенические параметры\\
\hline
\item Машина вычислительная электронная цифровая персональная (ПЭВМ) & 
\begin{itemize}
 \item уровни электромагнитных полей (ЭМП);
 \item уровни акустического шума;
 \item концентрация вредных веществ в воздухе;
 \item визуальные показатели видеодисплейного терминала
\end{itemize}									     
\\
\hline
\item Устройства периферийные: модем, клавиатура, принтер, устройства хранения информации & 
\begin{itemize}
 \item уровни ЭМП;
 \item уровни акустического шума;
 \item концентрация вредных веществ в воздухе
\end{itemize}
 \\
\hline
\end{tabularx}
\end{center}
\end{table}




\subsubsection{Требования к уровням шума на рабочих местах, оборудованных ПЭВМ}


~\cite{sanpin_2.4.1340-03}
~\ref{tab:life_2}


\begin{table}[h!]
\caption{ Допустимые значения уровней звукового давления }
\label{tab:life_2}
\begin{center}
\begin{tabularx}{\linewidth}{|X|X|X|X|X|X|X|X|X|X|}
\hline
\multicolumn{9}{|c|}{Уровни звукового давления в октавных полосах со среднегеометрическими частотами} & \multirow{2}{\hsize}{Уровни звука в дБА}\\
\cline{1-9}
31,5 Гц & 63 Гц & 125 Гц & 250 Гц & 500 Гц & 1000 Гц & 2000 Гц & 4000 Гц & 8000 Гц & \\
\hline
31,5 Гц & 63 Гц & 125 Гц & 250 Гц & 500 Гц & 1000 Гц & 2000 Гц & 4000 Гц & 8000 Гц & 50\\
\hline
\end{tabularx}
\end{center}
\end{table}




\subsubsection{Требования к уровням электромагнитных полей на рабочих местах, оборудованных ПЭВМ}

~\cite{sanpin_2.4.1340-03}
~\ref{tab:life_3}


\begin{table}[h!]
\caption{ Временные допустимые уровни ЭМП, создаваемых ПЭВМ на рабочих местах }
\label{tab:life_3}
\begin{center}
\begin{tabularx}{\linewidth}{|>{\hsize=0.45\hsize}X|>{\hsize=0.45\hsize}X|>{\hsize=0.1\hsize}X|}
\hline
\multicolumn{2}{|c|}{Наименование параметров} & ВДУ\\
\hline
\multirow{2}{\hsize}{Напряженность электрического поля} & в диапазоне частот 5 Гц -- 2 кГц & 25 В/м\\
\cline{2-3}
 & в диапазоне частот 2 кГц -- 400 кГц & 2,5 В/м \\
\hline
\multirow{2}{\hsize}{Плотность магнитного потока} & в диапазоне частот 5 Гц - 2 кГц & 250 нТл \\
\cline{2-3}
 & в диапазоне частот 2 кГц -- 400 кГц & 25 нТл \\
\hline
\multicolumn{2}{|c|}{Напряженность электростатического поля} & 15 кВ/м \\
\hline
\end{tabularx}
\end{center}
\end{table}


\subsubsection{Требования к визуальным параметрам устройств отображения информации}


~\cite{sanpin_2.4.1340-03}
~\ref{tab:life_4}


\begin{table}[h!]
\caption{ Визуальные параметры устройств отображения информации }
\label{tab:life_4}
\begin{center}
\begin{tabularx}{\linewidth}{|>{\hsize=0.7\hsize}X|>{\hsize=0.3\hsize}X|}
\hline
Параметры & Допустимые значения\\
\hline
Яркость белого поля & Не менее 35 кд/кв.м\\
\hline
Неравномерность яркости рабочего поля & Не более ± 20\%\\
\hline
Контрасность (для монохромного режима) & Не менее 3:1\\
\hline
Временная нестабильность изображений (непреднамеренное изменение во времени яркости 
изображения на экране дисплея) & Не должна фиксироваться\\
\hline
Пространственная нестабильность изображения (непреднамеренные изменения 
положения фрагментов изображения на экране)& Не более $2\times1E(-4L)$, где $L$~--- проектное расстояние наблюдения, мм\\
\hline
\end{tabularx}
\end{center}
\end{table}

Для дисплеев на ЭЛТ частота обновления изображения должна быть не менее 75 Гц при всех режимах 
разрешения экрана, гарантируемых нормативной документацией на конкретный тип дисплея и не менее 60 Гц 
для дисплеев на плоских дискретных экранах (жидкокристаллических, плазменных и т.п.).~\cite{sanpin_2.4.1340-03}


\subsubsection{Требования к микроклимату. Концентрации вредных веществ, выделяемых ПЭВМ в воздух помещения}


 Показатели микроклимата должны обеспечивать сохранение теплового баланса человека с окружающей средой и 
 поддержание оптимального или допустимого теплового состояния организма.
 Несоблюдение оптимальных микроклиматических условий может привести к ухудшению состояния здоровья,
 снижению работоспособности, ощущению дискомфорта, напряжению механизмов терморегуляции.
 
 Работа за ПЭВМ относится к категории Iа: с интенсивностью энерготрат до 120 ккал/ч (до 139 Вт), 
 производимая сидя и сопровождающиеся незначительным физическим напряжением.~\cite{sanpin_mikroclimate} 
 
 Согласно~\cite{sanpin_mikroclimate} допустимые величины показателей микроклимата на рабочих местах должны 
 соответствовать значениям, приведенным в таблице~\ref{tab:climate_1}. 
 
 Амплитуда колебания температуры воздуха в течение смены не должна превышать~\textdegree{4}C.
 
\begin{table}[h!]
\caption{ Оптимальные величины показателей микроклимата на рабочих местах производственных помещений категории Iа }
\label{tab:climate_1}
\begin{center}
\begin{tabularx}{\linewidth}{|X|X|X|X|X|}
\hline
Период года & Температура воздуха, \textdegree{}C & Температура поверхностей, \textdegree{}C & Относительная влажность воздуха & Скорость движения воздуха, м/с\\
\hline
Холодный & 22-24 & 21-25 & 60-40 & 0,1\\
\hline
Теплый & 23-25 & 22-26 & 60-40 & 0,1\\
\hline
\end{tabularx}
\end{center}
\end{table}





\subsubsection{Требования к освещенности}

\subsubsection{Требования к организации рабочих мест пользователей ПЭВМ}


\subsection{Инструкция по работе на персональном компьютере}

\subsubsection{Противопожарная безопасность}

\subsubsection{Электрическая безопасность}

\subsubsection{Требования охраны труда во время работы. Оказание первой медицинской помощи}