% \documentclass[russian,14pt,floatsection,nocolumnsxix, nocolumnxxxii,nocolumnxxxi]{eskdtext}
\usepackage[utf8x]{inputenc}

% закомментировать для рамок 
\usepackage[numbertop, numbercenter]{eskdplain}

% для объединения строк в таблице
\usepackage{multirow}



% для размера колонок
\usepackage{tabularx}
\usepackage{lscape}
\newcolumntype{n}{>{\hsize=.4\hsize}X}
\newcolumntype{m}{>{\hsize=.2\hsize}X}
% tables on top
\makeatletter
\setlength{\@fptop}{0pt}
\makeatother

\usepackage{setspace}
\onehalfspacing % полуторный интервал для всего текста

% - Подключаем шрифты из пакета scalable-cyrfonts-tex
\usepackage{cyrtimes}

% - Отступ красной строки
\setlength{\parindent}{1.25cm}

% - Убирает точку в списке литературы
\makeatletter
\def\@biblabel#1{#1 }

% ограничение для оглавления
%\usepackage{tocvsec2}
\setcounter{tocdepth}{2}

% - Точки для всех пунктов в оглавлении
\renewcommand*{\l@section}{\@dottedtocline{1}{1.5em}{2.3em}}
\renewcommand*{\l@subsection}{\@dottedtocline{1}{1.5em}{2.3em}}
% \renewcommand*{\l@subsubsection}{\@dottedtocline{1}{1.5em}{2.3em}}

% - Для переопределения списков
\renewcommand{\theenumi}{\arabic{enumi}}
\renewcommand{\labelenumi}{\theenumi)}
\makeatother

\usepackage{enumitem}
\setlist{nolistsep, itemsep=0.3cm,parsep=0pt}

% - ГОСТ списка литературы
\bibliographystyle{utf8gost705u}


% - Верикальные отступы заголовков 
\ESKDsectSkip{section}{1em}{1em}
\ESKDsectSkip{subsection}{1em}{1em}
\ESKDsectSkip{subsubsection}{1em}{1em}

% - Изменение заголовков
\usepackage{titlesec}
\titleformat{\section}{\normalfont\normalsize\centering}{\thesection}{1.0em}{}
\titleformat{\subsection}{\normalfont\normalsize\centering}{\thesubsection}{1.0em}{}
\titleformat{\subsubsection}{\normalfont\normalsize\centering}{\thesubsubsection}{1.0em}{}
\titleformat{\paragraph}{\normalfont\normalsize\centering}{\theparagraph}{1.0em}{}

% - Оставим место под ТЗ 
\setcounter{page}{1}

% - Для больших таблиц
\usepackage{longtable}
\usepackage{tabularx}
\renewcommand{\thetable}{\thesection.\arabic{table}}

% for centering tabularx
\newcolumntype{Y}{>{\centering\arraybackslash}X}

% - Используем графику в документе
\usepackage{graphicx}
\graphicspath{{images/}}
\renewcommand{\thefigure}{\thesection.\arabic{figure}}
% to have figure on top
%
\makeatletter
\setlength{\@fptop}{0pt}
\makeatother

% - Счётчики
\usepackage{eskdtotal}

% - Выравнивание по ширине
\sloppy

% - Разрешить перенос двух последних букв слова
\righthyphenmin=2

% - Оформление списков
\RequirePackage{enumitem}
\renewcommand{\alph}[1]{\asbuk{#1}}
\setlist{nolistsep}
\setitemize[1]{label=--, fullwidth, itemindent=\parindent, 
  listparindent=\parindent}% для дефисного списка
\setitemize[2]{label=--, fullwidth, itemindent=\parindent, 
  listparindent=\parindent, leftmargin=\parindent}
\setenumerate[1]{label=\arabic*), fullwidth, itemindent=\parindent, 
  listparindent=\parindent}% для нумерованного списка
\setenumerate[2]{label=\alph*), fullwidth, itemindent=\parindent, 
  listparindent=\parindent, leftmargin=\parindent}% для списка 2-ой ступени, который будет нумероваться а), б) и т.д.
  
% - Оформляем листинг кода (не использовать комментарии на русском!)
\usepackage{listings}  
\lstset{basicstyle=\ttfamily\scriptsize}
\lstset{extendedchars=\true}

% - выводим текст как есть с размером шрифта scriptsize
\makeatletter
\def\verbatim{\scriptsize\@verbatim \frenchspacing\@vobeyspaces \@xverbatim}
\makeatother

% - Вставка pdf
\usepackage[enable-survey]{pdfpages}

%межстрочный интервал
\usepackage{setspace}
\linespread{1.5}

%фамилии для рамок
\author{\ESKDfontII Мейта М.В.}
\ESKDchecker{\ESKDfontII Давыдова Е.М.}
\ESKDnormContr{\ESKDfontII Якимук А.Ю.}
\ESKDapprovedBy{\ESKDfontII Шелупанов А.А.}
\ESKDcolumnI{\ESKDfontIII Вопросы охраны труда и безопасности жизнедеятельности}
\ESKDcolumnIX{\ESKDfontIII ТУСУР, ФБ, каф.~КИБЭВС, гр.~722}
\ESKDsignature{КИБЭВС.58.29.29 ПЗ}

% 
% \begin{document}
%  
% \setcounter{page}{72}
% \setcounter{section}{12}
% 
% \ESKDstyle{formIIab}
% \ESKDthisStyle{formII}
% 
% 
% \section{Вопросы охраны труда и безопасности жизнедеятельности}
% \subsection{Анализ опасных и вредных производственных факторов на рабочем месте}
% 
% В ходе трудового процесса организм человека может подвергаться различным воздействиям, 
% оказывающим влияние на его здоровье и работоспособность. Подобное воздействие может приводить к 
% различным результирующим последствиям, которые зависят от характера воздействия (прямого или
% опосредованного), наличия тех или иных факторов производственной среды, а также условий
% их проявления. 
% 
% Принято разграничивать производственные факторы на две основные группы~--- опасные производственные факторы
% (ОПФ) и вредные производственные факторы (ВПФ). При этом однозначно отнести тот или иной
% фактор к подмножеству опасных или вредных не всегда представляется возможным, поскольку
% даже нейтральные производственные факторы при наличии определенных условий и обстоятельств
% могут становиться вредными или опасными для человека, приводить к травмам и заболеваниям, 
% связанным с трудовой деятельностью.
% 
% При работе за ПЭВМ воздействие на организм человека носит физическую природу. Кроме того
% работники подвергаются нервно-психическим перегрузкам.
% 
% Можно выделить следующие физические факторы, связанные с работой за ПЭВМ:
% \begin{itemize}
%  \item факторы шума;
%  \item температура, влажность и подвижность воздуха рабочей зоны;
%  \item значение напряжения в электрической цепи;
%  \item уровень статического электричества и электромагнитных излучений;
%  \item отсутствие или недостаток естественного и искусственного освещения;
%  \item повышенная яркость, пульсация света. 
% \end{itemize}

\setcounter{section}{12}

К нервно-психическим перегрузкам относят:
\begin{itemize}
 \item умственное перенапряжение, в том числе вызванное информационной перегрузкой;
 \item монотонность труда;
 \item эмоциональные перегрузки~\cite{gost_12.0.003-2015}.
\end{itemize}

Работа за персональной электронно-вычислительной машиной 
(ПЭВМ) относится к категории Iа: с интенсивностью энерготрат до 120 ккал/ч (до
139 Вт), производимая сидя и сопровождающаяся незначительным физическим напряжением~\cite{sanpin_mikroclimate}.
Требования к такого рода рабочим местам устанавливает СанПин 2.2.2/2.4.1340-031 
<<Гигиенические требования к персональным электронно-вычислительным машинам
и организации работы>>~\cite{sanpin_2.4.1340-03}. 

Перечень продукции и контролируемых гигиенических параметров огласно~\cite{sanpin_2.4.1340-03} 
приведен в таблице~\ref{tab:life_1}.

\begin{table}[h!]
\caption{ Перечень продукции, контролируемых гигиенических параметров }
\label{tab:life_1}
\begin{center}
\begin{tabularx}{\linewidth}{|X|X|}
\hline
Вид продукции & Контролируемые гигиенические параметры\\
\hline
\item Машина вычислительная электронная цифровая персональная (ПЭВМ) & 
\begin{itemize}
 \item уровни электромагнитных полей (ЭМП);
 \item уровни акустического шума;
 \item концентрация вредных веществ в воздухе;
 \item визуальные показатели видеодисплейного терминала
\end{itemize}									     
\\
\hline
\item Устройства периферийные: модем, клавиатура, устройства хранения информации & 
\begin{itemize}
 \item уровни ЭМП;
 \item уровни акустического шума;
 \item концентрация вредных веществ в воздухе
\end{itemize}
 \\
\hline
\end{tabularx}
\end{center}
\end{table}


\subsubsection{Требования к уровням электромагнитных полей на рабочих местах, оборудованных ПЭВМ}

Временные допустимые уровни ЭМП, создаваемых ПЭВМ на рабочих местах пользователей
представлены в таблице~\ref{tab:life_3}~\cite{sanpin_2.4.1340-03}.

\begin{table}[h!]
\caption{ Временные допустимые уровни ЭМП, создаваемых ПЭВМ на рабочих местах }
\label{tab:life_3}
\begin{center}
\begin{tabularx}{\linewidth}{|>{\hsize=0.45\hsize}X|>{\hsize=0.45\hsize}X|>{\hsize=0.1\hsize}X|}
\hline
\multicolumn{2}{|c|}{Наименование параметров} & ВДУ\\
\hline
\multirow{2}{\hsize}{Напряженность электрического поля} & в диапазоне частот 5 Гц -- 2 кГц & 25 В/м\\
\cline{2-3}
 & в диапазоне частот 2 кГц -- 400 кГц & 2,5 В/м \\
\hline
\multirow{2}{\hsize}{Плотность магнитного потока} & в диапазоне частот 5 Гц - 2 кГц & 250 нТл \\
\cline{2-3}
 & в диапазоне частот 2 кГц -- 400 кГц & 25 нТл \\
\hline
\multicolumn{2}{|c|}{Напряженность электростатического поля} & 15 кВ/м \\
\hline
\end{tabularx}
\end{center}
\end{table}


\subsubsection{Требования к визуальным параметрам устройств отображения информации}

Предельно допустимые значения визуальных параметров визуального дисплейного терминала, 
контролируемые на рабочих местах, представлены в таблице~\ref{tab:life_4}~\cite{sanpin_2.4.1340-03}.


\begin{table}[h!]
\caption{ Визуальные параметры устройств отображения информации }
\label{tab:life_4}
\begin{center}
\begin{tabularx}{\linewidth}{|>{\hsize=0.7\hsize}X|>{\hsize=0.3\hsize}X|}
\hline
Параметры & Допустимые значения\\
\hline
Яркость белого поля & Не менее 35 кд/кв.м\\
\hline
Неравномерность яркости рабочего поля & Не более ± 20\%\\
\hline
Контрасность (для монохромного режима) & Не менее 3:1\\
\hline
Временная нестабильность изображений (непреднамеренное изменение во времени яркости 
изображения на экране дисплея) & Не должна фиксироваться\\
\hline
Пространственная нестабильность изображения (непреднамеренные изменения 
положения фрагментов изображения на экране)& Не более $2\times1E(-4L)$, где $L$~--- проектное расстояние наблюдения, мм\\
\hline
\end{tabularx}
\end{center}
\end{table}

Для дисплеев на плоских дискретных экранах (жидкокристаллических, плазменных и т.п.) и не менее 60 Гц 
для дисплеев частота обновления изображения должна быть не менее 75 Гц при всех режимах 
разрешения экрана, гарантируемых нормативной документацией на конкретный тип дисплея~\cite{sanpin_2.4.1340-03}.


\subsubsection{Требования к микроклимату. Концентрации вредных веществ, выделяемых ПЭВМ в воздух помещения}


 Показатели микроклимата должны обеспечивать сохранение теплового баланса человека с окружающей средой и 
 поддержание оптимального или допустимого теплового состояния организма.
 Несоблюдение оптимальных микроклиматических условий может привести к ухудшению состояния здоровья,
 снижению работоспособности, ощущению дискомфорта, напряжению механизмов терморегуляции.
 
 Работа за ПЭВМ относится к категории Iа: с интенсивностью энерготрат до 120 ккал/ч (до 139 Вт), 
 производимая сидя и сопровождающаяся незначительным физическим напряжением~\cite{sanpin_mikroclimate}.
 
 Согласно~\cite{sanpin_mikroclimate} допустимые величины показателей микроклимата на рабочих местах производственных помещений категории Iа 
 должны соответствовать значениям, приведенным в таблице~\ref{tab:climate_1}. 
 
 Амплитуда колебания температуры воздуха в течение смены не должна превышать~4\textdegree{} C.
 
\begin{table}[h!]
\caption{ Оптимальные величины показателей микроклимата на рабочих местах производственных помещений категории Iа }
\label{tab:climate_1}
\begin{center}
\begin{tabularx}{\linewidth}{|X|X|X|X|X|}
\hline
Период года & Температура воздуха, \textdegree{}C & Температура поверхностей, \textdegree{}C & Относительная влажность воздуха & Скорость движения воздуха, м/с\\
\hline
Холодный & 22-24 & 21-25 & 60-40 & 0,1\\
\hline
Теплый & 23-25 & 22-26 & 60-40 & 0,1\\
\hline
\end{tabularx}
\end{center}
\end{table}


\subsubsection{Требования к освещению на рабочих местах, оборудованных ПЭВМ}

Согласно~\cite{sanpin_2.4.1340-03}, к освещению рабочих мест, оборудованных ПЭВМ, представляются
следующие требования:
\begin{enumerate}
 \item Рабочие столы следует размещать таким образом, чтобы видеодисплейные 
 терминалы были ориентированы боковой стороной к световым проемам, чтобы естественный свет 
 падал преимущественно слева.
 \item Искусственное освещение в помещениях для эксплуатации ПЭВМ должно осуществляться 
 системой общего равномерного освещения. В производственных и административно-общественных помещениях, 
 в случаях преимущественной работы с документами, следует применять системы комбинированного освещения 
 (к общему освещению дополнительно устанавливаются светильники местного освещения, предназначенные для 
 освещения зоны расположения документов).
 \item Освещенность на поверхности стола в зоне размещения рабочего документа должна быть 300-500 лк. 
 Освещение не должно создавать бликов на поверхности экрана. Освещенность поверхности экрана не должна 
 быть более 300 лк.
 \item Следует ограничивать прямую блесткость от источников освещения, при этом яркость светящихся 
 поверхностей (окна, светильники и др.), находящихся в поле зрения, должна быть не более 200 кд/м$^{2}$.
 \item Следует ограничивать отраженную блесткость на рабочих поверхностях (экран, стол, клавиатура и др.) 
 за счет правильного выбора типов светильников и расположения рабочих мест по отношению к источникам 
 естественного и искусственного освещения, при этом яркость бликов на экране ПЭВМ не должна превышать
 40 кд/м$^{2}$ и яркость потолка не должна превышать 200 кд/м$^{2}$.
 \item Показатель ослепленности для источников общего искусственного освещения в 
 производственных помещениях должен быть не более 20. Показатель дискомфорта в административно-общественных 
 помещениях не более 40, в дошкольных и учебных помещениях не более 15.
 \item Яркость светильников общего освещения в зоне углов излучения от 50 до 90° с вертикалью в 
 продольной и поперечной плоскостях должна составлять не более 200 кд/м$^{2}$, 
 защитный угол светильников должен быть не менее 40\textdegree{}.
 \item Светильники местного освещения должны иметь непросвечивающий отражатель с защитным углом не менее 40\textdegree{}.
 \item Следует ограничивать неравномерность распределения яркости в поле зрения пользователя ПЭВМ, 
 при этом соотношение яркости между рабочими поверхностями не должно превышать 3:1-5:1, 
 а между рабочими поверхностями и поверхностями стен и оборудования 10:1.
 \item Общее освещение при использовании люминесцентных светильников следует 
 выполнять в виде сплошных или прерывистых линий светильников, расположенных сбоку от рабочих мест, 
 параллельно линии зрения пользователя при рядном расположении видеодисплейных терминалов. 
 При периметральном расположении компьютеров линии светильников должны располагаться локализованно 
 над рабочим столом ближе к его переднему краю, обращенному к оператору.
 \item Коэффициент запаса (Кз) для осветительных установок общего освещения должен приниматься равным 1,4.
 \item Коэффициент пульсации не должен превышать 5\%.
 \item Для обеспечения нормируемых значений освещенности в помещениях для использования 
 ПЭВМ следует проводить чистку стекол оконных рам и светильников не реже двух раз в год и 
 проводить своевременную замену перегоревших ламп.
\end{enumerate}

\subsubsection{Требования к организации рабочих мест пользователей ПЭВМ}

Организация и оборудование рабочих мест с ПЭВМ для взрослых пользователей согласно~\cite{sanpin_2.4.1340-03}
включает в себя следующие требования:
\begin{enumerate}
 \item Высота рабочей поверхности стола для взрослых пользователей должна регулироваться в пределах 
 680-800 мм; при отсутствии такой возможности высота рабочей поверхности стола должна составлять 725 мм.
 \item Модульными размерами рабочей поверхности стола для ПЭВМ, на основании которых должны 
 рассчитываться конструктивные размеры, следует считать: ширину 800, 1000, 1200 и 1400 мм, глубину 800 
 и 1000 мм при нерегулируемой его высоте, равной 725 мм.
 \item Рабочий стол должен иметь пространство для ног высотой не менее 600 мм, шириной --- не менее 500 мм, 
 глубиной на уровне колен --- не менее 450 мм и на уровне вытянутых ног - не менее 650 мм.
 \item Конструкция рабочего стула должна обеспечивать:
 \begin{itemize}
  \item ширину и глубину поверхности сиденья не менее 400 мм;
  \item поверхность сиденья с закругленным передним краем;
  \item регулировку высоты поверхности сиденья в пределах 400-550 мм и углам наклона вперед до 15\textdegree{} и назад до 5\textdegree{};
  \item высоту опорной поверхности спинки 300±20 мм, ширину --- не менее 380 мм и радиус кривизны горизонтальной плоскости --- 400 мм;
  \item угол наклона спинки в вертикальной плоскости в пределах ±30\textdegree{};
  \item регулировку расстояния спинки от переднего края сиденья в пределах 260-400 мм;
  \item стационарные или съемные подлокотники длиной не менее 250 мм и шириной --- 50-70 мм;
  \item регулировку подлокотников по высоте над сиденьем в пределах 230±30 мм и внутреннего расстояния между 
  подлокотниками в пределах 350-500 мм.
 \end{itemize}
 \item Рабочее место пользователя ПЭВМ следует оборудовать подставкой для ног, имеющей ширину не менее 300 мм, глубину не менее 400 мм, 
 регулировку по высоте в пределах до 150 мм и по углу наклона опорной поверхности подставки 
 до 20\textdegree. Поверхность подставки должна быть рифленой и иметь по переднему краю бортик высотой 10 мм.
 \item Клавиатуру следует располагать на поверхности стола на расстоянии 100-300 мм от края, обращенного к пользователю, или на специальной, 
 регулируемой по высоте рабочей поверхности, отделенной от основной столешницы.
\end{enumerate}



\subsubsection{Оценка соответсвия автоматизированного рабочего места требованиям СанПин 2.2.2/2.4.1340-031 <<Гигиенические требования к персональным электронно-вычислительным машинам и организации работы>>}

Автоматизированное рабочее место исполнителя оборудовано ПЭВМ (ноутбук) и жидкокристаллическим 
видеодисплейным терминалом, расположено в помещении, где регулярно производятся влажная 
и сухая уборка, проветривание, а также поддерживается
оптимальная температура и влажность воздуха. В помещении большое окно, выходящее на восток,
что обеспечивает поступление большого количества естественного света в рабочие часы.
Оконный проем оборудован шторами. Искусственное освещение обеспечивается лампами
накаливания, а также настольной люминесцентной лампой. В помещении отсутствует шумящее оборудование.

В позволяет регулировать яркость и контрастность изображения и расположен таким образом, что 
естественный свет падает слева от экрана. Экран видеомонитора 
находится от глаз пользователя на расстоянии 500 мм.

Поверхность сиденья и спинки стула оборудована полумягким нескользящим покрытием, позволяет регулировать
высоту сиденья.


Рассчитаем показатели освещенности E на рабочем месте с помощью следующих формул~\cite{gost_8995}:
\begin{center}
$E = \dfrac{I}{r^{2}}cos(i),$
\end{center}
где~~~~~\ $\textit{I}$ --- сила света, кд;

$\textit{i}$ -- угол падения лучей света относительно нормали к поверхности;

$\textit{r}$ -- расстояние до источника света, м.

Формула для вычисления силы света:
\begin{center}
$I = \dfrac{F}{4\pi},$
\end{center}
где~~~~~\ $\textit{F}$ --- номинальный световой поток, лм.

Помещение оборудовано 1-ой люменисцентной лампой и 3-мя лампами накаливания. 
Световой поток люменисцентной лампы мощностью 10 Вт составляет около 400 Лм, 
одной лампы накаливания мощностью 60 Вт --- 710 Лм.

Сила света люменисцентной лампы:
\begin{center}
$I_{1} = \dfrac{400}{4\pi}= 31,83$ кд.
\end{center}

Сила света ламп накаливания:
\begin{center}
$I_{2} = \dfrac{3\times710}{4\pi}= 169,5$ кд.
\end{center}

Показатель освещенности на поверхности стола составляет:
\begin{center}
$E_{1} = \dfrac{31,83}{(0,3)^{2}}cos(30\textdegree{}) + \dfrac{169,5}{(1,5)^{2}}cos(45\textdegree{})= 359,56$ лк.
\end{center}

Показатель освещенности на поверхности экрана составляет:
\begin{center}
$E_{2} = \dfrac{31,83}{(0,3)^{2}}cos(60\textdegree{}) + \dfrac{169,5}{(1,6)^{2}}cos(60\textdegree{})= 209,94$ лк.
\end{center}

По результатам расчетов можно сделать вывод, что показатели освещенности соответствуют нормам, принятым 
в~\cite{sanpin_2.4.1340-03}: освещенность поверхности стола в диапазоне 300-500 лк, на поверхности экрана
не превышает 300 лк. 

<<Суммарное время регламентированных перерывов при 8-часовой рабочей смене
для категории работ Iа составляет 50 минут. В случаях, когда характер работы требует постоянного взаимодействия с ВДТ (набор текстов или ввод данных и т.п.) с напряжением внимания и сосредоточенности, при исключении возможности периодического переключения на другие виды трудовой деятельности, не связанные с ПЭВМ, рекомендуется организация перерывов на 10 - 15 мин. через каждые 45 - 60 мин. работы.
Продолжительность непрерывной работы с ВДТ без регламентированного перерыва не должна превышать 1 ч.>>~\cite{sanpin_2.4.1340-03}.

В процессе работы за ПЭВМ каждые 45 минут проводились 10-15 минутные перерывы
во избежание психо-эмоциональной перегрузки и для снятия физического напряжения. Во время 
перерывов осуществлялись легкая физическая разминка, а также упражнения для глаз.

В результате анализа можно сделать вывод, что организация рабочего места, на котором 
выполнялась дипломная работа, удовлетворяет перечисленным выше
требованиям правильной организации рабочего места оператора ЭВМ. Так как
и остальные условия работы в помещении являются удовлетворительными 
(микроклимат, освещение и т.д.), данное рабочее место работника можно считать соответствующим 
общим эргономическим требованиям.


% \end{document}






