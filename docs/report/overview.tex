На первом этапе научно-исследовательской работы необходимо было провести аналитический обзор информационных источников, рассмотреть существующие методы определения авторства исходного кода и различные подходы к решению такого рода задачи. 

В работе~\cite{frantz} представлен набор инструментов и техник, используемых для решения задач анализа авторства исходного кода, а также обзор некоторых наработок в данной предметной области. Кроме того, авторы приводят собственную классификацию проблем и подходов к их решению в рамках задачи деанонимизации авторов программного обеспечения. 

Среди проблем (задач) анализа авторства исходного кода выделены:
\begin{itemize}
  \item идентификация автора --- направлена на определение, принадлежит ли определенный фрагмент кода конкретному автору;
  \item характеристика автора --- базируется на анализе стиля программирования; 
  \item определение плагиата --- нахождение схожестей среди множества фрагментов файлов исходного кода;
  \item определение намерений автора --- был ли код изначально вредоносным или стал таковым в следствие программной ошибки;
  \item дискриминация авторов --- определение, был ли код написан одним автором или несколькими.
\end{itemize}

Подходы к решению вышеперечисленных проблем (задач):
\begin{itemize}
  \item анализ <<вручную>> --- данный подход включает в себя исследование и анализ фрагмента исходного кода экспертом; 
  \item вычисление схожести --- базируется на измерении и сравнении различных метрик или токенов для набора файлов исходного кода;
  \item статистический анализ --- в таком подходе используются статистические техники, такие как дискриминантный анализ и стилометрия, позволяющие определить различия между авторами;
  \item машинное обучение --- используются методы рассуждения на основе прецедентов и нейронные сети для классификации автора на базе некоторого набора метрик.
\end{itemize}

В работе~\cite{maevsky} предложен способ определения авторства программного обеспечения. в основе которого лежит система, состоящая из 100 метрик, отражающих <<почерк создателя>> программного обеспечения. На основе метрик составлен <<профиль почерка>> пяти разных программистов по текстам трех разработанных ими программных систем и проверено соответствие этому профилю других программ, написанных в том числе и другими программистами. Однако авторы привели сомнительные результаты вычислений, не указали способ составления <<профиля почерка>> и полученную точность, с которой программная система определяла авторство.

В~\cite{pellin} исходный код транслировался в абстрактные синтаксические деревья, после чего разбивался на функции. Дерево каждой функции принималось за отдельный документ с известным автором. Выборка, состоящая из такого рода деревьев подавалась на вход SVM-классификатору, оперирующему данными типа <<дерево>>. Классификатор обучался на файлах исходного кода двух авторов, в результате чего удалось достичь точности около 67-88\%.

В статье~\cite{burrows} рассматривался способ атрибуции исходного кода с использованием метода N-грамм. Вопрос определения авторства программ в данной работе рассматривался с точки зрения определения плагиата. В качестве выборки использовался набор из 1640 файлов исходного кода, написанных 100 авторами. Производилось ранжирование документов по схожести, после чего производилась оценка результатов. При этом составителям удалось успешно определить плагиат в 67\% случаев.

В~\cite{caliskan} применялся алгоритм классификации Random Forest~\cite{random_forest} и построение абстрактных синтаксических деревьев. Обучение и тестирование производилось для количества авторов от 250 до 1600. При этом удалось добиться высокой точности --- 94-98\%. Кроме того, авторы статьи выяснили в ходе работы, что сложнее определить авторов более простых примеров, нежели сложных программ, а также значительно выделяются авторы с большим опытом программирования на C/C++.

По результатам анализа вышеперечисленных источников было принято решение использовать подход, основанный на построении абстрактных синтаксических деревьев и классификации при помощи алгоритма Random Forest.





\subsection{Новый (подробный) обзор источников}










