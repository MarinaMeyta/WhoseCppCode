На первом этапе практики необходимо было провести подробный аналитический обзор информационных источников, 
рассмотреть существующие методы определения авторства исходного кода и различные подходы к решению 
такого рода задачи. 

В работе~\cite{frantz} представлен набор инструментов и техник, используемых для решения задач 
анализа авторства исходного кода, а также обзор некоторых наработок в данной предметной области. 
Кроме того, авторы приводят собственную классификацию проблем и подходов к их решению в рамках 
задачи деанонимизации авторов программного обеспечения. 

Frantzeskou~\cite{frantz} выделяет следующие проблемы (задачи) анализа авторства исходного кода:
\begin{itemize}
  \item идентификация автора --- направлена на определение, принадлежит ли определенный фрагмент кода конкретному автору;
  \item характеристика автора --- базируется на анализе стиля программирования; 
  \item определение плагиата --- нахождение схожестей среди множества фрагментов файлов исходного кода;
  \item определение намерений автора --- был ли код изначально вредоносным или стал таковым в следствие программной ошибки;
  \item дискриминация авторов --- определение, был ли код написан одним автором или несколькими.
\end{itemize}

Подходы к решению вышеперечисленных проблем (задач):
\begin{itemize}
  \item анализ <<вручную>> --- данный подход включает в себя исследование и анализ фрагмента исходного кода экспертом; 
  \item вычисление схожести --- базируется на измерении и сравнении различных метрик или токенов для набора файлов исходного кода;
  \item статистический анализ --- в таком подходе используются статистические техники, такие как дискриминантный анализ и стилометрия, позволяющие определить различия между авторами;
  \item машинное обучение --- используются методы рассуждения на основе прецедентов и нейронные сети для классификации автора на базе некоторого набора метрик.
\end{itemize}

В работе~\cite{frantz_2} предложен способ определения авторства программного обеспечения. в основе которого 
лежит статистический подсчет метрик, отражающих <<почерк создателя>> программного обеспечения. 
На основе метрик составлен <<профиль почерка>> программистов и вычисляется отклонение от данного профиля для 
каждого автора. Преимуществом данного метода является его независимость от языков программирования. Метод 
получил название SCAP (Source Code Author Profiles).

В~\cite{pellin} исходный код транслировался в абстрактные синтаксические деревья, 
после чего разбивался на функции. Дерево каждой функции принималось за отдельный документ с известным автором. 
Выборка, состоящая из такого рода деревьев подавалась на вход SVM-классификатору, 
оперирующему данными типа <<дерево>>. Классификатор обучался на файлах исходного кода двух авторов, в результате чего удалось достичь точности около 67-88\%.

В статье~\cite{burrows} рассматривался способ атрибуции исходного кода с использованием метода N-грамм. 
Вопрос определения авторства программ в данной работе рассматривался с точки зрения определения плагиата. 
В качестве выборки использовался набор из 1640 файлов исходного кода, написанных 100 авторами. Позднее 
удалось улучшить точность классификации данной модели до 77\% за счет применения рейтинговых схем.~\cite{burrows_big}

Характерная особенность описанных выше работ состоит в том, что тестирование производилось на очень 
ограниченном наборе данных, в основном --- студенческих работ или программных проектов, реализующих 
одну и ту же заранее поставленную в условиях эксперимента задачу.

В~\cite{caliskan} применялся алгоритм классификации Random Forest~\cite{random_forest} и построение 
абстрактных синтаксических деревьев. Обучение и тестирование производилось для 
количества авторов от 250 до 1600. При этом удалось добиться высокой точности --- 94-98\%. 
Кроме того, авторы статьи выяснили в ходе работы, что сложнее определить авторов более простых примеров, 
нежели сложных программ, а также значительно выделяются авторы с большим опытом программирования на C/C++. 
В своей дальнейшей работе~\cite{git_blame} авторы предложили применение данного подхода для анализа неполных, 
некомпилируемых образцов кода. 

На основании проведенного исследования было решено опробовать подход, основанный на вычислении 
статистических метрик, характеризующих авторский стиль написания программ, и методов машинного обучения. 

Сравнительная таблица с подробным описанием данных информационных источников и используемых в них методов 
приведена в приложении~Б.





