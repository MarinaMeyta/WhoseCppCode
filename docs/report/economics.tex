\subsection{Обоснование необходимости проводимого исследования}

Вследствие активного развития информационных технологий возрастает и количество преступлений в информационной сфере.
Зачастую злоумышленники, как внешние, так и внутренние, используют самостоятельно разработанные программы для осуществления 
атак на информационные ресурсы.
Системы, способные идентифицировать разработчиков вредоносного ПО, могут внести существенный вклад в развитие компьютерной
криминалистики, а также оказывать помощь в исследовании вопросов интеллектуальной собственности среди разработчиков программного
обеспечения.

Цель настроящей дипломной работы --- разработать ПО, способное идентифицировать автора программы по исходному коду, с перспективой его
дальнейшего применения в борьбе с киберпреступностью, в области лицензионных, патентных, и иных судебных разбирательств.


\subsection{Организация и планирование работы}

Основные задачи организации и планирования работ:
\begin{itemize}
 \item определение объема предстоящих работ;
 \item определение основных этапов работ;
 \item установление сроков выполнения запланированных работ;
 \item определение необходимых денежных, материальных и трудовых ресурсов.
\end{itemize}


При выполнении дипломной работы были задействованы следующие лица:
\begin{itemize}
 \item руководитель (рук.);
 \item разработчик (разр.).
\end{itemize}

Месячный оклад студента, не являющегося дипломированным специалистом, составляет 2324,40 рублей.
С учетом 24 рабочих дней и 6-часового рабочего дня стоимость одного часа работ равна 16,14 рублей.
Месячный оклад руководителя с ученой степенью кандидата наук и должностью доцента~\cite{rector_3106} составляет 14800 рублей.
Стоимость одного часа работ с учетом 24-ех 6-часовых рабочих дней равна 102,78 рублей.   

Руководитель работы оказывает помощь разработчику в планировании работ в период проектирования, рекомендует
необходимую литературу, проводит консультации разработчика, осуществляет контроль над выполнением всех 
намеченных этапов работы. Разработчик реализует объем работ, установленный в техническом задании.

График работ приведен в таблице~\ref{tab:job_is_done_1}. Зная длительность цикла каждого этапа и возможность их параллельно-последовательного выполнения, можно рассчитать срок завершения 
планируемых работ и составить ленточный и сетевой графики плана их выполнения (табл.~\ref{tab:job_is_done_2} и~\ref{tab:job_is_done_3}).

\begin{table}[!ht]
\caption{График выполнения работ}
\centering
\includegraphics[page=1, width=1\linewidth]{tables/economics/schedule.pdf}
\label{tab:job_is_done_1}
\end{table}


\begin{table}[!ht]
\caption{Ленточный график загрузки участников работ}
\centering
\includegraphics[page=1, width=1\linewidth]{tables/economics/schedule_2.pdf}
\label{tab:job_is_done_2}
\end{table}

\begin{table}[!ht]
\caption{Календарный график загрузки участников}
\centering
\includegraphics[page=1, width=1\linewidth]{tables/economics/schedule_3.pdf}
\label{tab:job_is_done_3}
\end{table}


\subsection{Определение сметной стоимости проекта}
\subsubsection{Общие положения}

Смета затрат для данной работы состоит из расходов, которые включают в себя следующие статьи:

\begin{itemize}
\item затраты на оборудование и амортизацию;
\item расходы на оплату труда и отчисления на социальные нужды;
\item затраты на основные и вспомогательные материалы;
\item затраты на электроэнергию.
\end{itemize}
\subsubsection{Затраты на оборудование и амортизацию}

Основным оборудованием при проведении работы являются компьютер и принтер, которые 
постановлением Правительства Российской Федерации от 1.01.02 г.~\textnumero~1 отнесены ко второй амортизационной группе – 
<<имущество со сроком полезного использования свыше 2 лет до 3 лет включительно>>~\cite{amort}. 
Месячная норма амортизации составляет 2,8\% и для ноутбука, и для принтера.

Результаты расчётов амортизационных отчислений приведены в таблице \ref{tab:job_is_done_4}.

\begin{table}[!ht]
\caption{Смета затрат на оборудование}
\centering
\includegraphics[page=1, width=1\linewidth]{tables/economics/schedule_4.pdf}
\label{tab:job_is_done_4}
\end{table}

\subsubsection{Расходы на оплату труда и отчисления на социальные нужды}

Статья затрат учитывает выплаты по заработной плате за выполненную работу, 
вычисленные на основании тарифных ставок и должностных окладов в соответствии с принятой в 
организации-разработчике системой оплаты труда. В этой статье также отражаются премии, надбавки и доплаты за 
условия труда, оплата ежегодных отпусков, выплата районного коэффициента и некоторые другие расходы. 
Отчисления на социальные нужды учитывают страховые взносы.

Результаты расчёта расходов на оплату труда участников проекта представлены в таблице~\ref{tab:job_is_done_5}.

\begin{table}[!ht]
\caption{Расчет расходов на оплату труда участников проекта}
\centering
\includegraphics[page=1, width=1\linewidth]{tables/economics/schedule_5.pdf}
\label{tab:job_is_done_5}
\end{table}


\subsubsection{Затраты на основные и вспомогательные материалы}

Статья включает расходы по приобретению и доставке основных и вспомогательных материалов, необходимых для опытно-экспериментальной проработки 
решения, для изготовления макета или опытного оборудования. Сюда включаются и стоимость необходимых материалов для изготовления образцов и 
макетов, и материалов необходимых для оформления требуемой документации.

Размер транспортно-заготовительных расходов (ТЗР), определяемый в процентах от стоимости, примем 10\%. 
Стоимость вспомогательных материалов принимается 10\% от стоимости основных материалов с учётом ТЗР. 
Результаты расчёта стоимости материалов представлены в \ref{tab:eco_6}.

\begin{table}[!ht]
\caption{Расчёт затрат на основные и вспомогательные материалы}
\centering
\includegraphics[page=1, width=1\linewidth]{tables/economics/econom.pdf}
\label{tab:eco_6}
\end{table}


\subsubsection{Расходы на электроэнергию}

Статья включает затраты по электроэнергии на технологические нужды. В настоящее время тариф на электроэнергию для населения г. Томска на 2017 год составляет 2,17 руб./ кВт ч. Тариф введен 
приказом от 23.12.2016 г. \textnumero~6-840 <<О тарифах на электрическую энергию для населения и 
потребителей, приравненных к категории население по Томской области на 2017 год>>~\cite{electr}, 
принятый департаментом тарифного регулирования Томской области.

Результаты расчётов приведены в \ref{tab:eco_7}.

\begin{table}[!ht]
\caption{Затраты на электроэнергию}
\centering
\includegraphics[page=1, width=1\linewidth]{tables/economics/econom_2.pdf}
\label{tab:eco_7}
\end{table}


\subsubsection{Накладные расходы}

Результаты расчёта накладных расходов приведены в таблице~\ref{tab:eco_8}.

\begin{table}[!ht]
\caption{Накладные расходы}
\centering
\includegraphics[page=1, width=1\linewidth]{tables/economics/econom_3.pdf}
\label{tab:eco_8}
\end{table}


\subsubsection{Сводная смета затрат}

На основании всех произведённых расчётов составим сводную смету затрат на выполнение работы в виде таблицы \ref{tab:eco_9}.

\begin{table}[!ht]
\caption{Сводная смета затрат}
\centering
\includegraphics[page=1, width=1\linewidth]{tables/economics/econom_4.pdf}
\label{tab:eco_9}
\end{table}


\subsection{Научно-технический эффект}
Количественная оценка научно-технического уровня может быть произведена путём расчёта результативности участников разработки по формуле:
$$K_{\textit{ну}} = \sum_{i=1}^{n}(K_{\textit{ду}}\cdot d_{i}),$$
где~~~~~\ $K_\textit{ну}$ – коэффициент научного или научно-технического уровня;

$K_\textit{дуi}$ – коэффициент достигнутого уровня $\textit{i}{}$-го фактора;

$d_{i}$– значимость $i$-го фактора;

$\textit{n}$ – количество факторов.

Весовые коэффициенты \textit{d} для каждого из факторов устанавливались экспертным путём. При этом сумма коэффициентов значимости по всем факторам равна единице. Коэффициенты достигнутого уровня факторов также установлены экспертным путём.

\begin{table}[!ht]
\caption{Оценка научно-технического уровня разработки}
\centering
\includegraphics[page=1, width=1\linewidth]{tables/economics/econom_5.pdf}
\label{tab:eco_10}
\end{table}

Рассчитанный коэффициент научно-технической результативности равен 0,7475. Полученное значение достаточно высоко, что говорит об эффективности проведённых работ выше среднего, однако отмечается необходимость дальнейшего развития проекта для достижения завершённости полученных результатов.


\clearpage