Считается, что у каждого программиста есть свои специфические профессиональные приемы, привычки, методы написания программного кода, свой так называемый «стиль программирования» и иные признаки, выдающие автора. Языки программирования высокого уровня предоставляют некоторую свободу для программистов при выборе способа расстановки скобок, табуляции и пробелов, обозначения переменных и функций. Таким образом, основываясь на методах стилометрии --- исследовании стилистики, включающего в себя статистический анализ текста, --- можно определить с некоторой точностью авторство исходного кода на заданной конечной выборке программистов (например, разработчиков одной компании).

Определение авторства исходного кода представляет собой актуальную задачу в сфере информационной безопасности, лицензирования в области разработки программного обеспечения, а также может оказать существенную помощь во время судебных разбирательств, при решении вопросов об интеллектуальной собственности и плагиате.

Работа посвящена исследованию различных методов анализа текста и определения его авторства, основанных на стилометрии и применительных к деанонимизации автора исходного кода программ.

На первом этапе будет проведен аналитический обзор различных информационных источников по теме методов анализа текста, основанных на стилометрии и применительных к анализу исходного кода программ на языках высокого уровня с целью определения авторства программного обеспечения.

На втором этапе будет разработан алгоритм определения авторства программного обеспечения на основе стилометрического анализа исходного кода программ, приведено его формализованное описание.

На третьем этапе будет выполнена программная реализация разработанного алгоритма определения авторства исходного кода программ на языках высокого уровня, выбраны критерии оценки эффективности, а также необходимые тестовые данные. Итогом третьего этапа работы будет проведение исследования эффективности разработанного алгоритма с приведением полученных результатов и их анализом. 

На четвертом этапе будут проведены оценка и анализ всех результатов, полученных в ходе исследования. Итогом работы станет написание научно-исследовательской статьи, в которой будут рассмотрены все вышепоставленные вопросы и задачи, а также результаты исследования. 
