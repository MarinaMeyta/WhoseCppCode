
\section*{ТЕХНИЧЕСКОЕ ЗАДАНИЕ\\
на выполнение научно-исследовательских работ (НИР) по теме:\\
<<Определение авторства исходного кода>>}


\section{Основание для проведения НИР}

1.1\hspace{5mm}Положение о системе организации научно-исследовательской работы студентов ТУСУРа от 30.08.2013, положение о системе организации научно-исследовательской работы студентов ФБ.
 
1.2\hspace{5mm}Начало работ: 1 сентября 2016 г.

1.3\hspace{5mm}Срок окончания работ: 22 декабря 2016 г.

\section{Исполнители НИР}

Студент Томского государственного университета систем управления и радиоэлектроники (ТУСУР), кафедры комплексной информационной безопасности и электронно-вычислительных систем (КИБЭВС), 5 курса, гр. 722 Мейта Марина Валерьевна.

Научный руководитель: кандидат технических наук, доцент кафедры безопасности информационных систем (БИС) Романов Александр Сергеевич.

\section{Цель и задачи выполнения НИР}

Целью работы является разработка алгоритма для определения авторства программного обеспечения, основанного на стилометрическом анализе исходного кода программ на языках высокого уровня. Для достижения данной цели были поставлены следующие задачи:
\begin{itemize}
  \item обзор существующих исследований, разработок, методов стилометрического анализа текста, в том числе исходного кода программ; 
  \item разработка алгоритм анализа исходного кода программ с применением стилометрии для определения авторства программного обеспечения;
  \item создание программной реализации разработанного алгоритма;
  \item исследование эффективности разработанного алгоритма анализа исходных кодов.
\end{itemize}
Объект исследования: деанонимизация автора программного обеспечения.

Предмет исследования: стилометрия исходного кода программ на языках высокого уровня.   

\section{Научные и научно-технические результаты выполнения НИР}

4.1\hspace{5mm}При выполнении НИР должны быть получены следующие научно-технические результаты:

1)\hspace{5mm}результаты исследования методов определения авторства программного обеспечения, представленные в отчете о НИР, содержащим, в том числе:

а)\hspace{5mm}обзор и анализ современной научно-технической литературы, рассматривающей различные методы анализа текста и определения его авторства, основанные на стилометрии и применительные к деанонимизации автора исходного кода программ;

б)\hspace{5mm}формализованное описание разрабатываемого алгоритма для определения авторства исходного кода программ;

в)\hspace{5mm}код программы, реализующей алгоритм определения авторства исходного кода;

г)\hspace{5mm}исследование эффективности разработанного алгоритма;

д)\hspace{5mm}обобщение и выводы по результатам НИР;

е)\hspace{5mm}рекомендации и предложения по использованию результатов НИР.

2)\hspace{5mm}программная реализация разработанного алгоритма определения авторства исходного кода программ на языках высокого уровня с применением методов стилометрии.

\section{Основные требования к выполнению НИР}
\subsection{Требования к выполняемым работам}

5.1.1\hspace{5mm}В ходе выполнения НИР должны быть выполнены:

5.1.1\hspace{5mm}аналитический обзор существующих методов анализа текста, основанных на стилометрии;

5.1.2\hspace{5mm}разработка алгоритма определения авторства программного обеспечения с применением одного из изученных методов анализа текста;

5.1.3\hspace{5mm}программная реализация разработанного алгоритма;

5.1.4\hspace{5mm}исследование эффективности разработанного алгоритма;

5.1.5\hspace{5mm}написание научной статьи по проблеме, исследуемой в рамках НИР, объемом более 1 листа А4 или эквивалентном в листах другого размера.

\subsection{Требования к разрабатываемой документации}
5.2.1\hspace{5mm}В ходе работы должны быть разработаны и согласованы с научным руководителем и ответственным за НИР	 следующие документы:

5.2.1.1\hspace{5mm}заключительный отчет о НИР, оформленный в соответствии с ОС ТУСУР 01-2013;

5.2.1.2\hspace{5mm}проект технического задания на проведение НИР.

\section{Перечень и сроки выполнения этапов}
\subsection{Наименование этапов и выполняемые работы}
\subsection*{Этап 1. Обзор предметной области, постановка задачи}

1.1\hspace{5mm}Аналитический обзор информационных источников по теме стилометрического анализа текста.

1.2\hspace{5mm}Исследование методов стилометрического анализа текста применительно к исходному коду программ с целью деанонимизации автора исходного кода.

1.3\hspace{5mm}Подведение итогов этапа НИР.

\subsection*{Этап 2. Разработка алгоритма}
2.1\hspace{5mm}Разработка алгоритма определения авторства программного обеспечения на основе стилометрического анализа исходного кода программ;

2.2\hspace{5mm}Подведение итогов этапа НИР.

\subsection*{Этап 3. Программная реализация разработанного алгоритма}
3.1\hspace{5mm}Программная реализация разработанного алгоритма определения авторства программного обеспечения на основе стилометрического анализа исходного кода программ.

3.2\hspace{5mm}Исследование эффективности разработанного алгоритма, включающее в себя выбор критериев эффективности, формирование тестовых данных и проведение вычислительных экспериментов.

3.3\hspace{5mm}Подведение итогов этапа НИР.

\subsection*{Этап 4. Подведение итогов}
4.1\hspace{5mm}Обобщение и оценка полученных результатов, включающее в себя:
\begin{itemize}
  \item обобщение результатов исследований, в том числе исследования эффективности разработанного алгоритма определения авторства исходного кода программ;
  \item анализ выполнения требований технического задания на НИР;
  \item оценка полноты решения задач и достижения поставленных целей НИР.
\end{itemize}

4.2\hspace{5mm}Разработка отчета о НИР.

4.3\hspace{5mm}Подготовка результатов исследования к публикации.

\section{Предполагаемое использование результатов НИР}

7.1\hspace{5mm}Результаты проведенных НИР могут быть использованы специалистами по информационной безопасности, юристами и иными заинтересованными лицами для определения автора вредоносного программного обеспечения, подтверждения интеллектуальной собственности, отслеживания исходного кода после кибератаки, во время судебных разбирательств и т.п.

\section{Порядок выполнения и приемки этапов НИР}

8.1\hspace{5mm}Работы должны выполняться поэтапно в соответствии с настоящим техническим заданием.

8.2\hspace{5mm}Сдача и приемка выполненных работ (этапов работ) осуществляется в соответствии с календарным планом работ.

8.3\hspace{5mm}При приемке оценивается научно-технический уровень исследований, соответствие полученных результатов требованиям настоящего технического задания, обоснованность предлагаемых решений по реализации и использованию результатов НИР.

\begin{center}
  КАЛЕНДАРНЫЙ ПЛАН ВЫПОЛНЕНИЯ РАБОТ
\end{center}

\begin{table}[h]
\label{tab:3}
\begin{center}
\begin{tabular}{|c|c|c|}
\hline
Номер этапа & Начало периода & Окончание периода \\ 
\hline
Этап 1 & 01.09.2016 & 10.10.2016 \\
\hline
Этап 2 & 11.10.2016 & 07.11.2016 \\
\hline
Этап 3 & 08.11.2016 & 05.12.2016 \\
\hline
Этап 4 & 06.12.2016 & 22.12.2016 \\
\hline
\end{tabular}
\end{center}
\end{table}

\titleformat{\section}{\centering\normalfont\normalsize}{\thesection}{1.0em}{}

\newpage
\section*{
ЛИСТ СОГЛАСОВАНИЯ\\
к проекту технического задания\\
на выполнение научно-технической работы (НИР) по теме:\\
<<Определение авторства исходного кода>>}


\begin{flushleft}
\underline{\hspace{4cm}} Костюченко Е.Ю, заказчик НИР\\
\underline{\hspace{4cm}} Романов А.С, научный руководитель НИР\\
\underline{\hspace{4cm}} Мейта М.В., ответственный исполнитель НИР\\
\end{flushleft}

\newpage
\section*{ПОЯСНИТЕЛЬНАЯ ЗАПИСКА}
Считается, что у каждого программиста есть свои специфические профессиональные приемы, привычки, методы написания программного кода, свой так называемый «стиль программирования» и иные признаки, выдающие автора. Языки программирования высокого уровня предоставляют некоторую свободу для программистов при выборе способа расстановки скобок, табуляции и пробелов, обозначения переменных и функций. Таким образом, основываясь на методах стилометрии --- исследовании стилистики, включающего в себя статистический анализ текста, --- можно определить с некоторой точностью авторство исходного кода на заданной конечной выборке программистов (например, разработчиков одной компании).

Определение авторства исходного кода представляет собой актуальную задачу в сфере информационной безопасности, лицензирования в области разработки программного обеспечения, а также может оказать существенную помощь во время судебных разбирательств, при решении вопросов об интеллектуальной собственности и плагиате.

Работа посвящена исследованию различных методов анализа текста и определения его авторства, основанных на стилометрии и применительных к деанонимизации автора исходного кода программ.

На первом этапе будет проведен аналитический обзор различных информационных источников по теме методов анализа текста, основанных на стилометрии и применительных к анализу исходного кода программ на языках высокого уровня с целью определения авторства программного обеспечения.

На втором этапе будет разработан алгоритм определения авторства программного обеспечения на основе стилометрического анализа исходного кода программ, приведено его формализованное описание.

На третьем этапе будет выполнена программная реализация разработанного алгоритма определения авторства исходного кода программ на языках высокого уровня, выбраны критерии оценки эффективности, а также необходимые тестовые данные. Итогом третьего этапа работы будет проведение исследования эффективности разработанного алгоритма с приведением полученных результатов и их анализом. 

На четвертом этапе будут проведены оценка и анализ всех результатов, полученных в ходе исследования. Итогом работы станет написание научно-исследовательской статьи, в которой будут рассмотрены все вышепоставленные вопросы и задачи, а также результаты исследования. 

