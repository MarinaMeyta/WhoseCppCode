\documentclass[russian,14pt,floatsection,nocolumnsxix, nocolumnxxxii,nocolumnxxxi]{eskdtext}
\usepackage[utf8x]{inputenc}

% закомментировать для рамок 
\usepackage[numbertop, numbercenter]{eskdplain}

% для объединения строк в таблице
\usepackage{multirow}



% для размера колонок
\usepackage{tabularx}
\usepackage{lscape}
\newcolumntype{n}{>{\hsize=.4\hsize}X}
\newcolumntype{m}{>{\hsize=.2\hsize}X}
% tables on top
\makeatletter
\setlength{\@fptop}{0pt}
\makeatother

\usepackage{setspace}
\onehalfspacing % полуторный интервал для всего текста

% - Подключаем шрифты из пакета scalable-cyrfonts-tex
\usepackage{cyrtimes}

% - Отступ красной строки
\setlength{\parindent}{1.25cm}

% - Убирает точку в списке литературы
\makeatletter
\def\@biblabel#1{#1 }

% ограничение для оглавления
%\usepackage{tocvsec2}
\setcounter{tocdepth}{2}

% - Точки для всех пунктов в оглавлении
\renewcommand*{\l@section}{\@dottedtocline{1}{1.5em}{2.3em}}
\renewcommand*{\l@subsection}{\@dottedtocline{1}{1.5em}{2.3em}}
% \renewcommand*{\l@subsubsection}{\@dottedtocline{1}{1.5em}{2.3em}}

% - Для переопределения списков
\renewcommand{\theenumi}{\arabic{enumi}}
\renewcommand{\labelenumi}{\theenumi)}
\makeatother

\usepackage{enumitem}
\setlist{nolistsep, itemsep=0.3cm,parsep=0pt}

% - ГОСТ списка литературы
\bibliographystyle{utf8gost705u}


% - Верикальные отступы заголовков 
\ESKDsectSkip{section}{1em}{1em}
\ESKDsectSkip{subsection}{1em}{1em}
\ESKDsectSkip{subsubsection}{1em}{1em}

% - Изменение заголовков
\usepackage{titlesec}
\titleformat{\section}{\normalfont\normalsize\centering}{\thesection}{1.0em}{}
\titleformat{\subsection}{\normalfont\normalsize\centering}{\thesubsection}{1.0em}{}
\titleformat{\subsubsection}{\normalfont\normalsize\centering}{\thesubsubsection}{1.0em}{}
\titleformat{\paragraph}{\normalfont\normalsize\centering}{\theparagraph}{1.0em}{}

% - Оставим место под ТЗ 
\setcounter{page}{1}

% - Для больших таблиц
\usepackage{longtable}
\usepackage{tabularx}
\renewcommand{\thetable}{\thesection.\arabic{table}}

% for centering tabularx
\newcolumntype{Y}{>{\centering\arraybackslash}X}

% - Используем графику в документе
\usepackage{graphicx}
\graphicspath{{images/}}
\renewcommand{\thefigure}{\thesection.\arabic{figure}}
% to have figure on top
%
\makeatletter
\setlength{\@fptop}{0pt}
\makeatother

% - Счётчики
\usepackage{eskdtotal}

% - Выравнивание по ширине
\sloppy

% - Разрешить перенос двух последних букв слова
\righthyphenmin=2

% - Оформление списков
\RequirePackage{enumitem}
\renewcommand{\alph}[1]{\asbuk{#1}}
\setlist{nolistsep}
\setitemize[1]{label=--, fullwidth, itemindent=\parindent, 
  listparindent=\parindent}% для дефисного списка
\setitemize[2]{label=--, fullwidth, itemindent=\parindent, 
  listparindent=\parindent, leftmargin=\parindent}
\setenumerate[1]{label=\arabic*), fullwidth, itemindent=\parindent, 
  listparindent=\parindent}% для нумерованного списка
\setenumerate[2]{label=\alph*), fullwidth, itemindent=\parindent, 
  listparindent=\parindent, leftmargin=\parindent}% для списка 2-ой ступени, который будет нумероваться а), б) и т.д.
  
% - Оформляем листинг кода (не использовать комментарии на русском!)
\usepackage{listings}  
\lstset{basicstyle=\ttfamily\scriptsize}
\lstset{extendedchars=\true}

% - выводим текст как есть с размером шрифта scriptsize
\makeatletter
\def\verbatim{\scriptsize\@verbatim \frenchspacing\@vobeyspaces \@xverbatim}
\makeatother

% - Вставка pdf
\usepackage[enable-survey]{pdfpages}

%межстрочный интервал
\usepackage{setspace}
\linespread{1.5}

%фамилии для рамок
\author{\ESKDfontII Мейта М.В.}
\ESKDchecker{\ESKDfontII Романов А.С.}
\renewcommand{\ESKDcolumnXfIVname}{Реценз.}
\ESKDcolumnXIfIV{\ESKDfontII Тушминцев А.А.}
\ESKDnormContr{\ESKDfontII Якимук А.Ю.}
\ESKDapprovedBy{\ESKDfontII Шелупанов А.А.}
\ESKDcolumnI{\ESKDfontIII Определение авторства исходного кода}
\ESKDcolumnIX{\ESKDfontIII ТУСУР, ФБ, каф.~КИБЭВС, гр.~722}
\ESKDsignature{КИБЭВС.58.29.29 ПЗ}


\begin{document}
\newpage
% \ESKDthisStyle{empty}
%  \includepdf[pages={1}]{title}
 \newpage
\ESKDthisStyle{empty}

\begin{center}
 \textbf{МИНИСТЕРСТВО ОБРАЗОВАНИЯ И НАУКИ РОССИЙСКОЙ ФЕДЕРАЦИИ}\\
 Федеральное государственное бюджетное образовательное учреждение высшего образования\\
 <<ТОМСКИЙ ГОСУДАРСТВЕННЫЙ УНИВЕРСИТЕТ СИСТЕМ УПРАВЛЕНИЯ И РАДИОЭЛЕКТРОНИКИ>> (ТУСУР)\\
 Кафедра комплексной информационной безопасности электронно-вычислительных систем (КИБЭВС)\\
\end{center} 

\vfill

\begin{flushright}
\begin{minipage}{0.45\textwidth}
 \begin{flushleft}
  К ЗАЩИТЕ ДОПУСТИТЬ\\
  заведующий каф. КИБЭВС\\
  д-р техн. наук, проф.\\
  \underline{\hspace{3cm}}А.А. Шелупанов \\
  <<\underline{\hspace{1cm}}>>\underline{\hspace{3cm}}2017г.\\
 \end{flushleft}
\end{minipage}
\end{flushright}

\vfill



\begin{center}
<<ОПРЕДЕЛЕНИЕ АВТОРСТВА ИСХОДНОГО КОДА>>

Дипломная работа по направлению 10.05.03 --

Информационная безопасность автоматизированных систем

КИБЭВС.58.29.29 ПЗ
\end{center}


\vfill


СОГЛАСОВАНО
\vspace{0.01cm}
\begin{singlespace}
 \begin{minipage}[left]{0.40\linewidth}
 Консультант по экономике:\\
 ст. преподаватель каф. КИБЭВС \\
 \underline{\hspace{2.5cm}}С.В. Глухарева \\
 <<\underline{\hspace{1cm}}>>\underline{\hspace{3cm}} 2017г.\\

 Консультант по безопасности\\ жизнедеятельности:\\
 канд. техн. наук, доцент каф. КИБЭВС\\
 \underline{\hspace{2.5cm}}Е.М. Давыдова\\
 <<\underline{\hspace{1cm}}>>\underline{\hspace{3cm}} 2017г.\\
 \end{minipage}
 \hfill
 \begin{minipage}[left]{0.5\linewidth}
  \vspace{0.7cm}
  Студентка гр. 722 \\
  \underline{\hspace{3cm}}М.В. Мейта  \\
 <<\underline{\hspace{1cm}}>>\underline{\hspace{3cm}} 2017г.\\
 \vspace{0.3cm}\\ 
  Руководитель: \\
  канд. техн. наук, доцент каф. БИС \\
  \underline{\hspace{3cm}} А.С. Романов \\
  <<\underline{\hspace{1cm}}>>\underline{\hspace{3cm}}2017г.\\
 \end{minipage}
\end{singlespace}



% \begin{flushright}
% \begin{minipage}{0.45\textwidth}
%  \begin{flushleft}
%   Студент гр. 722 \\
%   \underline{\hspace{3cm}} Мейта М.В. \\
%   <<\underline{\hspace{1cm}}>>\underline{\hspace{3cm}}2017г.\\
%  \end{flushleft}
% \end{minipage}
% \end{flushright}
% 
% \vfill
% 
% \begin{flushright}
% \begin{minipage}{0.45\textwidth}
%  \begin{flushleft}
%   Руководитель: \\
%   канд. техн. наук, доцент каф. БИС \\
%   \underline{\hspace{3cm}} Романов А.С. \\
%   <<\underline{\hspace{1cm}}>>\underline{\hspace{3cm}}2017г.\\
%  \end{flushleft}
% \end{minipage}
% \end{flushright}

\vfill

\begin{center}
 Томск 2017
\end{center}

 \newpage
 \ESKDthisStyle{empty}
 \includepdf{practise_task}
 \newpage
\ESKDthisStyle{empty}
\paragraph{\hfill РЕФЕРАТ \hfill}
Отчет содержит \ESKDtotal{page} страниц, \ESKDtotal{figure} рисунков, \ESKDtotal{table} таблиц, \ESKDtotal{bibitem} источников, \ESKDtotal{appendix} приложение.

СТИЛОМЕТРИЯ, ИСХОДНЫЙ КОД, ДЕАНОНИМИЗАЦИЯ АВТОРА, C++, КЛАССИФИКАЦИЯ, PYTHON, SKLEARN, RANDOM FOREST CLASSIFIER, LATEX.

Цель работы --- разработка алгоритма для определения авторства программного обеспечения, основанного на стилометрическом анализе исходного кода программ на языках высокого уровня.

В рамках научно-исследовательской работы на текущий семестр были поставленны следующие задачи: 
\begin{itemize}
  \item обзор существующих исследований, разработок, методов стилометрического анализа тек-
ста, в том числе исходного кода программ;
  \item разработка алгоритм анализа исходного кода программ с применением стилометрии для
определения авторства программного обеспечения;
  \item создание программной реализации разработанного алгоритма;
  \item исследование эффективности разработанного алгоритма анализа исходных кодов.
\end{itemize}

Объект исследования: деанонимизация автора программного обеспечения. 

Предмет исследования: стилометрия исходного кода программ на языках высокого уровня.

В результате работы было выполнено слудующее:

\begin{itemize}
  \item произведен аналитический обзор существующих методов анализа исходного кода программ с целью деанонимизации автора;
  \item выбран набор признаков для классификации авторов программного кода на языке С++; 
  \item в качестве алгоритма классификации выбран Random Forest Classifier;
  \item разработан алгоритм определения авторства исходного кода программ на языке С++;
  \item произведена программная реализация разработанного алгоритма на языке программирования высокого уровня Python;
  \item подготовлен тестовый набор данных;
  \item выбраны критерии оценки эффективности разработанного алгоритма;
  \item произведены вычислительные эксперименты на данном наборе данных;
  \item сделаны некоторые выводы на основе полученных результатов.
\end{itemize}

Отчет о НИР выполнен согласно ОС ТУСУР 01-2013~\cite{ostusur} при помощи системы компьютерной вёрстки \LaTeX. 

 \newpage
\ESKDthisStyle{empty}
\paragraph{\hfill РЕФЕРАТ \hfill}
Practise report contains \ESKDtotal{page} pages, \ESKDtotal{figure} pictures, \ESKDtotal{table} tables,
\ESKDtotal{bibitem} sources, \ESKDtotal{appendix} appendix.

STYLOMETRY, SOURCE CODE, AUTHORSHIP ATTRIBUTION, C++, CLASSIFICATION, PYTHON, SKLEARN,
RANDOM FOREST CLASSIFIER, LATEX.

Цель работы --- разработка алгоритма для определения авторства программного обеспечения, 
основанного на стилометрическом анализе исходного кода программ на языках высокого уровня.

The aim of this work is a software development of the tool for deanonimization of programmers, 
based on stylometry analysis of source code written in high-level programming languages.  
 
 % - содержание
 \newpage
%  \ESKDstyle{plain}
\ESKDstyle{formIIab}
\ESKDthisStyle{formII}
 \tableofcontents

% \newpage
% \ESKDstyle{plain}
%  \addcontentsline{toc}{section}{Техническое задание}
% \input{tz/tz_main}


\titleformat{\section}{\centering\normalfont\normalsize}{\thesection}{1.0em}{}
\titleformat{\subsection}{\centering\normalfont\normalsize}{\thesubsection}{1.0em}{}
\titleformat{\subsubsection}{\centering\normalfont\normalsize}{\thesubsection}{1.0em}{}


\newpage
% \ESKDstyle{plain}
\setcounter{section}{0}
\section*{Введение}
\addcontentsline{toc}{section}{Введение}
С распространением применения компьютерных систем и сетей возросло и количество преступлений в 
информационной сфере. Существует множество разновидностей кибератак --- различные компьютерные вирусы, 
трояны, несанкционированное копирование данных с кредитных карт, DDoS-атаки и многое другое. 
Возможность деанонимизации авторов вредоносного программного обеспечения может внести существенный вклад 
в развитие компьютерной криминалистики.

Целью данной работы является исследование методов стилометрии --- статистического анализа текста 
для выявления его стилистических особенностей, а также методов машинного обучения для решения задачи 
деанонимизации разработчика по исходному коду программного обеспечения.

Определение авторства исходного кода представляет собой актуальную задачу в сфере информационной 
безопасности, лицензирования в области разработки программного обеспечения, а также может оказать 
существенную помощь во время судебных разбирательств, при решении вопросов об интеллектуальной 
собственности и плагиате.


\newpage
\section{Кафедра КИБЭВС}
Местом прохождения практики была выбрана кафедра ТУСУРа – КИБЭВС (Кафедра комплексной информационной безопасности электронно-
вычислительных систем).

Кафедра организована в ТУСУР в 1971 году как кафедра «Конструирования и производства электронно-вычислительной аппаратуры»
(КиПЭВА) вскоре переименованной в кафедру «Конструирования электронно-вычислительной аппаратуры» (КЭВА). 

21 сентября 1999 г. в связи с открытием новой актуальной специальности 090105 – «Комплексное обеспечение информационной
безопасности автоматизированных систем» кафедра КЭВА была переименована в кафедру «Комплексной информационной безопасности
электронно-вычислительных систем» (КИБЭВС). Заведующим кафедрой КИБЭВС на сегодняшний день является ректор
ТУСУРа, Александр Александрович Шелупанов, лауреат премии Правительства Российской Федерации, действительный член 
Международной Академииnнаук высшей школы РФ, действительный член Международной Академии информации, 
Почетный работник высшего профессионального образования РФ, заместитель Председателя Сибирского регионального 
отделения учебно-методического объединения вузов России по образованию в области информационной безопасности, профессор, доктор технических наук.

С 2008 г. кафедра КИБЭВС входит в состав Института «Системной
интеграции и безопасности».

На базе кафедры КИБЭВС ТУСУР в 2002 году организовано «Сибирское региональное отделение учебно-методического объединения
Вузов России по образованию в области информационной безопасности [1].

Официальный сайт КИБЭВС [Электронный ресурс]. – Режим
доступа: http://kibevs.tusur.ru/pages/kafedra/index (дата обращения:


\newpage 
\section{Обзор информационных источников }
На первом этапе научно-исследовательской работы необходимо было провести аналитический обзор информационных источников, рассмотреть существующие методы определения авторства исходного кода и различные подходы к решению такого рода задачи. 

В работе~\cite{frantz} представлен набор инструментов и техник, используемых для решения задач анализа авторства исходного кода, а также обзор некоторых наработок в данной предметной области. Кроме того, авторы приводят собственную классификацию проблем и подходов к их решению в рамках задачи деанонимизации авторов программного обеспечения. 

Среди проблем (задач) анализа авторства исходного кода выделены:
\begin{itemize}
  \item идентификация автора --- направлена на определение, принадлежит ли определенный фрагмент кода конкретному автору;
  \item характеристика автора --- базируется на анализе стиля программирования; 
  \item определение плагиата --- нахождение схожестей среди множества фрагментов файлов исходного кода;
  \item определение намерений автора --- был ли код изначально вредоносным или стал таковым в следствие программной ошибки;
  \item дискриминация авторов --- определение, был ли код написан одним автором или несколькими.
\end{itemize}

Подходы к решению вышеперечисленных проблем (задач):
\begin{itemize}
  \item анализ <<вручную>> --- данный подход включает в себя исследование и анализ фрагмента исходного кода экспертом; 
  \item вычисление схожести --- базируется на измерении и сравнении различных метрик или токенов для набора файлов исходного кода;
  \item статистический анализ --- в таком подходе используются статистические техники, такие как дискриминантный анализ и стилометрия, позволяющие определить различия между авторами;
  \item машинное обучение --- используются методы рассуждения на основе прецедентов и нейронные сети для классификации автора на базе некоторого набора метрик.
\end{itemize}

В работе~\cite{maevsky} предложен способ определения авторства программного обеспечения. в основе которого лежит система, состоящая из 100 метрик, отражающих <<почерк создателя>> программного обеспечения. На основе метрик составлен <<профиль почерка>> пяти разных программистов по текстам трех разработанных ими программных систем и проверено соответствие этому профилю других программ, написанных в том числе и другими программистами. Однако авторы привели сомнительные результаты вычислений, не указали способ составления <<профиля почерка>> и полученную точность, с которой программная система определяла авторство.

В~\cite{pellin} исходный код транслировался в абстрактные синтаксические деревья, после чего разбивался на функции. Дерево каждой функции принималось за отдельный документ с известным автором. Выборка, состоящая из такого рода деревьев подавалась на вход SVM-классификатору, оперирующему данными типа <<дерево>>. Классификатор обучался на файлах исходного кода двух авторов, в результате чего удалось достичь точности около 67-88\%.

В статье~\cite{burrows} рассматривался способ атрибуции исходного кода с использованием метода N-грамм. Вопрос определения авторства программ в данной работе рассматривался с точки зрения определения плагиата. В качестве выборки использовался набор из 1640 файлов исходного кода, написанных 100 авторами. Производилось ранжирование документов по схожести, после чего производилась оценка результатов. При этом составителям удалось успешно определить плагиат в 67\% случаев.

В~\cite{caliskan} применялся алгоритм классификации Random Forest~\cite{random_forest} и построение абстрактных синтаксических деревьев. Обучение и тестирование производилось для количества авторов от 250 до 1600. При этом удалось добиться высокой точности --- 94-98\%. Кроме того, авторы статьи выяснили в ходе работы, что сложнее определить авторов более простых примеров, нежели сложных программ, а также значительно выделяются авторы с большим опытом программирования на C/C++.

По результатам анализа вышеперечисленных источников было принято решение использовать подход, основанный на построении абстрактных синтаксических деревьев и классификации при помощи алгоритма Random Forest.





\newpage 
\section{Выбор набора признаков, характеризующих автора программы}\label{features}
Признаки, использованные в ходе работы для идентификации автора программы, можно разделить на три группы: лексические, синтаксические и отдельно --- частоты ключевых слов языка С++.

\subsection{Лексические признаки}

Лексические признаки могут быть вычислены при непосредственном анализе исходного кода программы (в виде текстового файла). В таблице~\ref{tab:1} приведено описание использованных при классификации авторов признаков.

\begin{table}[h!]
\caption{ Лексические признаки }
\label{tab:1}
\begin{center}
\begin{tabularx}{\linewidth}{|X|X|X|}
\hline
\multicolumn{3}{|c|}{Лексические признаки} \\
\hline
\multicolumn{1}{|c|}{Признак} & \multicolumn{1}{|c|}{Обозначение} & \multicolumn{1}{|c|}{Определение} \\
\hline
Число комментариев & ln\_comments & Натуральный логарифм отношения числа комментариев к длине файла в символах \\
\hline
Число макросов & ln\_macros & Натуральный логарифм отношения числа макросов к длине файла в символах \\
\hline
Число пробелов & ln\_spaces & Натуральный логарифм отношения числа пробелов к длине файла в символах \\
\hline
Число символов табуляции & ln\_tabs & Натуральный логарифм отношения числа символов табуляции к длине файла в символах \\
\hline
Число переводов строки & ln\_newlines & Натуральный логарифм отношения числа переводов строки к длине файла в символах \\
\hline
Коэффициент пробельных символов & whitespace\_ratio & Натуральный логарифм отношения суммы всех пробельных символов (пробелов, символов табуляции, переводов строки) к длине файла в символах \\
\hline
Число строк кода & lines\_of\_code & Число строк кода, не включающее пустые строки\\
\hline
\end{tabularx}
\end{center}
\end{table}

\subsection{Синтаксические признаки}

К синтаксическим признакам относятся характеристики кода, которые могут быть вычислены только с использованием синтаксического анализатора --- программы, позволяющей анализировать конструкции языка С/С++ и выполнять построение абстрактных синтаксических деревьев.

Синтаксические признаки (табл.~\ref{tab:2}) позволяют определить предпочтения программиста при назначении идентификаторов --- последовательности символов, используемой для обозначения имен объектов, переменных, классов, функций и т.д.~\cite{identif}

\begin{table}[h!]
\caption{ Синтаксические признаки }
\label{tab:2}
\begin{center}
\begin{tabularx}{\linewidth}{|X|X|X|}
\hline
\multicolumn{3}{|c|}{Синтаксические признаки} \\
\hline
\multicolumn{1}{|c|}{Признак} & \multicolumn{1}{|c|}{Обозначение} & \multicolumn{1}{|c|}{Определение} \\
\hline
Число функций & ln\_number\_of\_functions & Натуральный логарифм отношения числа функций к длине файла в символах \\
\hline
Длина имени функции &  avg\_funcname\_len & Средняя длина имени функции \\
\hline
Длина имени переменной &  avg\_varname\_len & Средняя длина имени переменной \\
\hline
Специальные символы &  has\_specialcharnames & Факт наличия специальных символов в идентификаторах \\
\hline
Заглавные буквы &  has\_uppercasenames & Факт наличия заглавных букв в идентификаторах \\
\hline
\end{tabularx}
\end{center}
\end{table}

\subsection{Ключевые слова С++}\label{keycpp}

Ключевые слова С++ представляют собой список зарезервированных последовательностей символов, используемых языком, недоступных для переопределения.

Для ключевых слов языка С++ вычислялась статистическая мера TF (term frequency), отображающая число вхождения некоторого слова к общему количеству слов в документе. Словарь из 84 ключевых слов С++ (стандарт 11) был взят на сайте с официальной документацией~\cite{cppkeywords} и представлен в таблице~\ref{tab:3}.

\begin{table}[ht]
\caption{ Ключевые слова языка С++ (стандарт 11) }
\label{tab:3}
\begin{center}
\begin{tabularx}{\linewidth}{|X|X|X|X|X|X|}
\hline
\multicolumn{6}{|c|}{Ключевые слова языка С++} \\
\hline
alignas & char32\_t & enum & namespace & return & try\\
alignof & class & explicit & new & short & typedef\\
and & compl & export & noexcept & signed & typeid\\
and\_eq & const & extern & not & sizeof & typename\\
asm & constexpr & false & not\_eq & static & union\\
auto & const\_cast & float & nullptr & static\_assert & unsigned\\
bitand & continue & for & operator & static\_cast & using\\
bitor & decltype & friend & or & struct & virtual\\
bool & default & goto & or\_eq & switch & void\\
break & delete & if & private & template & volatile\\
case & do & inline & protected & this & wchar\_t\\
catch & double & int & public & thread\_local & while\\
char & dynamic\_cast & long & register & throw & xor\\
char16\_t & else & mutable & reinterpret\_cast & true & xor\_eq\\
\hline
\end{tabularx}
\end{center}
\end{table}
 
 
\newpage  
\section{Моделирование}\label{modeling}
Описание процесса определения авторства исходного кода программ в виде модели <<черного ящика>> согласно методологии IDEF0 представлено на рисунке~\ref{box_1:box_1}, его декомпозиция --- на рисунке~\ref{box_2:box_2}.

\begin{figure}[h!]
\center{\includegraphics[width=0.6\linewidth]{box_1}}
\caption{ Модель <<черного ящика>> процесса определения авторства исходного кода по методологии IDEF0 }
\label{box_1:box_1}
\end{figure} 

\begin{figure}[h!]
\center{\includegraphics[width=1\linewidth]{box_2}}
\caption{ Декомпозиция <<черного ящика>> процесса определения авторства исходного кода по методологии IDEF0 }
\label{box_2:box_2}
\end{figure} 

 
% \newpage
% \section{Тестирование модели}
% \input{}
 
\newpage  
\section{Классификация}\label{classifiers}
В данной работе в качестве базового алгоритма для всех классификаторов (см. разделы~\ref{random_forest},
\ref{ada}, \ref{extra}) были выбраны деревья решений 
(Decision Trees), тестирование и оценка модели производилась на основе 10-фолдовой кросс-валидации 
(см. раздел~\ref{crossval}).

Деревья решений (Decision Trees)~\cite{data_mining} или деревья принятия решений являются одним из наиболее популярных
методов решения задач классификации, регрессии и прогнозирования. Впервые деревья решений были предложены Ховилендом и Хантом (Hoveland, Hunt) 
в конце 50-х годов прошлого века и в наиболее простом виде представляют собой совокупность правил в иерархической 
структуре. Основа такой структуры --- это ветвление при проверке условий (<<Да>> --- <<Нет>>). 



\subsection{Алгоритм классификации Random Forest}\label{random_forest}
Алгоритм классификации Random Forest Classifier~\cite{random_forest} строится на двух базовых принципах: 
\begin{itemize}
  \item bagging --- мета-алгоритм в машинном обучении, при котором на основе большого числа <<слабых>> классификаторов (в данном случае деревьев решений) строится один <<сильный>> классификатор (рис.~\ref{forest:forest});
  \item метод случайных подпространств.
\end{itemize}

Преимущества данного алгоритма классификации:
\begin{itemize}
  \item способность эффективно обрабатывать данные с большим числом признаков и классов;
  \item нечувствительность к масштабированию (к любым монотонным преобразованиям) значений признаков;
  \item существует методы оценивания значимости отдельных признаков в модели;
  \item внутренняя оценка способности модели к обобщению (тест out-of-bag);
  \item высокая параллелизуемость и масштабируемость.
\end{itemize}

Недостатки алгоритма Random Forest Classifier:
\begin{itemize}
  \item алгоритм склонен к переобучению на некоторых задачах, особенно на зашумленных, однако для избежания переобучения используется энтропия Шеннона или коэффициент прироста информации (англ. Gain);
  \item большой размер получаемых моделей приводит к существенным затратам памяти на хранение деревьев, однако данный недостаток решается повышением вычислительных мощностей и распараллеливанием вычислений.
\end{itemize}

\begin{figure}[h!]
\center{\includegraphics[width=0.5\linewidth]{forest}}
\caption{ Random Forest Classifier }
\label{forest:forest}
\end{figure} 

 

\subsection{Алгоритм классификации AdaBoost}\label{ada}
Алгоритм AdaBoost (сокр. от adaptive boosting)~\cite{ada_boost} является мета-алгоритмом, в процессе обучения 
строит композицию из базовых алгоритмов обучения для улучшения их эффективности.

Достоинства:
\begin{itemize}
 \item хорошая обобщающая способность --- в реальных задачах (не всегда, но часто) удаётся строить композиции, 
превосходящие по качеству базовые алгоритмы, при этом обобщающая способность может улучшаться (в некоторых задачах) 
по мере увеличения числа базовых алгоритмов;
 \item простота реализации;
 \item время построения композиции практически полностью определяется временем обучения базовых алгоритмов.
\end{itemize}


Недостатки алгоритма классификаций AdaBoost:
\begin{itemize}
 \item склонен к переобучению при наличии значительного уровня шума в данных;
 \item требует достаточно длинных обучающих выборок;
 \item бустинг может приводить к построению громоздких композиций, состоящих из сотен алгоритмов, 
 такие композиции исключают возможность содержательной интерпретации, требуют больших объёмов памяти 
 для хранения базовых алгоритмов и существенных затрат времени на вычисление классификаций.
\end{itemize}

\subsection{Алгоритм классификации ExtraTrees}\label{extra}
Алгоритм ExtraTrees (Extremly Randomized Trees)~\cite{extra_trees} является модификацией 
алгоритма Random Forest Classifier (см. раздел~\ref{random_forest}), но отличается еще более рандомизированным
разделением входного набора данных на подвыборки. Как правило, результаты работы данного алгоритма
схожи с результатами Random Forest Classifier, однако в определенных случаях могут давать улучшение
точности классификации.





\newpage
\section{Конструкторско-технологическая часть}
\subsection{Среда разработки и язык программирования}
\newpage
\subsection{Описание программного обеспечения <<WhoseCppCode>>}
Программа <<WhoseCppCode>>, разработанная в ходе работы, состоит из двух основных частей:
\begin{itemize}
 \item программного модуля, реализующего все необходимые функции для сбора, анализа и обработки данных, 
а также построения модели классификации авторов исходного кода, описанной в разделе~\ref{modeling};
 \item программного интерфеса на основе технологии Jupyter Notebook~\cite{jupyter}, предназначенного 
 для визуализации полученных в ходе классификации результатов, сбора необходимых данных с ресурса GitHub~\cite{GitHub},
 построения матрицы объектов-признаков на основе входных данных, проведения вычислительных экспериментов.
\end{itemize}

Программный модуль реализован на языке прораммирования высокого уровня Python с использованием следующих 
программных библиотек:
\begin{itemize}
 \item Scikit-Learn~\cite{scikit} --- open-source библиотека для машинного обучения: классификации, регрессии, кластеризации и т.д.
 \item Plotly~\cite{plotly} --- графическая Python-библиотека для построения интерактивных графиков, таблиц, диаграмм.
 \item Numpy~\cite{numpy} --- библиотека для научных вычислений, предоставляющая методы работы с большими массивами данных.
 \item Scipy~\cite{scipy} --- предоставляет среду для проведения математических и научных вычислений.
 \item Pandas~\cite{pandas} --- open-source библиотека, предназначенная для анализа данных.
 \item Ipywidgets~\cite{widgets} --- интерактивные HTML виджеты для Jupyter Notebook.
\end{itemize}

Интерфейс основан на веб-технологиях, может использоваться для демонстрации возможностей
программ на языке Python. Библиотека Jupyter Notebook, с помощью которой был реализован
данный интерфейс, была выбрана за счет ряда преимуществ:
\begin{itemize}
 \item является свободным ПО;
 \item поддерживает множество языков программирования;
 \item позволяет хранить вместе код, изображения, комментарии, формулы и графики;
 \item не требует знаний и применения веб-технологий, таких как CSS, HTML, JavaScript;
 \item может быть запущен на любом сервере, необходим только доступ по ssh/http;
 \item позволяет экспортировать код и сам блокнот в любом формате;
 \item предназначена для демонстрации разработок на языке Python (в основном в машинном обучении).
\end{itemize}

Основной модуль программы <<WhoseCppCode>> может быть использован отдельно от Jupyter Notebook 
при разработке различного рода программ, систем и интерфейсов лицами, заинтересованными в задаче
классификации программистов.

Диаграмма действий в нотации UML, описывющая основной алгоритм работы программы <<WhoseCppCode>>
представлена на рисунке~\ref{flowchart:flowchart}, примеры ввода и вывода данных в интерфейсе Jupyter 
Notebook --- на рисунках~\ref{main_module:main_module}, \ref{newplot:newplot} и~\ref{newplot2:newplot2}.

\begin{figure}[h!]
\center{\includegraphics[width=0.9\linewidth]{flowchart}}
\caption{ Диаграмма действий алгоритма работы программы <<WhoseCppCode>> }
\label{flowchart:flowchart}
\end{figure}


\begin{figure}[h!]
\center{\includegraphics[width=0.7\linewidth]{main_module}}
\caption{ Вид основной формы программного интерфейса }
\label{main_module:main_module}
\end{figure}

\begin{figure}[h!]
\center{\includegraphics[width=0.7\linewidth]{newplot}}
\caption{ Вывод диаграммы результатов классификации }
\label{newplot:newplot}
\end{figure}

\begin{figure}[h!]
\center{\includegraphics[width=0.7\linewidth]{newplot2}}
\caption{ Пример вывода диаграммы для средней точности классификации }
\label{newplot2:newplot2}
\end{figure}

% Описание функций, реализованных в основном модуле программы, а также
% примеры ввода и вывода данных в интерфейсе Jupyter Notebook представлены в приложениях Г и Д соответственно. 


\subsection{Программные модули}
\subsection{Интерфейс разрабатываемого программного обеспечения}





\clearpage
\section{Программа и методика испытаний}
Раздел <<Программа и методика испытаний>> был составлен и оформлен в соответствии с ГОСТ 19.301--79.~\cite{gost_19301} 

\subsection{Объект испытаний}
\subsubsection{Полное наименование системы и ее условное обозначение}

Условное обозначение: <<WhoseCppCode>>.

\subsubsection{Область применения}

Разработанный программный комплекс можно использовать


\subsection{Цель испытаний}

Испытания системы предназначены для оценки адекватности модели, ее точности

\subsection{Требования к программе}

\subsection{Требования к программной документации}

Пояснительная записка к дипломной работе должна включать в себя:

\begin{itemize}
 \item задание по дипломному проектированию;
 \item руководство администратора (приложение~Г);
 \item руководство программиста (приложение~Д);
 \item руководство пользователя (приложение~Е):
 \item результаты вычислительных экспериментов.
\end{itemize}

\textbf{руководство пользователя должно быть оформлено 
Согласно ГОСТ 19.505-79 ЕСПД. Руководство оператора. Требования к содержанию и оформлению
http://techwrconsult.com/library/19.505

программиста - ГОСТ 19.504-79 РУКОВОДСТВО ПРОГРАММИСТА. 
ТРЕБОВАНИЯ К СОДЕРЖАНИЮ И ОФОРМЛЕНИЮ
http://www.rugost.com/index.php?option=com_content&view=article&id=64:19504-79&catid=19&Itemid=50}



\subsection{Средства и порядок испытаний}
\subsubsection{Технические и программные средства, используемые во время испытаний}



\subsubsection{Порядок проведения испытаний}


\subsection{Методы испытаний}\label{testing_methods}

Для тестирования аналитической модели в машинном обучении применяется процедура скользящего
контроля, получившая название кросс-валидации (cross-validation) или перекрестной проверки.~\cite{crossval}

Процедура кросс-валидации включает в себя случайное разбиение на k подгрупп (или фолдов) примерно
одинакового размера. Первый фолд служит для тестирования модели, остальные используются для обучения
классификатора. Для тестовой подвыборки вычисляется среднеквадратичное отклонение. Процедура повторяется k-1
раз, при этом каждая из подгрупп выступает в роли тестовой выборки.

В данной работе тестирование производилось с применением 10-фолдовой кросс-валидации. Всего было произведено
10 вычислительных экспериментов.


\newpage 
\subsection{Описание тестового набора данных}\label{test_data}
Burrows в работе~\cite{burrows_big} выделяет следующие ключевые параметры тестовых данных,
которые могут влиять на точность классификации:

\begin{itemize}
 \item Число авторов --- с увеличением числа авторов сложность классификации увеличивается, 
 точность --- снижается.
 \item Число экземпляров выборки для каждого автора --- желательно соблюдать одинаковым для всех авторов во 
 избежание отклонения в сторону наиболее точно описанных авторов, а также иметь больше экземпляров 
 для увеличения размера тестовой выборки. 
 \item Средняя длина образца кода (количество непустых строк кода) --- чем длиннее, тем выше точность
 классификации. Изменение длины экземпляров выборки может влиять на отклонение в сторону наиболее 
 точно описанных авторов, однако не представляется возможным соблюдать длину экземпляра выборки постоянной.
 \item <<Стилистическая зрелость>> (stylistic maturity) авторов --- уровень квалификации, 
 личные и профессиональные предпочтения в стиле написания программ.
 \item Временные метки образцов кода (подразумевается, что со временем программы устаревают, технологии
 и методы программирования меняются и, как следствие, изменяется стиль программирования).
 \item Репрезентативность выборки --- демографические, социальные и другие факторы.
 \item Типы авторов --- студент, фрилансер, профессиональный разработчик. 
 В идеале система должна включать в себя разные типы.
 \item Языки программирования --- если тестировать несколько языков одновременно, 
 результат будет зависеть от характерных признаков языка.
 \item Авторство в одном лице --- большинство проектов выполняются в 
 сотрудничестве с другими разработчиками.
 \item Корректное авторство --- без плагиата, копирования и т.п.
\end{itemize}

Burrows упоминает также от том, что характерный стиль программирования нестабилен в начале карьеры программиста,
что может существенно отличать начинающего специалиста и профессионала разработки.

Программное обеспечение <<WhoseCppCode>> тестировалось на трех наборах данных:

\begin{enumerate}
 \item <<Students>> --- выборка представляет собой работы студентов первого курса обучения по 
 предмету <<Основы программирования>>. Все программы реализуют решения однотипных задач в рамках учебной 
 дисциплины, что исключет их разделение при классификации по функциональному назначению вместо 
 стилистических особенностей и снижение точности классификации. 
 \item <<Google Code Jam>> --- общедоступные данные ежегодной международной олимпиады по программированию Google Code Jam 
 2016.~\cite{GoogleCodeJam} Так же, как и в первой выборке, авторы решали схожие задачи, используя различные
 подходы и алгоритмы. 
\item <<GitHub>> --- данные, собранные с сайта GitHub~\cite{GitHub} --- крупнейшего~\cite{GH_domain} 
веб-сервиса для хостинга IT-проектов и их совместной разработки. 
\end{enumerate}


Сбор данных с веб-хостинга GitHub производился по следующему принципу: 
\begin{enumerate}
 \item Выбирались крупные open-source репозитории (удаленные хранилища программного кода и данных),  
 посвященные разработке проектов на C/C++.
 \item Просматривался список контрибьюторов (пользователей, вносивших изменения в проект).
 \item В качестве авторов выбирались те контрибьюторы, у которых имеются личные проекты, написанные 
 на C/C++.
 \item На основе списка пользователей автоматически, средствами программы <<WhoseCppCode>>, производился
 сбор и сохрание файлов исходного кода для каждого автора. 
\end{enumerate}

В таблице~\ref{tab:data} приводится описание некоторых характеристик каждого набора данных.
В данном случае под смешанным типом авторов подразумевается, что разработчики могли быть совершенно
разного уровня квалификации и рода деятельности (студенты, фрилансеры, начинающие и 
профессиональные разработчики, программисты-любители и т.д.).

\begin{table}[h!]
\caption{ Тестовые данные }
\label{tab:data}
\begin{center}
\begin{tabularx}{\linewidth}{|X|X|X|X|X|X|}
\hline
Набор данных & <<Students>> & <<Google Code Jam>> & <<GitHub>> \\
\hline
Число авторов & 3 & 30 & 30 \\
\hline
исло файлов исходного кода на одного автора & 14 & 9 & 78 \\
\hline
Всего файлов исходного кода & 42 & 278 & 2334 \\
\hline
Минимальное число строк кода & 33 & 36 & 26 \\
\hline
Максимальное число строк кода & 160 & 461 & 16348 \\
\hline
Среднее число строк кода на один файл исходного кода & 45 & 87 & 234 \\
\hline
Тип авторов & Студенты & Смешанный & Смешанный \\
\hline
\end{tabularx}
\end{center}
\end{table}


Каждая выборка представляет собой совокупность файлов исходного кода программ на языке C/C++ с расширениями
*.cpp, *.c, *.h, *.hpp, *.cxx, *.cc, *.ii, *.ixx, *.ipp, *.inl, *.txx, *.tpp, *.tpl.

Тестовая и обучающая выборки генерировались случайным образом без повторений из начальной выборки, 
описанной выше. При этом четверть всех примеров использовалась для тестирования (классификации), а также
использовалась процедура скользящего контроля --- \textbf{10-фолдовая кросс-валидация}. Подробное описание процедуры рзбиения данных
и тестирования \textbf{ обучаемой модели содержится в разделах}




\clearpage
\subsection{Критерии оценки эффективности классификации}\label{eval}
Критерии оценки работы классификатора~\cite{metrics} представлены в таблице~\ref{tab:eval}, где:
\begin{itemize}
  \item tp --- истинно-положительное решение;
  \item tn --- истинно-отрицательное решение;
  \item fp --- ложно-положительное решение;
  \item fn --- ложно-отрицательное решение;
  \item Accuracy (точность) --- отношение количества документов, по которым классификатор принял правильное решение, к общему числу документов (примеров файлов исходного кода); 
  \item Precision (правильность) --- доля документов, действительно принадлежащих данному классу, относительно всех документов, которые система отнесла к этому классу;
  \item Recall (полнота) --- доля найденных классфикатором документов, принадлежащих классу, относительно всех документов этого класса в тестовой выборке;
  \item F1-score (F1-мера) --- гармоническое среднее между правильностью и полнотой.
\end{itemize}

\begin{table}[h!]
\caption{ Критерии оценки работы классификатора }
\label{tab:eval}
\begin{center}
\begin{tabularx}{\linewidth}{|c|c|X|X|}
\hline
Критерий & Формула & Луч. знач. & Худ. знач. \\
\hline
Accuracy (точность) & (tp + tn) / число примеров * 100 \% & 100 \% & 0 \% \\
\hline
Precision (правильность) & tp / (tp + fp) & 1 & 0 \\
\hline
Recall (полнота) & tp / (tp + fn) & 1 & 0 \\
\hline
F1-score (F1-мера) & 2 * (precision * recall) / (precision + recall) & 1 & 0 \\
\hline
\end{tabularx}
\end{center}
\end{table}

\newpage
\subsection{Результаты классификации}
При тестировании классификатора использовались критерии оценки, описанные в разделе~\ref{eval}, 
а также время работы программы. Результаты работы классификатора представлены в таблице~\ref{tab:results}, а также
на рисунках~\ref{students_res:students_res}, \ref{google:google} и~\ref{github:github}.

\begin{table}[h!]
\caption{ Результаты работы }
\label{tab:results}
\begin{center}
\begin{tabularx}{\linewidth}{|X|X|X|X|X|X|}
\hline
\multicolumn{6}{|c|}{Набор данных <<Students>>} \\
\hline
Классифика- тор & Accuracy, \% & Precision & Recall & F1-score & Время работы, сек.\\
\hline
RandomForest & 89,55 & 0,90 & 0,93 & 0,90 & 93,61\\
\hline
AdaBoost & 70,45 & 0,70 & 0,74 & 0,70 & 53,22\\
\hline
ExtraTrees & 91,85 & 0,92 & 0,95 & 0,92 & 53,70 \\
\hline
\multicolumn{6}{|c|}{Набор данных <<Google Code Jam>>} \\
\hline
Классифика- тор & Accuracy, \% & Precision & Recall & F1-score & Время работы, сек.\\
\hline
RandomForest & 86,66 & 0,86 & 0,88 & 0,87 & 110,65\\
\hline
AdaBoost & 19,43 & 0,16 & 0,16 & 0,19 & 103,53\\
\hline
ExtraTrees & 88,09 & 0,88 & 0,90 & 0,88 & 60,34 \\
\hline
\multicolumn{6}{|c|}{Набор данных <<GitHub>>} \\
\hline
Классифика- тор & Accuracy, \% & Precision & Recall & F1-score & Время работы, сек.\\
\hline
RandomForest & 69,92 & 0,69 & 0,71 & 0,70 & 223,77\\
\hline
AdaBoost & 16,44 & 0,09 & 0,11 & 0,16 & 451,63\\
\hline
ExtraTrees & 70,99 & 0,70 & 0,72 & 0,71 & 201,13 \\
\hline
\end{tabularx}
\end{center}
\end{table}

\begin{figure}[h!]
\center{\includegraphics[width=0.6\linewidth]{students}}
\caption{ <<Students>> }
\label{students_res:students_res}
\end{figure}

\begin{figure}[h!]
\center{\includegraphics[width=0.6\linewidth]{google}}
\caption{ <<Google Code Jam>> }
\label{google:google}
\end{figure}

\begin{figure}[h!]
\center{\includegraphics[width=0.6\linewidth]{github}}
\caption{ <<GitHub>> }
\label{github:github}
\end{figure}

Наихудшие результаты показал алгоритм AdaBoost, в то время как наиболее точным и быстрым из трех 
представленных алгоритмов оказался ExtraTrees (см. раздел~\ref{extra}). 

С использованием метода, основанного на извлечении лексических признаков и классификации с помощью алгоритма 
ExtraTrees (Extremly Randomized Trees) точность классификации составила 70-71\% на выборке данных из 30 авторов
и 2334 неполных, немопилируемых файлов с веб-хостинга GitHub. 

По результатам классификации можно сделать следующие выводы:
\begin{enumerate}
 \item Заменив алгоритм классификации RandomForest
 на его модифицированную версию, ExtraTrees (Extremly Randomized Trees),
 можно повысить точность классификации, достигнутую в работе~\cite{git_blame},
 что в итоге, при воссоздании эксперимента коллег, даст наилучший результат классификации
 авторов исходного кода на языке С/С++ на сегодняшний день.
 \item Исследованные методы могут применяться не только в <<лабораторных>> условиях, когда
тестовая выборка генерируется на основе студенческих работ или результатов олимпиад по программированию,
где решаются схожие задачи, ограниченные по времени и объему кода, но и в условиях реального мира.
\end{enumerate}


\clearpage


\newpage
\section{Безопасность жизнедеятельности}
\subsection{Анализ опасных и вредных производственных факторов на рабочем месте}

В ходе трудового процесса организм человека может подвергаться различным воздействиям, 
оказывающим влияние на его здоровье и работоспособность. Подобное воздействие может приводить к 
различным результирующим последствиям, которые зависят от характера воздействия (прямого или
опосредованного), наличия тех или иных факторов производственной среды, а также условий
их проявления. 

Принято разграничивать производственные факторы на две основные группы --- опасные производственные факторы
(ОПФ) и вредные производственные факторы (ВПФ). При этом однозначного отнести тот или иной
фактор к подмножеству опасных или вредных не всегда представляется возможным, поскольку
даже нейтральные производственные факторы пр наличии определенных условий и обстоятельств
могут становиться вредными или опасными для человека, приводить к травмам и заболеваниям, 
связанным с трудовой деятельностью.

Согласно~\cite{gost_12.0.003-2015} опасные и вредные производственные факторы производственной
среды по природе их воздействия на организм работающего человека подразделяются на:
\begin{itemize}
 \item факторы, воздействие которых носит физическую природу;
 \item факторы, воздействие которых носит химическую природу;
 \item факторы, воздействие которых носит биологическую природу.
\end{itemize}

При работе за ПЭВМ воздействие на организм человека носит физическую природу. Кроме того
работники подвергаются нервно-психическим перегрузкам.

Основываясь на классификации вредных и опасных факторов производства из ГОСТ 12.0.003--2015~\cite{gost_12.0.003-2015} 
можно выделить следующие физические факторы, связанные с работой за ПЭВМ:
\begin{itemize}
 \item повышенный уровень и другие неблагоприятные факторы шума;
 \item повышенная или пониженная температура воздуха рабочей зоны;
 \item повышенная или пониженная влажность воздуха;
 \item повышенная или пониженная подвижность воздуха;
 \item повышенное значение напряжения в электрической цепи;
 \item повышенный уровень статического электричества;
 \item повышенный уровень электромагнитных излучений;
 \item отсутствие или недостаток естественного освещения;
 \item отсутствие или недостаток искусственного освещения;
 \item повышенная яркость, пульсация света. 
\end{itemize}

К нервно-психическим перегрузкам относят~\cite{gost_12.0.003-2015}:
\begin{itemize}
 \item умственное перенапряжение, в том числе вызванное информационной перегрузкой;
 \item перенапряжение анализаторов, в том числе вызванное информационной перегрузкой;
 \item монотонность труда, вызывающая монотонию;
 \item эмоциональные перегрузки.
\end{itemize}

Под информационной перегрузкой понимается воспринимаемая сенсорными системами организма
человека интенсивность поступления информации, воздействующая на центральную нервную систему человека
и способная приводить к различным неблагоприятным последствиям для здоровья.

\subsubsection{Перечень продукции и контролируемые гигиенические параметры вредных и опасных факторов}

Перечень продукции и контролируемых гигиенических параметров огласно~\cite{sanpin_2.4.1340-03} 
приведен в таблице~\ref{tab:life_1}.

\begin{table}[h!]
\caption{ Перечень продукции, контролируемых гигиенических параметров }
\label{tab:life_1}
\begin{center}
\begin{tabularx}{\linewidth}{|X|X|}
\hline
Вид продукции & Контролируемые гигиенические параметры\\
\hline
\item Машина вычислительная электронная цифровая персональная (ПЭВМ) & 
\begin{itemize}
 \item уровни электромагнитных полей (ЭМП);
 \item уровни акустического шума;
 \item концентрация вредных веществ в воздухе;
 \item визуальные показатели видеодисплейного терминала
\end{itemize}									     
\\
\hline
\item Устройства периферийные: модем, клавиатура, принтер, устройства хранения информации & 
\begin{itemize}
 \item уровни ЭМП;
 \item уровни акустического шума;
 \item концентрация вредных веществ в воздухе
\end{itemize}
 \\
\hline
\end{tabularx}
\end{center}
\end{table}




\subsubsection{Требования к уровням шума на рабочих местах, оборудованных ПЭВМ}

Допустимые уровни звукового давления и уровней звука, создаваемых ПЭВМ, не должны превышать значений, 
представленных в таблице~\ref{tab:life_2}. Измерение уровня звука и уровней звукового давления 
проводится на расстоянии 50 см от поверхности оборудования и на высоте расположения 
источника(ков) звука. Шумящее оборудование (печатающие устройства, серверы и т.п.), уровни шума которого превышают нормативные, должно 
размещаться вне помещений с ПЭВМ.~\cite{sanpin_2.4.1340-03}

\begin{table}[h!]
\caption{ Допустимые значения уровней звукового давления }
\label{tab:life_2}
\begin{center}
\begin{tabularx}{\linewidth}{|X|X|X|X|X|X|X|X|X|X|}
\hline
\multicolumn{9}{|c|}{Уровни звукового давления в октавных полосах со среднегеометрическими частотами} & \multirow{2}{\hsize}{Уровни звука в дБА}\\
\cline{1-9}
31,5 Гц & 63 Гц & 125 Гц & 250 Гц & 500 Гц & 1000 Гц & 2000 Гц & 4000 Гц & 8000 Гц & \\
\hline
31,5 Гц & 63 Гц & 125 Гц & 250 Гц & 500 Гц & 1000 Гц & 2000 Гц & 4000 Гц & 8000 Гц & 50\\
\hline
\end{tabularx}
\end{center}
\end{table}




\subsubsection{Требования к уровням электромагнитных полей на рабочих местах, оборудованных ПЭВМ}

Временные допустимые уровни ЭМП, создаваемых ПЭВМ на рабочих местах пользователей
представлены в таблице~\ref{tab:life_3}.~\cite{sanpin_2.4.1340-03}

\begin{table}[h!]
\caption{ Временные допустимые уровни ЭМП, создаваемых ПЭВМ на рабочих местах }
\label{tab:life_3}
\begin{center}
\begin{tabularx}{\linewidth}{|>{\hsize=0.45\hsize}X|>{\hsize=0.45\hsize}X|>{\hsize=0.1\hsize}X|}
\hline
\multicolumn{2}{|c|}{Наименование параметров} & ВДУ\\
\hline
\multirow{2}{\hsize}{Напряженность электрического поля} & в диапазоне частот 5 Гц -- 2 кГц & 25 В/м\\
\cline{2-3}
 & в диапазоне частот 2 кГц -- 400 кГц & 2,5 В/м \\
\hline
\multirow{2}{\hsize}{Плотность магнитного потока} & в диапазоне частот 5 Гц - 2 кГц & 250 нТл \\
\cline{2-3}
 & в диапазоне частот 2 кГц -- 400 кГц & 25 нТл \\
\hline
\multicolumn{2}{|c|}{Напряженность электростатического поля} & 15 кВ/м \\
\hline
\end{tabularx}
\end{center}
\end{table}


\subsubsection{Требования к визуальным параметрам устройств отображения информации}

Предельно допустимые значения визуальных параметров визуального дисплейного терминала, 
контролируемые на рабочих местах, представлены в таблице~\ref{tab:life_4}.~\cite{sanpin_2.4.1340-03}


\begin{table}[h!]
\caption{ Визуальные параметры устройств отображения информации }
\label{tab:life_4}
\begin{center}
\begin{tabularx}{\linewidth}{|>{\hsize=0.7\hsize}X|>{\hsize=0.3\hsize}X|}
\hline
Параметры & Допустимые значения\\
\hline
Яркость белого поля & Не менее 35 кд/кв.м\\
\hline
Неравномерность яркости рабочего поля & Не более ± 20\%\\
\hline
Контрасность (для монохромного режима) & Не менее 3:1\\
\hline
Временная нестабильность изображений (непреднамеренное изменение во времени яркости 
изображения на экране дисплея) & Не должна фиксироваться\\
\hline
Пространственная нестабильность изображения (непреднамеренные изменения 
положения фрагментов изображения на экране)& Не более $2\times1E(-4L)$, где $L$~--- проектное расстояние наблюдения, мм\\
\hline
\end{tabularx}
\end{center}
\end{table}

Для дисплеев на ЭЛТ частота обновления изображения должна быть не менее 75 Гц при всех режимах 
разрешения экрана, гарантируемых нормативной документацией на конкретный тип дисплея и не менее 60 Гц 
для дисплеев на плоских дискретных экранах (жидкокристаллических, плазменных и т.п.).~\cite{sanpin_2.4.1340-03}


\subsubsection{Требования к микроклимату. Концентрации вредных веществ, выделяемых ПЭВМ в воздух помещения}


 Показатели микроклимата должны обеспечивать сохранение теплового баланса человека с окружающей средой и 
 поддержание оптимального или допустимого теплового состояния организма.
 Несоблюдение оптимальных микроклиматических условий может привести к ухудшению состояния здоровья,
 снижению работоспособности, ощущению дискомфорта, напряжению механизмов терморегуляции.
 
 Работа за ПЭВМ относится к категории Iа: с интенсивностью энерготрат до 120 ккал/ч (до 139 Вт), 
 производимая сидя и сопровождающиеся незначительным физическим напряжением.~\cite{sanpin_mikroclimate} 
 
 Согласно~\cite{sanpin_mikroclimate} допустимые величины показателей микроклимата на рабочих местах производственных помещений категории Iа 
 должны соответствовать значениям, приведенным в таблице~\ref{tab:climate_1}. 
 
 Амплитуда колебания температуры воздуха в течение смены не должна превышать~4\textdegree{} C.
 
\begin{table}[h!]
\caption{ Оптимальные величины показателей микроклимата на рабочих местах производственных помещений категории Iа }
\label{tab:climate_1}
\begin{center}
\begin{tabularx}{\linewidth}{|X|X|X|X|X|}
\hline
Период года & Температура воздуха, \textdegree{}C & Температура поверхностей, \textdegree{}C & Относительная влажность воздуха & Скорость движения воздуха, м/с\\
\hline
Холодный & 22-24 & 21-25 & 60-40 & 0,1\\
\hline
Теплый & 23-25 & 22-26 & 60-40 & 0,1\\
\hline
\end{tabularx}
\end{center}
\end{table}





\subsubsection{Требования к освещенности}

\subsubsection{Требования к организации рабочих мест пользователей ПЭВМ}


\subsection{Инструкция по работе на персональном компьютере}

\subsubsection{Противопожарная безопасность}

\subsubsection{Электрическая безопасность}

\subsubsection{Требования охраны труда во время работы. Оказание первой медицинской помощи}

\newpage
\section{Технико-экономическое обоснование}
\subsection{Актуальность работы}

\subsection{Организация и планирование работы}

Основные задачи организации и планирования работ:
\begin{itemize}
 \item определение объема предстоящих работ;
 \item определение основных этапов работ;
 \item установление сроков выполнения запланированных работ;
 \item определение необходимых денежных, материальных и трудовых ресурсов.
\end{itemize}


При выполнении дипломной работы были задействованы следующие лица:
\begin{itemize}
 \item руководитель (рук.);
 \item разработчик (разр.).
\end{itemize}

Месячный оклад студента, не являющегося дипломированным специалистом, составляет 2324,40 рублей.
С учетом 24 рабочих дней и 6-часового рабочего дня стоимость одного часа работ равна 16,14 рублей.
Месячный оклад руководителя с ученой степенью кандидата наук и должностью доцента составляет 14800 рублей.
Стоимость одного часа работ с учетом 24-ех 6-часовых рабочих дней равна 102,78 рублей.   

Руководитель работы оказывает помощь разработчику в планировании работ в период проектирования, рекомендует
необходимую литературу, проводит консультации разработчика, осуществляет контроль над выполнением всех 
намеченных этапов работы. Разработчик реализует объем работ, установленный в техническом задании.

График работ приведен в таблице~\ref{tab:job_is_done_1}.

\begin{table}[!ht]
\caption{График выполнения работ}
\centering
\includegraphics[page=1, width=1\linewidth]{schedule.pdf}
\label{tab:job_is_done_1}
\end{table}


\begin{table}[!ht]
\caption{Ленточный график загрузки участников работ}
\centering
\includegraphics[page=1, width=1\linewidth]{schedule_2.pdf}
\label{tab:job_is_done_2}
\end{table}

\begin{table}[!ht]
\caption{Календарный график загрузки участников}
\centering
\includegraphics[page=1, width=1\linewidth]{schedule_3.pdf}
\label{tab:job_is_done_3}
\end{table}

\begin{table}[!ht]
\caption{Смета затрат на оборудование}
\centering
\includegraphics[page=1, width=1\linewidth]{schedule_4.pdf}
\label{tab:job_is_done_4}
\end{table}

\begin{table}[!ht]
\caption{Расчет расходов на оплату труда участников проекта}
\centering
\includegraphics[page=1, width=1\linewidth]{schedule_5.pdf}
\label{tab:job_is_done_5}
\end{table}

\clearpage




% \newpage  
% \section{Руководство администратора}
% \input{admin}
% 
% \newpage  
% \section{Руководство программиста}
% \newpage
\setcounter{section}{0}
\section*{Приложение Д}
\section*{(Справочное)}
\section*{Руководство программиста}

\hfill

\begin{center}
 ПРОГРАММНОЕ ОБЕСПЕЧЕНИЕ ДЛЯ\\
 ОПРЕДЕЛЕНИЯ АВТОРСТВА ИСХОДНОГО КОДА ПРОГРАММ НА ЯЗЫКЕ С/С++\\
 <<WhoseCppCode>>\\
 
 Руководство программиста\\
 
 Листов 7
\end{center}


\hfill

\newpage
\section*{Аннотация}

\section{Назначение и условия применения программы}
\subsection{Назначение}

Программное обеспечение <<WhoseCppCode>> предназначено для определения авторства программ на 
языке С/С++ по исходному коду.

\subsection{Функции}

\subsection{Условия, необходимые для выполнения программы}

\section{Характеристики программного обеспечения}

\subsection{Режим работы}

\subsection{Описание особенностей программного обеспечения}

\subsection{Обеспечение надежности программного обеспечения}

\section{Обращение к программе}

\section{Алгоритм}


\section{Входные данные}

\section{Выходные данные}

\section{Сообщения}





% 
% \newpage  
% \section{Руководство пользователя}
% \begin{center}
 Приложение Е
 
 (Справочное)
 
 Руководство пользователя
\end{center}




\vspace*{7cm}
\begin{center}%
\LARGE {Lucas Kanade}\\\large{1.4}\\
\vspace*{1cm}
{\large Создано системой Doxygen 1.8.9.1}\\
\end{center}

\newpage
\section*{Введение}
\subsection{Область применения}
\subsection{}
\subsection{}
\subsection{}

\section{Назначение и условия применения}
\subsection{}
\subsection{}


\section{Подготовка к работе}
\subsection{}
\subsection{}
\subsection{}

\section{Описание операций}
\subsection{}
\subsection{}

\section{Аварийные ситуации}

\section{Рекомендации по освоению}






 
\newpage
\section*{Заключение}
\addcontentsline{toc}{section}{Заключение}
В ходе дипломного проектирования были выполнены все поставленные задачи:
\begin{itemize}
 \item проведен обзор актуальных информационных источников в области деанонимизации
 авторов программного обеспечения по его исходному коду;
 \item построена модель процесса определения авторства исходного кода;
 \item сформирован и обработан набор данных, состоящий из трех подвыборок, имеющих характерные особенности,
 которые оказывают влияние на точность классификации;
 \item произведена программная реализация разработанной модели и ее тестирование;
 \item разработан интерфейс на основе технологии Jupyter Notebook;
 \item проведены вычислительные эксперименты;
 \item произведен анализ полученных результатов;
 \item приведено технико-экономическое обоснование работы;
 \item рассмотрены вопросы безопасности жизнедеятельности.
\end{itemize}

Итогом дипломного проектирования является разработанное программное обеспечение <<WhoseCppCode>> на языке
программирования Python, предназначенное для сбора и обработки данных,
классификации авторов исходного кода на языке C/C++,
визуализации полученных результатов при помощи интерфейса. Основной программный модуль может быть использован
отдельно при разработке различных систем, интерфейсов и программ, а также дальнейших
научных исследований задачи деанонимизации авторов исходного кода. 



 
 
 \newpage
 \renewcommand{\refname}{Список использованных источников}
 \addcontentsline{toc}{section}{Список использованных источников}
 \bibliography{lit}

 \ESKDappendix{Обязательное}{\normalfont Компакт-диск}
 Компакт-диск содержит: 
 \begin{itemize}
 \item электронную версию пояснительной записки в форматах *.tex и *.pdf;
 \item итоговую презентацию результатов работы в форматах *.pptx и *.pdf;
 \item актуальную версию программы, реализованную на языке программирования Python, для определения авторсва исходного кода программ на языке C/C++;
 \item тестовые данные для работы с программой.
 \end{itemize}
 
%  \ESKDappendix{Справочное}{\normalfont Сравнительный обзор информационных источников }
 \newpage
 \addcontentsline{toc}{section}{Приложение Б Сравнительный обзор информационных источников}
 \includepdf[pages={1-2}]{overview_table}
%  \includepdf[pages={1-}, width=1\linewidth]{overview_table}
 
%  \begin{table}[h!]
\caption{ Обзор источников }
\label{tab:results}
\begin{center}
\begin{tabularx}{\linewidth}{|X|X|X|X|X|X|}
\hline
Название & Авторы, год & Методы & Данные & Точность & Язык\\
\hline
Using classification techniques to determine source code authorship~\cite{pellin} & Pellin, 2008 & АСТ, SVM & 4 схожие программы, 2 автора & 67 -- 68\% & Java\\
\hline
Source code authorship attribution using n-grams~\cite{burrows} & Burrows, Tahaghoghi, 2007 & N-граммы & 1640 файлов исходного кода, 100 авторов & 67\% & C\\
\hline
Identifying Authorship by Byte-Level N-Grams~\cite{frantz_2} & Frantzeskou, Stamatatos, Gritzalis, 2007 & Профиль программиста на основе статистических метрик & Не указано & 88\% для С++, 100\% для Java & Java, C++\\
\hline
Application of information retrieval tech- niques for source code authorship attribution~\cite{burrows_3} & Burrows, Uitdenbogerd, Urpin, 2009 & N-граммы, рейтинговые схемы & 100 авторов, классифицировались по 10, 1579 программных файлов & 77\% & C\\
\hline
De-anonymi- zing Program- mers via Code Stylometry~\cite{caliskan} & Caliskan-Islam, Harang, Liu 2015 & Статистичес- кий подсчет признаков, нечеткие АСТ & 250 авторов, 1600 файлов & 94 -- 98\% & C/C++, Python \\
\hline
Git Blame Who?: Stylistic Authorship Attribution of Small, Incomplete Source Code Fragments~\cite{git_blame} & Caliskan-Islam, Dauber, Harang, Greenstadt, 2017 & Калибровоч- ные кривые, нечеткие АСТ, Random Forest & Некомпилиру- емые неполные образцы кода с ресурса GitHub & 70 -- 100\% & C/C++\\
\hline
\end{tabularx}
\end{center}
\end{table}
 
%  \ESKDappendix{Справочное}{\normalfont Описание стилистических признаков}
 \newpage
 \addcontentsline{toc}{section}{Приложение В Описание стилистических признаков}
 \includepdf[pages={1-2}]{lexical_features}
 
\end{document}
