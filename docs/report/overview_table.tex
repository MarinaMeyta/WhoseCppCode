\begin{table}[h!]
\caption{ Обзор источников }
\label{tab:results}
\begin{center}
\begin{tabularx}{\linewidth}{|X|X|X|X|X|X|}
\hline
Название & Авторы, год & Методы & Данные & Точность & Язык\\
\hline
Using classification techniques to determine source code authorship~\cite{pellin} & Pellin, 2008 & АСТ, SVM & 4 схожие программы, 2 автора & 67 -- 68\% & Java\\
\hline
Source code authorship attribution using n-grams~\cite{burrows} & Burrows, Tahaghoghi, 2007 & N-граммы & 1640 файлов исходного кода, 100 авторов & 67\% & C\\
\hline
Identifying Authorship by Byte-Level N-Grams~\cite{frantz_2} & Frantzeskou, Stamatatos, Gritzalis, 2007 & Профиль программиста на основе статистических метрик & Не указано & 88\% для С++, 100\% для Java & Java, C++\\
\hline
Application of information retrieval tech- niques for source code authorship attribution~\cite{burrows_3} & Burrows, Uitdenbogerd, Urpin, 2009 & N-граммы, рейтинговые схемы & 100 авторов, классифицировались по 10, 1579 программных файлов & 77\% & C\\
\hline
De-anonymi- zing Program- mers via Code Stylometry~\cite{caliskan} & Caliskan-Islam, Harang, Liu 2015 & Статистичес- кий подсчет признаков, нечеткие АСТ & 250 авторов, 1600 файлов & 94 -- 98\% & C/C++, Python \\
\hline
Git Blame Who?: Stylistic Authorship Attribution of Small, Incomplete Source Code Fragments~\cite{git_blame} & Caliskan-Islam, Dauber, Harang, Greenstadt, 2017 & Калибровоч- ные кривые, нечеткие АСТ, Random Forest & Некомпилиру- емые неполные образцы кода с ресурса GitHub & 70 -- 100\% & C/C++\\
\hline
\end{tabularx}
\end{center}
\end{table}