Критерии оценки работы классификатора~\cite{metrics} представлены в таблице~\ref{tab:eval}, где:
\begin{itemize}
  \item tp --- истинно-положительное решение;
  \item tn --- истинно-отрицательное решение;
  \item fp --- ложно-положительное решение;
  \item fn --- ложно-отрицательное решение;
  \item accuracy (точность) --- отношение количества документов, по которым классификатор принял правильное решение, к общему числу документов (примеров файлов исходного кода); 
  \item precision (правильность) --- доля документов, действительно принадлежащих данному классу, относительно всех документов, которые система отнесла к этому классу;
  \item recall (полнота) --- доля найденных классфикатором документов, принадлежащих классу, относительно всех документов этого класса в тестовой выборке;
  \item f1-score (f1-мера) --- гармоническое среднее между правильностью и полнотой.
\end{itemize}

\begin{table}[h!]
\caption{ Критерии оценки работы классификатора }
\label{tab:eval}
\begin{center}
\begin{tabularx}{\linewidth}{|c|c|X|X|}
\hline
Критерий & Формула & Луч. знач. & Худ. знач. \\
\hline
Accuracy (точность) & (tp + tn) / число примеров * 100 \% & 100 \% & 0 \% \\
\hline
Precision (правильность) & tp / (tp + fp) & 1 & 0 \\
\hline
Recall (полнота) & tp / (tp + fn) & 1 & 0 \\
\hline
F1-score (F1-мера) & 2 * (precision * recall) / (precision + recall) & 1 & 0 \\
\hline
\end{tabularx}
\end{center}
\end{table}
