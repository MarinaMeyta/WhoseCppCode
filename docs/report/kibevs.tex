Местом прохождения практики была выбрана кафедра ТУСУРа – КИБЭВС (Кафедра комплексной информационной безопасности электронно-
вычислительных систем).

Согласно информации с официального сайта~\cite{kibevs}, кафедра организована в ТУСУР в 1971 году как кафедра «Конструирования и производства электронно-вычислительной аппаратуры»
(КиПЭВА) вскоре переименованной в кафедру «Конструирования электронно-вычислительной аппаратуры» (КЭВА). 

21 сентября 1999 г. в связи с открытием новой актуальной специальности 090105 – «Комплексное обеспечение информационной
безопасности автоматизированных систем» кафедра КЭВА была переименована в кафедру «Комплексной информационной безопасности
электронно-вычислительных систем» (КИБЭВС). Заведующим кафедрой КИБЭВС, а также ректором ТУСУРа на сегодняшний день является ректор
ТУСУРа, Александр Александрович Шелупанов, лауреат премии Правительства Российской Федерации, действительный член 
Международной Академииnнаук высшей школы РФ, действительный член Международной Академии информации, 
Почетный работник высшего профессионального образования РФ, заместитель Председателя Сибирского регионального 
отделения учебно-методического объединения вузов России по образованию в области информационной безопасности, профессор, доктор технических наук.

Кадровый состав непрерывно укреплялся с момента её образования в 1971 году. 
В 2011 году, в год 40-летия кафедры, её коллектив состоял из 53 человек, в их числе 34 опытных 
высококвалифицированных специалиста и 19 аспирантов. Среди сотрудников кафедры  члены Академий наук РФ; 4
профессора; 12 доцентов, кандидатов наук; старшие научные сотрудники, кандидаты наук и др.

С 2008 г. кафедра КИБЭВС входит в состав Института «Системной интеграции и безопасности».

На базе кафедры КИБЭВС ТУСУР в 2002 году организовано «Сибирское региональное отделение учебно-методического объединения
Вузов России по образованию в области информационной безопасности.

