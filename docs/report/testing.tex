\subsection{Объект испытаний}

\subsection{Цель испытаний}

\subsection{Требования к программе}

\subsection{Требования к программной документации}

\subsection{Средства и порядок испытаний}

\subsection{Методы испытаний}

Для тестирования аналитической модели в машинном обучении применяется процедура скользящего
контроля, получившая название кросс-валидации (cross-validation) или перекрестной проверки.~\cite{crossval}

Процедура кросс-валидации включает в себя случайное разбиение на k подгрупп (или фолдов) примерно
одинакового размера. Первый фолд служит для тестирования модели, остальные используются для обучения
классификатора. Для тестовой подвыборки вычисляется среднеквадратичное отклонение. Процедура повторяется k-1
раз, при этом каждая из подгрупп выступает в роли тестовой выборки.

В данной работе тестирование производилось с применением 10-фолдовой кросс-валидации. Всего было произведено
10 вычислительных экспериментов.
