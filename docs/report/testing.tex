Раздел <<Программа и методика испытаний>> был составлен и оформлен в соответствии с ГОСТ 19.301--79.~\cite{gost_19301} 

\subsection{Объект испытаний}
\subsubsection{Полное наименование системы и ее условное обозначение}

Условное обозначение: <<WhoseCppCode>>.

\subsubsection{Область применения}

Разработанный программный комплекс можно использовать


\subsection{Цель испытаний}

Испытания системы предназначены для оценки адекватности модели, ее точности

\subsection{Требования к программе}

\subsection{Требования к программной документации}

Пояснительная записка к дипломной работе должна включать в себя:

\begin{itemize}
 \item задание по дипломному проектированию;
 \item руководство администратора (приложение~Г);
 \item руководство программиста (приложение~Д);
 \item руководство пользователя (приложение~Е):
 \item результаты вычислительных экспериментов.
\end{itemize}

\textbf{руководство пользователя должно быть оформлено 
Согласно ГОСТ 19.505-79 ЕСПД. Руководство оператора. Требования к содержанию и оформлению
http://techwrconsult.com/library/19.505

программиста - ГОСТ 19.504-79 РУКОВОДСТВО ПРОГРАММИСТА. 
ТРЕБОВАНИЯ К СОДЕРЖАНИЮ И ОФОРМЛЕНИЮ
http://www.rugost.com/index.php?option=com_content&view=article&id=64:19504-79&catid=19&Itemid=50}



\subsection{Средства и порядок испытаний}
\subsubsection{Технические и программные средства, используемые во время испытаний}



\subsubsection{Порядок проведения испытаний}


\subsection{Методы испытаний}\label{testing_methods}

Для тестирования аналитической модели в машинном обучении применяется процедура скользящего
контроля, получившая название кросс-валидации (cross-validation) или перекрестной проверки.~\cite{crossval}

Процедура кросс-валидации включает в себя случайное разбиение на k подгрупп (или фолдов) примерно
одинакового размера. Первый фолд служит для тестирования модели, остальные используются для обучения
классификатора. Для тестовой подвыборки вычисляется среднеквадратичное отклонение. Процедура повторяется k-1
раз, при этом каждая из подгрупп выступает в роли тестовой выборки.

В данной работе тестирование производилось с применением 10-фолдовой кросс-валидации. Всего было произведено
10 вычислительных экспериментов.

