Раздел <<Программа и методика испытаний>> был составлен и оформлен в соответствии с ГОСТ 19.301--79.~\cite{gost_19301} 

\subsection{Объект испытаний}
\subsubsection{Полное наименование системы и ее условное обозначение}

Условное обозначение: <<WhoseCppCode>>.

Область применения

Разработанный программный комплекс можно использовать


\subsection{Цель испытаний}

Испытания системы предназначены для оценки адекватности модели, ее точности

\subsection{Требования к программе}

\subsection{Требования к программной документации}

Пояснительная записка к дипломному 

\subsection{Средства и порядок испытаний}

\subsection{Методы испытаний}

Для тестирования аналитической модели в машинном обучении применяется процедура скользящего
контроля, получившая название кросс-валидации (cross-validation) или перекрестной проверки.~\cite{crossval}

Процедура кросс-валидации включает в себя случайное разбиение на k подгрупп (или фолдов) примерно
одинакового размера. Первый фолд служит для тестирования модели, остальные используются для обучения
классификатора. Для тестовой подвыборки вычисляется среднеквадратичное отклонение. Процедура повторяется k-1
раз, при этом каждая из подгрупп выступает в роли тестовой выборки.

В данной работе тестирование производилось с применением 10-фолдовой кросс-валидации. Всего было произведено
10 вычислительных экспериментов.

