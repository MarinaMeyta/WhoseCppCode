Раздел <<Программа и методика испытаний>> описывает тестирование

был составлен и оформлен в соответствии с ГОСТ 19.301--79.~\cite{gost_19301} 

\subsection{Объект испытаний}
\subsubsection{Полное наименование системы и ее условное обозначение}

Условное обозначение: <<WhoseCppCode>>.

\subsubsection{Область применения}

Разработанный программный комплекс можно использовать


\subsection{Цель испытаний}

Испытания системы предназначены для оценки адекватности модели, ее точности

\subsection{Требования к программе}

\subsection{Требования к программной документации}

Пояснительная записка к дипломной работе должна включать в себя:

\begin{itemize}
 \item задание по дипломному проектированию;
 \item руководство администратора (приложение~Г);
 \item руководство программиста (приложение~Д);
 \item руководство пользователя (приложение~Е):
 \item результаты вычислительных экспериментов.
\end{itemize}

руководство пользователя должно быть оформлено согласно~\cite{gost_19.505} 

программиста --- согласно~\cite{gost_19.504}  



\subsection{Средства и порядок испытаний}
\subsubsection{Технические и программные средства, используемые во время испытаний}



\subsubsection{Порядок проведения испытаний}


\subsection{Методы испытаний}\label{testing_methods}

В качестве методики тестирования применялась процедура кросс-валидации, описанная в разделе~\ref{crossval}.

