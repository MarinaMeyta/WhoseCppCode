В ходе преддипломной практики были выполнены все поставленные задачи:
\begin{itemize}
 \item проведен обзор актуальных на сегодняшний день информационных источников в области деанонимизации
 авторов программного обеспечения по его исходному коду;
 \item построена модель процесса определения авторства исходного кода;
 \item сформирован и обработан набор данных, состоящий из трех подвыборок, имеющих характерные особенности,
 которые оказывают влияние на точность классификации;
 \item произведена программная реализация разработанной модели и ее тестирование;
 \item разработан интерфейс на основе технологии Jupyter Notebook;
 \item проведены вычислительные эксперименты и анализ полученных результатов.
\end{itemize}

Итогом практики является разработанное программное обеспечение <<WhoseCppCode>> на языке
программирования Python, предназначенное для сбора и обработки данных,
классификации авторов исходного кода на языке C/C++,
визуализации полученных результатов при помощи интерфейса. Основной программный модуль может быть использован
отдельно при разработке различных систем, интерфейсов и программ, а также дальнейших
научных исследований задачи деанонимизации авторов исходного кода. 


