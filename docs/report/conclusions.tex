По результатам, полученным в ходе научно-исследовательской работы, можно сделать следующие выводы:
\begin{enumerate}
  \item Тестовые данные (см. раздел~\ref{test_data}) имели определенные недостатки, которые в конечном итоге привели к снижению точности классификации:
  \begin{itemize}
    \item программы, на которых производилось обучение и тестирование классификатора, были написаны студентами в ходе изучения основ программирования, т.е. у авторов в выборке отсутствовал опыт программирования на С/С++;
    \item среднее количество строк кода на файл составило всего 45 строк;
    \item в основном примеры содержали множество конструкций ввода-вывода и простые расчеты (например, площадей различных геометрических фигур);
    \item задания выполнялись по примерам из методического обеспечения, что повлияло на предпочтение использования тех или иных конструкций языка.
  \end{itemize}
  \item Не все признаки, используемые в классификации, равнозначны. Так, например, частоты ключевых слов <<int>> или <<float>> (см. раздел~\ref{keycpp}) не так важны для определения стилистических особенностей написания программы, как, к примеру, определенные предпочтения автора при назначении идентификаторов.
  \item Оптимальные параметры для задач классификации с помощью алгоритма Random Forest (см. раздел~\ref{random_forest}) были подобраны эмпирически (см. раздел~\ref{trees_feat}) и соответствовали рекомендациям из различных источников.
\end{enumerate}

В ходе работы был построен классификатор, позволяющий отнести исходный код к конктретному автору с точностью около 73\%.

Среди задач на будущее можно выделить следующие:
\begin{itemize}
  \item расширение набора признаков, а также расчет (с использованием экспертной оценки) веса для каждого из них (см. раздел~\ref{features}) для повышения точности классификации;
  \item подбор тестовых данных более сложных программ для большего количества авторов; при этом желательно, чтобы авторы программ имели некоторый опыт программирования на языке С/С++;
  \item исследование набора признаков для программ на других языках высокого уровня;
  \item исследование алгоритмов обнаружения плагиата в исходных кодах программ.
\end{itemize}

