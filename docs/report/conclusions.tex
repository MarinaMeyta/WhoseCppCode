В ходе преддипломной практики были выполнены все поставленные задачи:
\begin{itemize}
 \item проведен обзор актуальных на сегодняшний день информационных источников в области деанонимизации
 авторов программного обеспечения по его исходному коду;
 \item выявлен набор стилистических признаков, способных идентифицировать авторов программ на языке программирования
 C/C++
 \item сформирован и обработан набор данных, состоящий из трех подвыборок, имеющих характерные особенности,
 которые оказывают влияние на точность классификации;
 \item произведена программная реализация разработанной модели, ее тестирование и оценка;
 \item разработан интерфейс на основе технологии Jupyter Notebook;
 \item проведены вычислительные эксперименты и анализ полученных результатов.
\end{itemize}

Итогом практики является разработанное программное обеспечение <<WhoseCppCode>>, предназначенное для построения,
тестирования и оценки модели классификации авторов исходного кода программ на языке C/C++,
визуализации полученных результатов. Данное программное обеспечение может быть использовано самостоятельно
или в качестве программного модуля при разработке различных систем, интерфейсов и программ, а также дальнейших
научных исследований задачи деанонимизации авторов исходного кода. 


