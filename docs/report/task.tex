 \newpage
 \ESKDstyle{empty}

\begin{center}
 \textbf{МИНИСТЕРСТВО ОБРАЗОВАНИЯ И НАУКИ РОССИЙСКОЙ ФЕДЕРАЦИИ}\\
 Федеральное государственное бюджетное образовательное учреждение высшего образования\\
 <<ТОМСКИЙ ГОСУДАРСТВЕННЫЙ УНИВЕРСИТЕТ СИСТЕМ УПРАВЛЕНИЯ И РАДИОЭЛЕКТРОНИКИ>> (ТУСУР)\\
 Кафедра комплексной информационной безопасности электронно-вычислительных систем (КИБЭВС)\\
\end{center} 

\vfill

\begin{flushright}
\begin{minipage}{0.45\textwidth}
 \begin{flushleft}
  УТВЕРЖДАЮ\\
  заведующий каф. КИБЭВС\\
  д-р техн. наук, проф.\\
  \underline{\hspace{3cm}}А.А. Шелупанов \\
  <<\underline{\hspace{1cm}}>>\underline{\hspace{3cm}}2017г.\\
 \end{flushleft}
\end{minipage}
\end{flushright}

\vfill



\begin{center}
ЗАДАНИЕ

на выполнение выпускной квалификационной работы

студентке Мейта Марине Валерьевне

группы 722 факультета безопасности
\end{center}


1. Тема проекта (работы): Определение авторства исходного кода.

2. Срок сдачи студентом законченного проекта:
<<\underline{\hspace{1cm}}>>\underline{\hspace{3cm}}2017г.

3. Исходные данные к проекту:
\begin{itemize}
 \item научно-техническая литература по методам машинного обучения,
определения авторства текстов, анализу исходного кода;
 \item программа на языке программирования Python, разработанная в ходе
научно-исследовательской работы по аналогичной теме.
\end{itemize}

4. Содержание расчетно-пояснительной записки:
\begin{itemize}
 \item обзор существующих исследований, разработок, методов стилометрического анализа исходного кода программ;
 \item разработка программного обеспечения для анализа исходного кода программ с применением стилометрии для определения авторства программного обеспечения;
 \item подготовка и обработка тестового набора данных;
 \item исследование эффективности разработанной программы на основе модели анализа исходных кодов;
 \item анализ результатов;
 \item технико-экономическое обоснование;
 \item вопросы безопасности жизнедеятельности.
\end{itemize}

5. Перечень графического материала: 
\begin{itemize}
 \item презентация;
 \item примеры вывода результатов работы программы;
 \item UML-диаграммы, блок-схемы, графики, описывающие алгоритм работы программы.
\end{itemize}

6. Консультанты по проекту:
\vspace{0.01cm}
\begin{singlespace}
 \begin{minipage}[left]{0.40\linewidth}
 Консультант по экономике:\\
 ст. преподаватель каф. КИБЭВС \\

 Консультант по безопасности\\ жизнедеятельности:\\
 канд. техн. наук,\\ доцент каф. КИБЭВС\\
 \end{minipage}
  \hfill
 \begin{minipage}[left]{0.45\linewidth}
  \vspace{0.7cm}
 \underline{\hspace{2.5cm}}С.В. Глухарева \\
 <<\underline{\hspace{1cm}}>>\underline{\hspace{3cm}} 2017г.\\
 \vspace{0.3cm}\\ 
 \underline{\hspace{2.5cm}}Е.М. Давыдова\\
 <<\underline{\hspace{1cm}}>>\underline{\hspace{3cm}} 2017г.\\
 \end{minipage}
\end{singlespace}

7. Дата выдачи задания: 
<<\underline{\hspace{1cm}}>>\underline{\hspace{3cm}}2017г.

\begin{singlespace}
 \begin{minipage}[left]{0.40\linewidth}
 Руководитель: \\
  канд. техн. наук, \\доцент каф. БИС \\
  

 Задание приняла к исполнению:\\
 студентка гр. 722\\
 \end{minipage}
  \hfill
 \begin{minipage}[left]{0.45\linewidth}
  \vspace{0.7cm}
\underline{\hspace{3cm}} А.С. Романов \\
  <<\underline{\hspace{1cm}}>>\underline{\hspace{3cm}}2017г.\\
 \vspace{0.3cm}\\ 
  \underline{\hspace{3cm}}М.В. Мейта  \\
 <<\underline{\hspace{1cm}}>>\underline{\hspace{3cm}} 2017г.\\
 \end{minipage}
\end{singlespace}
