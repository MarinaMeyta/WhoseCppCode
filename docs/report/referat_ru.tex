
\newpage
\ESKDthisStyle{empty}
\paragraph{\hfill РЕФЕРАТ \hfill}
Отчет содержит \ESKDtotal{page} страницу, \ESKDtotal{figure} рисунков, 6 таблиц, \ESKDtotal{bibitem} источник,
6 приложений.

% \ESKDtotal{table}
% \ESKDtotal{appendix}

СТИЛОМЕТРИЯ, ИСХОДНЫЙ КОД, ДЕАНОНИМИЗАЦИЯ АВТОРА, C/C++, КЛАССИФИКАЦИЯ, PYTHON, SKLEARN, JUPYTER NOTEBOOK, DECISION
TREES, RANDOM FOREST CLASSIFIER, CROSS-VALIDATION, ADA BOOST, EXTREMLY RANDOMIZED TREES, GITHUB, LATEX.

Цель работы --- разработка программного обеспечения для определения авторства исходного кода программ
на на языке C/C++, основанного на методах стилометрического анализа текста, с перспективой его
дальнейшего применения в борьбе с киберпреступностью, в области лицензионных, патентных, и иных судебных разбирательств.

В рамках преддипломной практики были поставленны следующие задачи: 
\begin{itemize}
  \item обзор существующих исследований, разработок, методов стилометрического анализа тек-
ста, в том числе исходного кода программ;
\item построение модели процесса определения авторства исходного кода;
  \item разработка программного обеспечения для анализа исходного кода программ с применением стилометрии для
определения авторства программного обеспечения;
  \item разработка программного интерфейса;
  \item подготовка и обработка тестового набора данных;
  \item исследование эффективности разработанной программы на основе модели анализа исходных кодов, анализ результатов.
\end{itemize}

Объект исследования: деанонимизация автора программного обеспечения. 

Предмет исследования: стилометрия исходного кода программ на языках высокого уровня.

Достигнутые результаты: главным результатом преддипломной практики является 
программное обеспечение <<WhoseCppCode>>, предназначенное для построения,
тестирования и оценки модели классификации авторов исходного кода на языке C/C++,
а также визуализации полученных результатов.

% \begin{itemize}
%   \item произведен аналитический обзор существующих методов анализа исходного кода программ с целью деанонимизации автора;
%   \item выбран набор признаков для классификации авторов программного кода на языке C/С++; 
%   \item выбраны алгоритмы классификации;
%   \item построена модель на определения авторства исходного кода;
%   \item разработано программное обеспечение для определения определения авторства  на языке программирования высокого уровня Python;
%   \item подготовлен тестовый набор данных;
%   \item реализован программный интерфейс с использованием технологии Jupyter Notebook;~\cite{jupyter}
%   \item выбраны критерии оценки эффективности разработанного алгоритма;
%   \item произведены вычислительные эксперименты на данном наборе данных;
%   \item сделаны выводы на основе полученных результатов.
% \end{itemize}

Отчет по преддипломной практике выполнен согласно ОС ТУСУР 01-2013~\cite{ostusur} при помощи системы компьютерной вёрстки \LaTeX. 
