\newpage
\ESKDthisStyle{empty}
\paragraph{\hfill РЕФЕРАТ \hfill}
Отчет содержит \ESKDtotal{page} страниц, \ESKDtotal{figure} рисунков, \ESKDtotal{table} таблиц, \ESKDtotal{bibitem} источников, \ESKDtotal{appendix} приложение.

СТИЛОМЕТРИЯ, ИСХОДНЫЙ КОД, ДЕАНОНИМИЗАЦИЯ АВТОРА, C++, КЛАССИФИКАЦИЯ, PYTHON, SKLEARN, RANDOM FOREST CLASSIFIER, LATEX.

Цель работы --- разработка алгоритма для определения авторства программного обеспечения, основанного на стилометрическом анализе исходного кода программ на языках высокого уровня.

В рамках научно-исследовательской работы на текущий семестр были поставленны следующие задачи: 
\begin{itemize}
  \item обзор существующих исследований, разработок, методов стилометрического анализа тек-
ста, в том числе исходного кода программ;
  \item разработка алгоритм анализа исходного кода программ с применением стилометрии для
определения авторства программного обеспечения;
  \item создание программной реализации разработанного алгоритма;
  \item исследование эффективности разработанного алгоритма анализа исходных кодов.
\end{itemize}

Объект исследования: деанонимизация автора программного обеспечения. 

Предмет исследования: стилометрия исходного кода программ на языках высокого уровня.

В результате работы было выполнено слудующее:

\begin{itemize}
  \item произведен аналитический обзор существующих методов анализа исходного кода программ с целью деанонимизации автора;
  \item выбран набор признаков для классификации авторов программного кода на языке С++; 
  \item в качестве алгоритма классификации выбран Random Forest Classifier;
  \item разработан алгоритм определения авторства исходного кода программ на языке С++;
  \item произведена программная реализация разработанного алгоритма на языке программирования высокого уровня Python;
  \item подготовлен тестовый набор данных;
  \item выбраны критерии оценки эффективности разработанного алгоритма;
  \item произведены вычислительные эксперименты на данном наборе данных;
  \item сделаны некоторые выводы на основе полученных результатов.
\end{itemize}

Отчет по преддипломной практике выполнен согласно ОС ТУСУР 01-2013~\cite{ostusur} при помощи системы компьютерной вёрстки \LaTeX. 
