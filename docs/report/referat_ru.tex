\newpage
\ESKDthisStyle{empty}
\paragraph{\hfill РЕФЕРАТ \hfill}
Отчет содержит \ESKDtotal{page} страниц, \ESKDtotal{figure} рисунков, \ESKDtotal{table} таблиц, \ESKDtotal{bibitem} источников, \ESKDtotal{appendix} приложение.

СТИЛОМЕТРИЯ, ИСХОДНЫЙ КОД, ДЕАНОНИМИЗАЦИЯ АВТОРА, C/C++, КЛАССИФИКАЦИЯ, PYTHON, SKLEARN, JUPYTER NOTEBOOK, DECISION
TREES, RANDOM FOREST CLASSIFIER, ADA BOOST, EXTRA TREES, GITHUB, LATEX.

Цель работы --- разработка алгоритма для определения авторства программного обеспечения, основанного на стилометрическом анализе исходного кода программ на языках высокого уровня.

В рамках преддипломной практики были поставленны следующие задачи: 
\begin{itemize}
  \item обзор существующих исследований, разработок, методов стилометрического анализа тек-
ста, в том числе исходного кода программ;
  \item разработка алгоритм анализа исходного кода программ с применением стилометрии для
определения авторства программного обеспечения;
  \item создание программной реализации разработанного алгоритма;
  \item разработка программного интерфейса;
  \item подготовка и обработка тестового набора данных;
  \item исследование эффективности разработанного алгоритма анализа исходных кодов, анализ результатов.
\end{itemize}

Объект исследования: деанонимизация автора программного обеспечения. 

Предмет исследования: стилометрия исходного кода программ на языках высокого уровня.

Достигнутые результаты:

\begin{itemize}
  \item произведен аналитический обзор существующих методов анализа исходного кода программ с целью деанонимизации автора;
  \item выбран набор признаков для классификации авторов программного кода на языке C/С++; 
  \item выбраны алгоритмы классификации;
  \item разработан алгоритм определения авторства исходного кода программ на языке C/С++;
  \item произведена программная реализация разработанного алгоритма на языке программирования высокого уровня Python;
  \item подготовлен тестовый набор данных;
  \item реализован программный интерфейс с использованием технологии Jupyter Notebook;~\cite{jupyter}
  \item выбраны критерии оценки эффективности разработанного алгоритма;
  \item произведены вычислительные эксперименты на данном наборе данных;
  \item сделаны некоторые выводы на основе полученных результатов.
\end{itemize}

Отчет по преддипломной практике выполнен согласно ОС ТУСУР 01-2013~\cite{ostusur} при помощи системы компьютерной вёрстки \LaTeX. 
