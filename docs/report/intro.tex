С распространением применения компьютерных систем и сетей возросло и количество преступлений в информационной сфере. Существует множество разновидностей кибератак --- различные компьютерные вирусы, трояны, несанкционированное копирование данных с кредитных карт, DDoS-атаки и многое другое. Возможность деанонимизации авторов вредоносного программного обеспечения может внести существенный вклад в развитие компьютерной криминалистики.

Целью данной работы является исследование методов стилометрии --- статистического анализа текста для выявления его стилистических особенностей, а также методов машинного обучения для решения задачи деанонимизации разработчика по исходному коду программного обеспечения.

Определение авторства исходного кода представляет собой актуальную задачу в сфере информационной безопасности, лицензирования в области разработки программного обеспечения, а также может оказать существенную помощь во время судебных разбирательств, при решении вопросов об интеллектуальной собственности и плагиате.
