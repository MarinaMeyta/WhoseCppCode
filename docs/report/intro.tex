Задача определения авторства является широко распространенной проблемой в рамках исследования естественных языков, однако в меньшей степени --- для языков программирования. 
Тем не менее, с распространением применения компьютерных систем и сетей возросло и количество преступлений в 
информационной сфере. Существует множество разновидностей кибератак --- различные компьютерные вирусы, 
трояны, несанкционированное копирование данных с кредитных карт, DDoS-атаки и многое другое. 
Возможность деанонимизации авторов вредоносного программного обеспечения может внести существенный вклад 
в развитие компьютерной криминалистики.

Индивидуальный почерк развивается в течение всей жизни и при этом на него оказывает влияние множество факторов, таких как образование, семья, общество. 
Несмотря на то, что языки программирования могут показаться слишком формальными для проявления индивидуального авторского стиля, т.к.:
существуют стандарты кодирования (примеры стандартов?), описывающие рекомендации по комментированию, разметке кода, организации файлов, соглашения о наименовании переменных и т.п.
программы следуют строгим синтаксическим правилам, необходимым для обработки (компиляции, интерпретации, выполнению) программы компьютером

Определение авторства исходного кода представляет собой актуальную задачу в сфере информационной 
безопасности, лицензирования в области разработки программного обеспечения, а также может оказать 
существенную помощь во время судебных разбирательств, при решении вопросов об интеллектуальной 
собственности и плагиате.

Целью данной работы является исследование методов стилометрии --- статистического анализа текста 
для выявления его стилистических особенностей, а также методов машинного обучения для решения задачи 
деанонимизации разработчика по исходному коду программного обеспечения.
