Задача определения авторства является широко распространенной проблемой в рамках исследования 
естественных языков, однако в меньшей степени для языков программирования. 
Тем не менее, с распространением применения компьютерных систем и сетей возросло и количество преступлений в 
информационной сфере. Существует множество разновидностей кибератак --- различные компьютерные вирусы, 
трояны, несанкционированное копирование данных с кредитных карт, DDoS-атаки и многое другое. 
Возможность деанонимизации авторов вредоносного программного обеспечения может внести существенный вклад 
в развитие компьютерной криминалистики.

Считается, что у каждого программиста есть свои специфические профессиональные приемы, 
привычки, методы написания программного кода, свой так называемый <<стиль программирования>> и 
иные признаки, идентифицирующие автора. При этом, как и в случае с естественными языками, 
на индивидуальный <<почерк>> программиста может оказывать влияние
множество факторов, таких как образование, географическое место проживания, уровень квалификации и другие. 
<<Почерк>> также может изменяться с течением времени, развитием технологий и общепринятых норм <<хорошего>> стиля
написания программ. При этом под <<хорошим>> стилем обычно понимается набор правил, позволяющих писать 
код, удобный для чтения, понимания, внедрения дальнейших изменений и рефакторинга.
Крупные IT-компании и корпорации обычно вводят свои собственные стандарты кодирования, которые  
зачастую используются сторонними организациями и индивидуальными программистами в своей работе.
Например, руководства по стилю программирования на языке C++ компаний Google~\cite{google_style} и Geosoft~\cite{geosoft}.


Определение авторства исходного кода представляет собой актуальную задачу в сфере информационной 
безопасности, лицензирования в области разработки программного обеспечения, а также может оказать 
существенную помощь во время судебных разбирательств, при решении вопросов об интеллектуальной 
собственности и плагиате.

Целью преддипломной практики является разработка программного обеспечения для определения авторства 
исходного кода программ на языке C/C++, основанного на методах стилометрического анализа текста.
