Перечисленные в данном разделе признаки, по которым иднтифицировались авторы, являются характерными для языков C и С++, однако могут быть 
использованы для исследования C-подобных языков, например, D, Java, Objective C, C\#, PHP, perl и другие.

\subsection{Лексические признаки}

Главная особенность данной группы признаков состоит в том, что они могут быть вычислены при 
непосредственном анализе исходного кода программы в виде текстового файла. 
При этом код программы может быть некомпилируемым, неполным, содержащим
синтаксические или программные ошибки.

Лексические признаки, как правило, улучшают читаемость кода и включают в себя:

\begin{itemize}
 \item Стиль комментирования --- данная подгруппа определяет преобладающий в тексте вид комментариев 
 (однострочные или многострочные), а также общее их количество.
 \item Стиль расстановки фигурных скобок --- к наиболее известным относят <<K\&R>>, <<Whitesmith>>,
 <<One True Bracing Style>>, стиль Алмена и другие.~\cite{bracing_styles} 
 \item Стиль разметки --- расстановка пробелов, табуляций, число переносов строки к общей длине файла.
\end{itemize}


Дополнительно вычисляются:
\begin{itemize}
 \item Число макросов --- использует ли программист директивы препроцессора.\cite{macros}
 \item Средняя длина строки --- позволяет также оценить читаемость кода (слишком длинные программные файлы
 плохо воспринимаются человеком).
\end{itemize}


В приложении В приведено описание использованных при классификации авторов 
признаков.

\subsection{Ключевые слова С++}\label{keycpp}
Ключевые слова С++ представляют собой список зарезервированных последовательностей символов, 
используемых языком, недоступных для переопределения.

Для ключевых слов языка С++ вычислялась статистическая мера TF (term frequency), 
отображающая число вхождения некоторого ключевого слова к общему количеству слов в документе.
Подсчет частот ключевых слов может дать представление о предпочтениях автора в определенного рода конструкциях,
например, циклов <<for>> относитльно <<while>> или <<do while>>, 
а также об уровне его профессиональной квалификации (определенные конструкции языка C/C++ используются крайне
редко, сложны для понимания и предназначены для решения узкоспециализированных задач).

Словарь из 84 ключевых слов С++ (стандарт 11) был взят на сайте с официальной документацией~\cite{cppkeywords} 
и представлен в таблице~\ref{tab:3}.

\begin{table}[ht]
\caption{ Ключевые слова языка С++ (стандарт 11) }
\label{tab:3}
\begin{center}
\begin{tabularx}{\linewidth}{|X|X|X|X|X|X|}
\hline
\multicolumn{6}{|c|}{Ключевые слова языка С++} \\
\hline
alignas & char32\_t & enum & namespace & return & try\\
alignof & class & explicit & new & short & typedef\\
and & compl & export & noexcept & signed & typeid\\
and\_eq & const & extern & not & sizeof & typename\\
asm & constexpr & false & not\_eq & static & union\\
auto & const\_cast & float & nullptr & static\_assert & unsigned\\
bitand & continue & for & operator & static\_cast & using\\
bitor & decltype & friend & or & struct & virtual\\
bool & default & goto & or\_eq & switch & void\\
break & delete & if & private & template & volatile\\
case & do & inline & protected & this & wchar\_t\\
catch & double & int & public & thread\_local & while\\
char & dynamic\_cast & long & register & throw & xor\\
char16\_t & else & mutable & reinterpret & true & xor\_eq\\
\hline
\end{tabularx}
\end{center}
\end{table}