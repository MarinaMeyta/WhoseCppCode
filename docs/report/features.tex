Признаки, использованные в ходе работы для идентификации автора программы, можно разделить на три группы: лексические, синтаксические и отдельно --- частоты ключевых слов языка С++.

\subsection{Лексические признаки}

Лексические признаки могут быть вычислены при непосредственном анализе исходного кода программы (в виде текстового файла). В таблице~\ref{tab:1} приведено описание использованных при классификации авторов признаков.

\begin{table}[h!]
\caption{ Лексические признаки }
\label{tab:1}
\begin{center}
\begin{tabularx}{\linewidth}{|X|X|X|}
\hline
\multicolumn{3}{|c|}{Лексические признаки} \\
\hline
\multicolumn{1}{|c|}{Признак} & \multicolumn{1}{|c|}{Обозначение} & \multicolumn{1}{|c|}{Определение} \\
\hline
Число комментариев & ln\_comments & Натуральный логарифм отношения числа комментариев к длине файла в символах \\
\hline
Число макросов & ln\_macros & Натуральный логарифм отношения числа макросов к длине файла в символах \\
\hline
Число пробелов & ln\_spaces & Натуральный логарифм отношения числа пробелов к длине файла в символах \\
\hline
Число символов табуляции & ln\_tabs & Натуральный логарифм отношения числа символов табуляции к длине файла в символах \\
\hline
Число переводов строки & ln\_newlines & Натуральный логарифм отношения числа переводов строки к длине файла в символах \\
\hline
Коэффициент пробельных символов & whitespace\_ratio & Натуральный логарифм отношения суммы всех пробельных символов (пробелов, символов табуляции, переводов строки) к длине файла в символах \\
\hline
Число строк кода & lines\_of\_code & Число строк кода, не включающее пустые строки\\
\hline
\end{tabularx}
\end{center}
\end{table}

\subsection{Синтаксические признаки}

К синтаксическим признакам относятся характеристики кода, которые могут быть вычислены только с использованием синтаксического анализатора --- программы, позволяющей анализировать конструкции языка С/С++ и выполнять построение абстрактных синтаксических деревьев.

Синтаксические признаки (табл.~\ref{tab:2}) позволяют определить предпочтения программиста при назначении идентификаторов --- последовательности символов, используемой для обозначения имен объектов, переменных, классов, функций и т.д.~\cite{identif}

\begin{table}[h!]
\caption{ Синтаксические признаки }
\label{tab:2}
\begin{center}
\begin{tabularx}{\linewidth}{|X|X|X|}
\hline
\multicolumn{3}{|c|}{Синтаксические признаки} \\
\hline
\multicolumn{1}{|c|}{Признак} & \multicolumn{1}{|c|}{Обозначение} & \multicolumn{1}{|c|}{Определение} \\
\hline
Число функций & ln\_number\_of\_functions & Натуральный логарифм отношения числа функций к длине файла в символах \\
\hline
Длина имени функции &  avg\_funcname\_len & Средняя длина имени функции \\
\hline
Длина имени переменной &  avg\_varname\_len & Средняя длина имени переменной \\
\hline
Специальные символы &  has\_specialcharnames & Факт наличия специальных символов в идентификаторах \\
\hline
Заглавные буквы &  has\_uppercasenames & Факт наличия заглавных букв в идентификаторах \\
\hline
\end{tabularx}
\end{center}
\end{table}

\subsection{Ключевые слова С++}\label{keycpp}

Ключевые слова С++ представляют собой список зарезервированных последовательностей символов, используемых языком, недоступных для переопределения.

Для ключевых слов языка С++ вычислялась статистическая мера TF (term frequency), отображающая число вхождения некоторого слова к общему количеству слов в документе. Словарь из 84 ключевых слов С++ (стандарт 11) был взят на сайте с официальной документацией~\cite{cppkeywords} и представлен в таблице~\ref{tab:3}.

\begin{table}[ht]
\caption{ Ключевые слова языка С++ (стандарт 11) }
\label{tab:3}
\begin{center}
\begin{tabularx}{\linewidth}{|X|X|X|X|X|X|}
\hline
\multicolumn{6}{|c|}{Ключевые слова языка С++} \\
\hline
alignas & char32\_t & enum & namespace & return & try\\
alignof & class & explicit & new & short & typedef\\
and & compl & export & noexcept & signed & typeid\\
and\_eq & const & extern & not & sizeof & typename\\
asm & constexpr & false & not\_eq & static & union\\
auto & const\_cast & float & nullptr & static\_assert & unsigned\\
bitand & continue & for & operator & static\_cast & using\\
bitor & decltype & friend & or & struct & virtual\\
bool & default & goto & or\_eq & switch & void\\
break & delete & if & private & template & volatile\\
case & do & inline & protected & this & wchar\_t\\
catch & double & int & public & thread\_local & while\\
char & dynamic\_cast & long & register & throw & xor\\
char16\_t & else & mutable & reinterpret\_cast & true & xor\_eq\\
\hline
\end{tabularx}
\end{center}
\end{table}
