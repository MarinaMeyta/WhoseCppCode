Ниже преведены определения основных понятий, использованных в настоящем документе:
\begin{itemize}
 \item исходный код --- текстовый файл, содержащий текст компьютерной программы на каком-
 либо языке программирования и визуально понятный человеку;
 \item стилометрия --- исследование стилистики, включающее статистический анализ текста на предмет признаков,
 характеризующих индивидуальный стиль автора;
 \item классификация --- процесс отнесения исследуемой сущности к какому-либо заданному классу;
 \item кросс-валидация --- процедура, применяемая в машинном обучении для проверки, насколько успешно
 построенная аналитическая модель может работать на независимом наборе данных;
 \item open-source --- концепция свободного (открытого) программного обеспечения, т.е. программ с исходным кодом,
 открытым для использования и изменения сторонними разработчиками;
 \item репозиторий --- удаленное хранилище файлов исходного кода и данных;
 \item контрибьютор --- пользователь, вносивший изменения в open-source проект. 
\end{itemize}
